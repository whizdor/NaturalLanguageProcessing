\documentclass[11pt]{article}

    \usepackage[breakable]{tcolorbox}
    \usepackage{parskip} % Stop auto-indenting (to mimic markdown behaviour)
    
    \usepackage{iftex}
    \ifPDFTeX
    	\usepackage[T1]{fontenc}
    	\usepackage{mathpazo}
    \else
    	\usepackage{fontspec}
    \fi

    % Basic figure setup, for now with no caption control since it's done
    % automatically by Pandoc (which extracts ![](path) syntax from Markdown).
    \usepackage{graphicx}
    % Maintain compatibility with old templates. Remove in nbconvert 6.0
    \let\Oldincludegraphics\includegraphics
    % Ensure that by default, figures have no caption (until we provide a
    % proper Figure object with a Caption API and a way to capture that
    % in the conversion process - todo).
    \usepackage{caption}
    \DeclareCaptionFormat{nocaption}{}
    \captionsetup{format=nocaption,aboveskip=0pt,belowskip=0pt}

    \usepackage[Export]{adjustbox} % Used to constrain images to a maximum size
    \adjustboxset{max size={0.9\linewidth}{0.9\paperheight}}
    \usepackage{float}
    \floatplacement{figure}{H} % forces figures to be placed at the correct location
    \usepackage{xcolor} % Allow colors to be defined
    \usepackage{enumerate} % Needed for markdown enumerations to work
    \usepackage{geometry} % Used to adjust the document margins
    \usepackage{amsmath} % Equations
    \usepackage{amssymb} % Equations
    \usepackage{textcomp} % defines textquotesingle
    % Hack from http://tex.stackexchange.com/a/47451/13684:
    \AtBeginDocument{%
        \def\PYZsq{\textquotesingle}% Upright quotes in Pygmentized code
    }
    \usepackage{upquote} % Upright quotes for verbatim code
    \usepackage{eurosym} % defines \euro
    \usepackage[mathletters]{ucs} % Extended unicode (utf-8) support
    \usepackage{fancyvrb} % verbatim replacement that allows latex
    \usepackage{grffile} % extends the file name processing of package graphics 
                         % to support a larger range
    \makeatletter % fix for grffile with XeLaTeX
    \def\Gread@@xetex#1{%
      \IfFileExists{"\Gin@base".bb}%
      {\Gread@eps{\Gin@base.bb}}%
      {\Gread@@xetex@aux#1}%
    }
    \makeatother

    % The hyperref package gives us a pdf with properly built
    % internal navigation ('pdf bookmarks' for the table of contents,
    % internal cross-reference links, web links for URLs, etc.)
    \usepackage{hyperref}
    % The default LaTeX title has an obnoxious amount of whitespace. By default,
    % titling removes some of it. It also provides customization options.
    \usepackage{titling}
    \usepackage{longtable} % longtable support required by pandoc >1.10
    \usepackage{booktabs}  % table support for pandoc > 1.12.2
    \usepackage[inline]{enumitem} % IRkernel/repr support (it uses the enumerate* environment)
    \usepackage[normalem]{ulem} % ulem is needed to support strikethroughs (\sout)
                                % normalem makes italics be italics, not underlines
    \usepackage{mathrsfs}
    

    
    % Colors for the hyperref package
    \definecolor{urlcolor}{rgb}{0,.145,.698}
    \definecolor{linkcolor}{rgb}{.71,0.21,0.01}
    \definecolor{citecolor}{rgb}{.12,.54,.11}

    % ANSI colors
    \definecolor{ansi-black}{HTML}{3E424D}
    \definecolor{ansi-black-intense}{HTML}{282C36}
    \definecolor{ansi-red}{HTML}{E75C58}
    \definecolor{ansi-red-intense}{HTML}{B22B31}
    \definecolor{ansi-green}{HTML}{00A250}
    \definecolor{ansi-green-intense}{HTML}{007427}
    \definecolor{ansi-yellow}{HTML}{DDB62B}
    \definecolor{ansi-yellow-intense}{HTML}{B27D12}
    \definecolor{ansi-blue}{HTML}{208FFB}
    \definecolor{ansi-blue-intense}{HTML}{0065CA}
    \definecolor{ansi-magenta}{HTML}{D160C4}
    \definecolor{ansi-magenta-intense}{HTML}{A03196}
    \definecolor{ansi-cyan}{HTML}{60C6C8}
    \definecolor{ansi-cyan-intense}{HTML}{258F8F}
    \definecolor{ansi-white}{HTML}{C5C1B4}
    \definecolor{ansi-white-intense}{HTML}{A1A6B2}
    \definecolor{ansi-default-inverse-fg}{HTML}{FFFFFF}
    \definecolor{ansi-default-inverse-bg}{HTML}{000000}

    % commands and environments needed by pandoc snippets
    % extracted from the output of `pandoc -s`
    \providecommand{\tightlist}{%
      \setlength{\itemsep}{0pt}\setlength{\parskip}{0pt}}
    \DefineVerbatimEnvironment{Highlighting}{Verbatim}{commandchars=\\\{\}}
    % Add ',fontsize=\small' for more characters per line
    \newenvironment{Shaded}{}{}
    \newcommand{\KeywordTok}[1]{\textcolor[rgb]{0.00,0.44,0.13}{\textbf{{#1}}}}
    \newcommand{\DataTypeTok}[1]{\textcolor[rgb]{0.56,0.13,0.00}{{#1}}}
    \newcommand{\DecValTok}[1]{\textcolor[rgb]{0.25,0.63,0.44}{{#1}}}
    \newcommand{\BaseNTok}[1]{\textcolor[rgb]{0.25,0.63,0.44}{{#1}}}
    \newcommand{\FloatTok}[1]{\textcolor[rgb]{0.25,0.63,0.44}{{#1}}}
    \newcommand{\CharTok}[1]{\textcolor[rgb]{0.25,0.44,0.63}{{#1}}}
    \newcommand{\StringTok}[1]{\textcolor[rgb]{0.25,0.44,0.63}{{#1}}}
    \newcommand{\CommentTok}[1]{\textcolor[rgb]{0.38,0.63,0.69}{\textit{{#1}}}}
    \newcommand{\OtherTok}[1]{\textcolor[rgb]{0.00,0.44,0.13}{{#1}}}
    \newcommand{\AlertTok}[1]{\textcolor[rgb]{1.00,0.00,0.00}{\textbf{{#1}}}}
    \newcommand{\FunctionTok}[1]{\textcolor[rgb]{0.02,0.16,0.49}{{#1}}}
    \newcommand{\RegionMarkerTok}[1]{{#1}}
    \newcommand{\ErrorTok}[1]{\textcolor[rgb]{1.00,0.00,0.00}{\textbf{{#1}}}}
    \newcommand{\NormalTok}[1]{{#1}}
    
    % Additional commands for more recent versions of Pandoc
    \newcommand{\ConstantTok}[1]{\textcolor[rgb]{0.53,0.00,0.00}{{#1}}}
    \newcommand{\SpecialCharTok}[1]{\textcolor[rgb]{0.25,0.44,0.63}{{#1}}}
    \newcommand{\VerbatimStringTok}[1]{\textcolor[rgb]{0.25,0.44,0.63}{{#1}}}
    \newcommand{\SpecialStringTok}[1]{\textcolor[rgb]{0.73,0.40,0.53}{{#1}}}
    \newcommand{\ImportTok}[1]{{#1}}
    \newcommand{\DocumentationTok}[1]{\textcolor[rgb]{0.73,0.13,0.13}{\textit{{#1}}}}
    \newcommand{\AnnotationTok}[1]{\textcolor[rgb]{0.38,0.63,0.69}{\textbf{\textit{{#1}}}}}
    \newcommand{\CommentVarTok}[1]{\textcolor[rgb]{0.38,0.63,0.69}{\textbf{\textit{{#1}}}}}
    \newcommand{\VariableTok}[1]{\textcolor[rgb]{0.10,0.09,0.49}{{#1}}}
    \newcommand{\ControlFlowTok}[1]{\textcolor[rgb]{0.00,0.44,0.13}{\textbf{{#1}}}}
    \newcommand{\OperatorTok}[1]{\textcolor[rgb]{0.40,0.40,0.40}{{#1}}}
    \newcommand{\BuiltInTok}[1]{{#1}}
    \newcommand{\ExtensionTok}[1]{{#1}}
    \newcommand{\PreprocessorTok}[1]{\textcolor[rgb]{0.74,0.48,0.00}{{#1}}}
    \newcommand{\AttributeTok}[1]{\textcolor[rgb]{0.49,0.56,0.16}{{#1}}}
    \newcommand{\InformationTok}[1]{\textcolor[rgb]{0.38,0.63,0.69}{\textbf{\textit{{#1}}}}}
    \newcommand{\WarningTok}[1]{\textcolor[rgb]{0.38,0.63,0.69}{\textbf{\textit{{#1}}}}}
    
    
    % Define a nice break command that doesn't care if a line doesn't already
    % exist.
    \def\br{\hspace*{\fill} \\* }
    % Math Jax compatibility definitions
    \def\gt{>}
    \def\lt{<}
    \let\Oldtex\TeX
    \let\Oldlatex\LaTeX
    \renewcommand{\TeX}{\textrm{\Oldtex}}
    \renewcommand{\LaTeX}{\textrm{\Oldlatex}}
    % Document parameters
    % Document title
    \title{C1\_W1\_Assignment}
    
    
    
    
    
% Pygments definitions
\makeatletter
\def\PY@reset{\let\PY@it=\relax \let\PY@bf=\relax%
    \let\PY@ul=\relax \let\PY@tc=\relax%
    \let\PY@bc=\relax \let\PY@ff=\relax}
\def\PY@tok#1{\csname PY@tok@#1\endcsname}
\def\PY@toks#1+{\ifx\relax#1\empty\else%
    \PY@tok{#1}\expandafter\PY@toks\fi}
\def\PY@do#1{\PY@bc{\PY@tc{\PY@ul{%
    \PY@it{\PY@bf{\PY@ff{#1}}}}}}}
\def\PY#1#2{\PY@reset\PY@toks#1+\relax+\PY@do{#2}}

\expandafter\def\csname PY@tok@w\endcsname{\def\PY@tc##1{\textcolor[rgb]{0.73,0.73,0.73}{##1}}}
\expandafter\def\csname PY@tok@c\endcsname{\let\PY@it=\textit\def\PY@tc##1{\textcolor[rgb]{0.25,0.50,0.50}{##1}}}
\expandafter\def\csname PY@tok@cp\endcsname{\def\PY@tc##1{\textcolor[rgb]{0.74,0.48,0.00}{##1}}}
\expandafter\def\csname PY@tok@k\endcsname{\let\PY@bf=\textbf\def\PY@tc##1{\textcolor[rgb]{0.00,0.50,0.00}{##1}}}
\expandafter\def\csname PY@tok@kp\endcsname{\def\PY@tc##1{\textcolor[rgb]{0.00,0.50,0.00}{##1}}}
\expandafter\def\csname PY@tok@kt\endcsname{\def\PY@tc##1{\textcolor[rgb]{0.69,0.00,0.25}{##1}}}
\expandafter\def\csname PY@tok@o\endcsname{\def\PY@tc##1{\textcolor[rgb]{0.40,0.40,0.40}{##1}}}
\expandafter\def\csname PY@tok@ow\endcsname{\let\PY@bf=\textbf\def\PY@tc##1{\textcolor[rgb]{0.67,0.13,1.00}{##1}}}
\expandafter\def\csname PY@tok@nb\endcsname{\def\PY@tc##1{\textcolor[rgb]{0.00,0.50,0.00}{##1}}}
\expandafter\def\csname PY@tok@nf\endcsname{\def\PY@tc##1{\textcolor[rgb]{0.00,0.00,1.00}{##1}}}
\expandafter\def\csname PY@tok@nc\endcsname{\let\PY@bf=\textbf\def\PY@tc##1{\textcolor[rgb]{0.00,0.00,1.00}{##1}}}
\expandafter\def\csname PY@tok@nn\endcsname{\let\PY@bf=\textbf\def\PY@tc##1{\textcolor[rgb]{0.00,0.00,1.00}{##1}}}
\expandafter\def\csname PY@tok@ne\endcsname{\let\PY@bf=\textbf\def\PY@tc##1{\textcolor[rgb]{0.82,0.25,0.23}{##1}}}
\expandafter\def\csname PY@tok@nv\endcsname{\def\PY@tc##1{\textcolor[rgb]{0.10,0.09,0.49}{##1}}}
\expandafter\def\csname PY@tok@no\endcsname{\def\PY@tc##1{\textcolor[rgb]{0.53,0.00,0.00}{##1}}}
\expandafter\def\csname PY@tok@nl\endcsname{\def\PY@tc##1{\textcolor[rgb]{0.63,0.63,0.00}{##1}}}
\expandafter\def\csname PY@tok@ni\endcsname{\let\PY@bf=\textbf\def\PY@tc##1{\textcolor[rgb]{0.60,0.60,0.60}{##1}}}
\expandafter\def\csname PY@tok@na\endcsname{\def\PY@tc##1{\textcolor[rgb]{0.49,0.56,0.16}{##1}}}
\expandafter\def\csname PY@tok@nt\endcsname{\let\PY@bf=\textbf\def\PY@tc##1{\textcolor[rgb]{0.00,0.50,0.00}{##1}}}
\expandafter\def\csname PY@tok@nd\endcsname{\def\PY@tc##1{\textcolor[rgb]{0.67,0.13,1.00}{##1}}}
\expandafter\def\csname PY@tok@s\endcsname{\def\PY@tc##1{\textcolor[rgb]{0.73,0.13,0.13}{##1}}}
\expandafter\def\csname PY@tok@sd\endcsname{\let\PY@it=\textit\def\PY@tc##1{\textcolor[rgb]{0.73,0.13,0.13}{##1}}}
\expandafter\def\csname PY@tok@si\endcsname{\let\PY@bf=\textbf\def\PY@tc##1{\textcolor[rgb]{0.73,0.40,0.53}{##1}}}
\expandafter\def\csname PY@tok@se\endcsname{\let\PY@bf=\textbf\def\PY@tc##1{\textcolor[rgb]{0.73,0.40,0.13}{##1}}}
\expandafter\def\csname PY@tok@sr\endcsname{\def\PY@tc##1{\textcolor[rgb]{0.73,0.40,0.53}{##1}}}
\expandafter\def\csname PY@tok@ss\endcsname{\def\PY@tc##1{\textcolor[rgb]{0.10,0.09,0.49}{##1}}}
\expandafter\def\csname PY@tok@sx\endcsname{\def\PY@tc##1{\textcolor[rgb]{0.00,0.50,0.00}{##1}}}
\expandafter\def\csname PY@tok@m\endcsname{\def\PY@tc##1{\textcolor[rgb]{0.40,0.40,0.40}{##1}}}
\expandafter\def\csname PY@tok@gh\endcsname{\let\PY@bf=\textbf\def\PY@tc##1{\textcolor[rgb]{0.00,0.00,0.50}{##1}}}
\expandafter\def\csname PY@tok@gu\endcsname{\let\PY@bf=\textbf\def\PY@tc##1{\textcolor[rgb]{0.50,0.00,0.50}{##1}}}
\expandafter\def\csname PY@tok@gd\endcsname{\def\PY@tc##1{\textcolor[rgb]{0.63,0.00,0.00}{##1}}}
\expandafter\def\csname PY@tok@gi\endcsname{\def\PY@tc##1{\textcolor[rgb]{0.00,0.63,0.00}{##1}}}
\expandafter\def\csname PY@tok@gr\endcsname{\def\PY@tc##1{\textcolor[rgb]{1.00,0.00,0.00}{##1}}}
\expandafter\def\csname PY@tok@ge\endcsname{\let\PY@it=\textit}
\expandafter\def\csname PY@tok@gs\endcsname{\let\PY@bf=\textbf}
\expandafter\def\csname PY@tok@gp\endcsname{\let\PY@bf=\textbf\def\PY@tc##1{\textcolor[rgb]{0.00,0.00,0.50}{##1}}}
\expandafter\def\csname PY@tok@go\endcsname{\def\PY@tc##1{\textcolor[rgb]{0.53,0.53,0.53}{##1}}}
\expandafter\def\csname PY@tok@gt\endcsname{\def\PY@tc##1{\textcolor[rgb]{0.00,0.27,0.87}{##1}}}
\expandafter\def\csname PY@tok@err\endcsname{\def\PY@bc##1{\setlength{\fboxsep}{0pt}\fcolorbox[rgb]{1.00,0.00,0.00}{1,1,1}{\strut ##1}}}
\expandafter\def\csname PY@tok@kc\endcsname{\let\PY@bf=\textbf\def\PY@tc##1{\textcolor[rgb]{0.00,0.50,0.00}{##1}}}
\expandafter\def\csname PY@tok@kd\endcsname{\let\PY@bf=\textbf\def\PY@tc##1{\textcolor[rgb]{0.00,0.50,0.00}{##1}}}
\expandafter\def\csname PY@tok@kn\endcsname{\let\PY@bf=\textbf\def\PY@tc##1{\textcolor[rgb]{0.00,0.50,0.00}{##1}}}
\expandafter\def\csname PY@tok@kr\endcsname{\let\PY@bf=\textbf\def\PY@tc##1{\textcolor[rgb]{0.00,0.50,0.00}{##1}}}
\expandafter\def\csname PY@tok@bp\endcsname{\def\PY@tc##1{\textcolor[rgb]{0.00,0.50,0.00}{##1}}}
\expandafter\def\csname PY@tok@fm\endcsname{\def\PY@tc##1{\textcolor[rgb]{0.00,0.00,1.00}{##1}}}
\expandafter\def\csname PY@tok@vc\endcsname{\def\PY@tc##1{\textcolor[rgb]{0.10,0.09,0.49}{##1}}}
\expandafter\def\csname PY@tok@vg\endcsname{\def\PY@tc##1{\textcolor[rgb]{0.10,0.09,0.49}{##1}}}
\expandafter\def\csname PY@tok@vi\endcsname{\def\PY@tc##1{\textcolor[rgb]{0.10,0.09,0.49}{##1}}}
\expandafter\def\csname PY@tok@vm\endcsname{\def\PY@tc##1{\textcolor[rgb]{0.10,0.09,0.49}{##1}}}
\expandafter\def\csname PY@tok@sa\endcsname{\def\PY@tc##1{\textcolor[rgb]{0.73,0.13,0.13}{##1}}}
\expandafter\def\csname PY@tok@sb\endcsname{\def\PY@tc##1{\textcolor[rgb]{0.73,0.13,0.13}{##1}}}
\expandafter\def\csname PY@tok@sc\endcsname{\def\PY@tc##1{\textcolor[rgb]{0.73,0.13,0.13}{##1}}}
\expandafter\def\csname PY@tok@dl\endcsname{\def\PY@tc##1{\textcolor[rgb]{0.73,0.13,0.13}{##1}}}
\expandafter\def\csname PY@tok@s2\endcsname{\def\PY@tc##1{\textcolor[rgb]{0.73,0.13,0.13}{##1}}}
\expandafter\def\csname PY@tok@sh\endcsname{\def\PY@tc##1{\textcolor[rgb]{0.73,0.13,0.13}{##1}}}
\expandafter\def\csname PY@tok@s1\endcsname{\def\PY@tc##1{\textcolor[rgb]{0.73,0.13,0.13}{##1}}}
\expandafter\def\csname PY@tok@mb\endcsname{\def\PY@tc##1{\textcolor[rgb]{0.40,0.40,0.40}{##1}}}
\expandafter\def\csname PY@tok@mf\endcsname{\def\PY@tc##1{\textcolor[rgb]{0.40,0.40,0.40}{##1}}}
\expandafter\def\csname PY@tok@mh\endcsname{\def\PY@tc##1{\textcolor[rgb]{0.40,0.40,0.40}{##1}}}
\expandafter\def\csname PY@tok@mi\endcsname{\def\PY@tc##1{\textcolor[rgb]{0.40,0.40,0.40}{##1}}}
\expandafter\def\csname PY@tok@il\endcsname{\def\PY@tc##1{\textcolor[rgb]{0.40,0.40,0.40}{##1}}}
\expandafter\def\csname PY@tok@mo\endcsname{\def\PY@tc##1{\textcolor[rgb]{0.40,0.40,0.40}{##1}}}
\expandafter\def\csname PY@tok@ch\endcsname{\let\PY@it=\textit\def\PY@tc##1{\textcolor[rgb]{0.25,0.50,0.50}{##1}}}
\expandafter\def\csname PY@tok@cm\endcsname{\let\PY@it=\textit\def\PY@tc##1{\textcolor[rgb]{0.25,0.50,0.50}{##1}}}
\expandafter\def\csname PY@tok@cpf\endcsname{\let\PY@it=\textit\def\PY@tc##1{\textcolor[rgb]{0.25,0.50,0.50}{##1}}}
\expandafter\def\csname PY@tok@c1\endcsname{\let\PY@it=\textit\def\PY@tc##1{\textcolor[rgb]{0.25,0.50,0.50}{##1}}}
\expandafter\def\csname PY@tok@cs\endcsname{\let\PY@it=\textit\def\PY@tc##1{\textcolor[rgb]{0.25,0.50,0.50}{##1}}}

\def\PYZbs{\char`\\}
\def\PYZus{\char`\_}
\def\PYZob{\char`\{}
\def\PYZcb{\char`\}}
\def\PYZca{\char`\^}
\def\PYZam{\char`\&}
\def\PYZlt{\char`\<}
\def\PYZgt{\char`\>}
\def\PYZsh{\char`\#}
\def\PYZpc{\char`\%}
\def\PYZdl{\char`\$}
\def\PYZhy{\char`\-}
\def\PYZsq{\char`\'}
\def\PYZdq{\char`\"}
\def\PYZti{\char`\~}
% for compatibility with earlier versions
\def\PYZat{@}
\def\PYZlb{[}
\def\PYZrb{]}
\makeatother


    % For linebreaks inside Verbatim environment from package fancyvrb. 
    \makeatletter
        \newbox\Wrappedcontinuationbox 
        \newbox\Wrappedvisiblespacebox 
        \newcommand*\Wrappedvisiblespace {\textcolor{red}{\textvisiblespace}} 
        \newcommand*\Wrappedcontinuationsymbol {\textcolor{red}{\llap{\tiny$\m@th\hookrightarrow$}}} 
        \newcommand*\Wrappedcontinuationindent {3ex } 
        \newcommand*\Wrappedafterbreak {\kern\Wrappedcontinuationindent\copy\Wrappedcontinuationbox} 
        % Take advantage of the already applied Pygments mark-up to insert 
        % potential linebreaks for TeX processing. 
        %        {, <, #, %, $, ' and ": go to next line. 
        %        _, }, ^, &, >, - and ~: stay at end of broken line. 
        % Use of \textquotesingle for straight quote. 
        \newcommand*\Wrappedbreaksatspecials {% 
            \def\PYGZus{\discretionary{\char`\_}{\Wrappedafterbreak}{\char`\_}}% 
            \def\PYGZob{\discretionary{}{\Wrappedafterbreak\char`\{}{\char`\{}}% 
            \def\PYGZcb{\discretionary{\char`\}}{\Wrappedafterbreak}{\char`\}}}% 
            \def\PYGZca{\discretionary{\char`\^}{\Wrappedafterbreak}{\char`\^}}% 
            \def\PYGZam{\discretionary{\char`\&}{\Wrappedafterbreak}{\char`\&}}% 
            \def\PYGZlt{\discretionary{}{\Wrappedafterbreak\char`\<}{\char`\<}}% 
            \def\PYGZgt{\discretionary{\char`\>}{\Wrappedafterbreak}{\char`\>}}% 
            \def\PYGZsh{\discretionary{}{\Wrappedafterbreak\char`\#}{\char`\#}}% 
            \def\PYGZpc{\discretionary{}{\Wrappedafterbreak\char`\%}{\char`\%}}% 
            \def\PYGZdl{\discretionary{}{\Wrappedafterbreak\char`\$}{\char`\$}}% 
            \def\PYGZhy{\discretionary{\char`\-}{\Wrappedafterbreak}{\char`\-}}% 
            \def\PYGZsq{\discretionary{}{\Wrappedafterbreak\textquotesingle}{\textquotesingle}}% 
            \def\PYGZdq{\discretionary{}{\Wrappedafterbreak\char`\"}{\char`\"}}% 
            \def\PYGZti{\discretionary{\char`\~}{\Wrappedafterbreak}{\char`\~}}% 
        } 
        % Some characters . , ; ? ! / are not pygmentized. 
        % This macro makes them "active" and they will insert potential linebreaks 
        \newcommand*\Wrappedbreaksatpunct {% 
            \lccode`\~`\.\lowercase{\def~}{\discretionary{\hbox{\char`\.}}{\Wrappedafterbreak}{\hbox{\char`\.}}}% 
            \lccode`\~`\,\lowercase{\def~}{\discretionary{\hbox{\char`\,}}{\Wrappedafterbreak}{\hbox{\char`\,}}}% 
            \lccode`\~`\;\lowercase{\def~}{\discretionary{\hbox{\char`\;}}{\Wrappedafterbreak}{\hbox{\char`\;}}}% 
            \lccode`\~`\:\lowercase{\def~}{\discretionary{\hbox{\char`\:}}{\Wrappedafterbreak}{\hbox{\char`\:}}}% 
            \lccode`\~`\?\lowercase{\def~}{\discretionary{\hbox{\char`\?}}{\Wrappedafterbreak}{\hbox{\char`\?}}}% 
            \lccode`\~`\!\lowercase{\def~}{\discretionary{\hbox{\char`\!}}{\Wrappedafterbreak}{\hbox{\char`\!}}}% 
            \lccode`\~`\/\lowercase{\def~}{\discretionary{\hbox{\char`\/}}{\Wrappedafterbreak}{\hbox{\char`\/}}}% 
            \catcode`\.\active
            \catcode`\,\active 
            \catcode`\;\active
            \catcode`\:\active
            \catcode`\?\active
            \catcode`\!\active
            \catcode`\/\active 
            \lccode`\~`\~ 	
        }
    \makeatother

    \let\OriginalVerbatim=\Verbatim
    \makeatletter
    \renewcommand{\Verbatim}[1][1]{%
        %\parskip\z@skip
        \sbox\Wrappedcontinuationbox {\Wrappedcontinuationsymbol}%
        \sbox\Wrappedvisiblespacebox {\FV@SetupFont\Wrappedvisiblespace}%
        \def\FancyVerbFormatLine ##1{\hsize\linewidth
            \vtop{\raggedright\hyphenpenalty\z@\exhyphenpenalty\z@
                \doublehyphendemerits\z@\finalhyphendemerits\z@
                \strut ##1\strut}%
        }%
        % If the linebreak is at a space, the latter will be displayed as visible
        % space at end of first line, and a continuation symbol starts next line.
        % Stretch/shrink are however usually zero for typewriter font.
        \def\FV@Space {%
            \nobreak\hskip\z@ plus\fontdimen3\font minus\fontdimen4\font
            \discretionary{\copy\Wrappedvisiblespacebox}{\Wrappedafterbreak}
            {\kern\fontdimen2\font}%
        }%
        
        % Allow breaks at special characters using \PYG... macros.
        \Wrappedbreaksatspecials
        % Breaks at punctuation characters . , ; ? ! and / need catcode=\active 	
        \OriginalVerbatim[#1,codes*=\Wrappedbreaksatpunct]%
    }
    \makeatother

    % Exact colors from NB
    \definecolor{incolor}{HTML}{303F9F}
    \definecolor{outcolor}{HTML}{D84315}
    \definecolor{cellborder}{HTML}{CFCFCF}
    \definecolor{cellbackground}{HTML}{F7F7F7}
    
    % prompt
    \makeatletter
    \newcommand{\boxspacing}{\kern\kvtcb@left@rule\kern\kvtcb@boxsep}
    \makeatother
    \newcommand{\prompt}[4]{
        \ttfamily\llap{{\color{#2}[#3]:\hspace{3pt}#4}}\vspace{-\baselineskip}
    }
    

    
    % Prevent overflowing lines due to hard-to-break entities
    \sloppy 
    % Setup hyperref package
    \hypersetup{
      breaklinks=true,  % so long urls are correctly broken across lines
      colorlinks=true,
      urlcolor=urlcolor,
      linkcolor=linkcolor,
      citecolor=citecolor,
      }
    % Slightly bigger margins than the latex defaults
    
    \geometry{verbose,tmargin=1in,bmargin=1in,lmargin=1in,rmargin=1in}
    
    

\begin{document}
    
    \maketitle
    
    

    
    \hypertarget{assignment-1-logistic-regression}{%
\section{Assignment 1: Logistic
Regression}\label{assignment-1-logistic-regression}}

Welcome to week one of this specialization. You will learn about
logistic regression. Concretely, you will be implementing logistic
regression for sentiment analysis on tweets. Given a tweet, you will
decide if it has a positive sentiment or a negative one. Specifically
you will:

\begin{itemize}
\tightlist
\item
  Learn how to extract features for logistic regression given some text
\item
  Implement logistic regression from scratch
\item
  Apply logistic regression on a natural language processing task
\item
  Test using your logistic regression
\item
  Perform error analysis
\end{itemize}

\hypertarget{important-note-on-submission-to-the-autograder}{%
\subsection{Important Note on Submission to the
AutoGrader}\label{important-note-on-submission-to-the-autograder}}

Before submitting your assignment to the AutoGrader, please make sure
you are not doing the following:

\begin{enumerate}
\def\labelenumi{\arabic{enumi}.}
\tightlist
\item
  You have not added any \emph{extra} \texttt{print} statement(s) in the
  assignment.
\item
  You have not added any \emph{extra} code cell(s) in the assignment.
\item
  You have not changed any of the function parameters.
\item
  You are not using any global variables inside your graded exercises.
  Unless specifically instructed to do so, please refrain from it and
  use the local variables instead.
\item
  You are not changing the assignment code where it is not required,
  like creating \emph{extra} variables.
\end{enumerate}

If you do any of the following, you will get something like,
\texttt{Grader\ Error:\ Grader\ feedback\ not\ found} (or similarly
unexpected) error upon submitting your assignment. Before asking for
help/debugging the errors in your assignment, check for these first. If
this is the case, and you don't remember the changes you have made, you
can get a fresh copy of the assignment by following these
\href{https://www.coursera.org/learn/classification-vector-spaces-in-nlp/supplement/YLuAg/h-ow-to-refresh-your-workspace}{instructions}.

Lets get started!

We will be using a data set of tweets. Hopefully you will get more than
99\% accuracy.\\
Run the cell below to load in the packages.

    \hypertarget{table-of-contents}{%
\subsection{Table of Contents}\label{table-of-contents}}

\begin{itemize}
\tightlist
\item
  Section \ref{0}
\item
  Section \ref{1}

  \begin{itemize}
  \tightlist
  \item
    Section \ref{1-1}

    \begin{itemize}
    \tightlist
    \item
      Section \ref{ex-1}
    \end{itemize}
  \item
    Section \ref{1-2}

    \begin{itemize}
    \tightlist
    \item
      Section \ref{ex-2}
    \end{itemize}
  \end{itemize}
\item
  Section \ref{2}

  \begin{itemize}
  \tightlist
  \item
    Section \ref{ex-3}
  \end{itemize}
\item
  Section \ref{3}
\item
  Section \ref{4}

  \begin{itemize}
  \tightlist
  \item
    Section \ref{ex-4}
  \item
    Section \ref{4-1}

    \begin{itemize}
    \tightlist
    \item
      Section \ref{ex-5}
    \end{itemize}
  \end{itemize}
\item
  Section \ref{5}
\item
  Section \ref{6}
\end{itemize}

    \#\# Import Functions and Data

    \begin{tcolorbox}[breakable, size=fbox, boxrule=1pt, pad at break*=1mm,colback=cellbackground, colframe=cellborder]
\prompt{In}{incolor}{1}{\boxspacing}
\begin{Verbatim}[commandchars=\\\{\}]
\PY{c+c1}{\PYZsh{} run this cell to import nltk}
\PY{k+kn}{import} \PY{n+nn}{nltk}
\PY{k+kn}{from} \PY{n+nn}{os} \PY{k+kn}{import} \PY{n}{getcwd}
\PY{k+kn}{import} \PY{n+nn}{w1\PYZus{}unittest}
\PY{k+kn}{import} \PY{n+nn}{ssl}

\PY{k}{try}\PY{p}{:}
    \PY{n}{\PYZus{}create\PYZus{}unverified\PYZus{}https\PYZus{}context} \PY{o}{=} \PY{n}{ssl}\PY{o}{.}\PY{n}{\PYZus{}create\PYZus{}unverified\PYZus{}context}
\PY{k}{except} \PY{n+ne}{AttributeError}\PY{p}{:}
    \PY{k}{pass}
\PY{k}{else}\PY{p}{:}
    \PY{n}{ssl}\PY{o}{.}\PY{n}{\PYZus{}create\PYZus{}default\PYZus{}https\PYZus{}context} \PY{o}{=} \PY{n}{\PYZus{}create\PYZus{}unverified\PYZus{}https\PYZus{}context}


\PY{n}{nltk}\PY{o}{.}\PY{n}{download}\PY{p}{(}\PY{l+s+s1}{\PYZsq{}}\PY{l+s+s1}{twitter\PYZus{}samples}\PY{l+s+s1}{\PYZsq{}}\PY{p}{)}
\PY{n}{nltk}\PY{o}{.}\PY{n}{download}\PY{p}{(}\PY{l+s+s1}{\PYZsq{}}\PY{l+s+s1}{stopwords}\PY{l+s+s1}{\PYZsq{}}\PY{p}{)}
\end{Verbatim}
\end{tcolorbox}

    \begin{Verbatim}[commandchars=\\\{\}]
[nltk\_data] Downloading package twitter\_samples to
[nltk\_data]     /home/jovyan/nltk\_data{\ldots}
[nltk\_data]   Package twitter\_samples is already up-to-date!
[nltk\_data] Downloading package stopwords to /home/jovyan/nltk\_data{\ldots}
[nltk\_data]   Package stopwords is already up-to-date!
    \end{Verbatim}

            \begin{tcolorbox}[breakable, size=fbox, boxrule=.5pt, pad at break*=1mm, opacityfill=0]
\prompt{Out}{outcolor}{1}{\boxspacing}
\begin{Verbatim}[commandchars=\\\{\}]
True
\end{Verbatim}
\end{tcolorbox}
        
    \hypertarget{imported-functions}{%
\subsubsection{Imported Functions}\label{imported-functions}}

Download the data needed for this assignment. Check out the
\href{http://www.nltk.org/howto/twitter.html}{documentation for the
twitter\_samples dataset}.

\begin{itemize}
\tightlist
\item
  twitter\_samples: if you're running this notebook on your local
  computer, you will need to download it using:
\end{itemize}

\begin{Shaded}
\begin{Highlighting}[]
\NormalTok{nltk.download(}\StringTok{\textquotesingle{}twitter\_samples\textquotesingle{}}\NormalTok{)}
\end{Highlighting}
\end{Shaded}

\begin{itemize}
\tightlist
\item
  stopwords: if you're running this notebook on your local computer, you
  will need to download it using:
\end{itemize}

\begin{Shaded}
\begin{Highlighting}[]
\NormalTok{nltk.download(}\StringTok{\textquotesingle{}stopwords\textquotesingle{}}\NormalTok{)}
\end{Highlighting}
\end{Shaded}

\hypertarget{import-some-helper-functions-that-we-provided-in-the-utils.py-file}{%
\paragraph{Import some helper functions that we provided in the utils.py
file:}\label{import-some-helper-functions-that-we-provided-in-the-utils.py-file}}

\begin{itemize}
\tightlist
\item
  process\_tweet: cleans the text, tokenizes it into separate words,
  removes stopwords, and converts words to stems.
\item
  build\_freqs: this counts how often a word in the `corpus' (the entire
  set of tweets) was associated with a positive label `1' or a negative
  label `0', then builds the `freqs' dictionary, where each key is the
  (word,label) tuple, and the value is the count of its frequency within
  the corpus of tweets.
\end{itemize}

    \begin{tcolorbox}[breakable, size=fbox, boxrule=1pt, pad at break*=1mm,colback=cellbackground, colframe=cellborder]
\prompt{In}{incolor}{2}{\boxspacing}
\begin{Verbatim}[commandchars=\\\{\}]
\PY{n}{filePath} \PY{o}{=} \PY{l+s+sa}{f}\PY{l+s+s2}{\PYZdq{}}\PY{l+s+si}{\PYZob{}}\PY{n}{getcwd}\PY{p}{(}\PY{p}{)}\PY{l+s+si}{\PYZcb{}}\PY{l+s+s2}{/../tmp2/}\PY{l+s+s2}{\PYZdq{}}
\PY{n}{nltk}\PY{o}{.}\PY{n}{data}\PY{o}{.}\PY{n}{path}\PY{o}{.}\PY{n}{append}\PY{p}{(}\PY{n}{filePath}\PY{p}{)}
\end{Verbatim}
\end{tcolorbox}

    \begin{tcolorbox}[breakable, size=fbox, boxrule=1pt, pad at break*=1mm,colback=cellbackground, colframe=cellborder]
\prompt{In}{incolor}{3}{\boxspacing}
\begin{Verbatim}[commandchars=\\\{\}]
\PY{k+kn}{import} \PY{n+nn}{numpy} \PY{k}{as} \PY{n+nn}{np}
\PY{k+kn}{import} \PY{n+nn}{pandas} \PY{k}{as} \PY{n+nn}{pd}
\PY{k+kn}{from} \PY{n+nn}{nltk}\PY{n+nn}{.}\PY{n+nn}{corpus} \PY{k+kn}{import} \PY{n}{twitter\PYZus{}samples} 

\PY{k+kn}{from} \PY{n+nn}{utils} \PY{k+kn}{import} \PY{n}{process\PYZus{}tweet}\PY{p}{,} \PY{n}{build\PYZus{}freqs}
\end{Verbatim}
\end{tcolorbox}

    \hypertarget{prepare-the-data}{%
\subsubsection{Prepare the Data}\label{prepare-the-data}}

\begin{itemize}
\tightlist
\item
  The \texttt{twitter\_samples} contains subsets of five thousand
  positive\_tweets, five thousand negative\_tweets, and the full set of
  10,000 tweets.

  \begin{itemize}
  \tightlist
  \item
    If you used all three datasets, we would introduce duplicates of the
    positive tweets and negative tweets.\\
  \item
    You will select just the five thousand positive tweets and five
    thousand negative tweets.
  \end{itemize}
\end{itemize}

    \begin{tcolorbox}[breakable, size=fbox, boxrule=1pt, pad at break*=1mm,colback=cellbackground, colframe=cellborder]
\prompt{In}{incolor}{4}{\boxspacing}
\begin{Verbatim}[commandchars=\\\{\}]
\PY{c+c1}{\PYZsh{} select the set of positive and negative tweets}
\PY{n}{all\PYZus{}positive\PYZus{}tweets} \PY{o}{=} \PY{n}{twitter\PYZus{}samples}\PY{o}{.}\PY{n}{strings}\PY{p}{(}\PY{l+s+s1}{\PYZsq{}}\PY{l+s+s1}{positive\PYZus{}tweets.json}\PY{l+s+s1}{\PYZsq{}}\PY{p}{)}
\PY{n}{all\PYZus{}negative\PYZus{}tweets} \PY{o}{=} \PY{n}{twitter\PYZus{}samples}\PY{o}{.}\PY{n}{strings}\PY{p}{(}\PY{l+s+s1}{\PYZsq{}}\PY{l+s+s1}{negative\PYZus{}tweets.json}\PY{l+s+s1}{\PYZsq{}}\PY{p}{)}
\end{Verbatim}
\end{tcolorbox}

    \begin{itemize}
\tightlist
\item
  Train test split: 20\% will be in the test set, and 80\% in the
  training set.
\end{itemize}

    \begin{tcolorbox}[breakable, size=fbox, boxrule=1pt, pad at break*=1mm,colback=cellbackground, colframe=cellborder]
\prompt{In}{incolor}{5}{\boxspacing}
\begin{Verbatim}[commandchars=\\\{\}]
\PY{c+c1}{\PYZsh{} split the data into two pieces, one for training and one for testing (validation set) }
\PY{n}{test\PYZus{}pos} \PY{o}{=} \PY{n}{all\PYZus{}positive\PYZus{}tweets}\PY{p}{[}\PY{l+m+mi}{4000}\PY{p}{:}\PY{p}{]}
\PY{n}{train\PYZus{}pos} \PY{o}{=} \PY{n}{all\PYZus{}positive\PYZus{}tweets}\PY{p}{[}\PY{p}{:}\PY{l+m+mi}{4000}\PY{p}{]}
\PY{n}{test\PYZus{}neg} \PY{o}{=} \PY{n}{all\PYZus{}negative\PYZus{}tweets}\PY{p}{[}\PY{l+m+mi}{4000}\PY{p}{:}\PY{p}{]}
\PY{n}{train\PYZus{}neg} \PY{o}{=} \PY{n}{all\PYZus{}negative\PYZus{}tweets}\PY{p}{[}\PY{p}{:}\PY{l+m+mi}{4000}\PY{p}{]}

\PY{n}{train\PYZus{}x} \PY{o}{=} \PY{n}{train\PYZus{}pos} \PY{o}{+} \PY{n}{train\PYZus{}neg} 
\PY{n}{test\PYZus{}x} \PY{o}{=} \PY{n}{test\PYZus{}pos} \PY{o}{+} \PY{n}{test\PYZus{}neg}
\end{Verbatim}
\end{tcolorbox}

    \begin{itemize}
\tightlist
\item
  Create the numpy array of positive labels and negative labels.
\end{itemize}

    \begin{tcolorbox}[breakable, size=fbox, boxrule=1pt, pad at break*=1mm,colback=cellbackground, colframe=cellborder]
\prompt{In}{incolor}{6}{\boxspacing}
\begin{Verbatim}[commandchars=\\\{\}]
\PY{c+c1}{\PYZsh{} combine positive and negative labels}
\PY{n}{train\PYZus{}y} \PY{o}{=} \PY{n}{np}\PY{o}{.}\PY{n}{append}\PY{p}{(}\PY{n}{np}\PY{o}{.}\PY{n}{ones}\PY{p}{(}\PY{p}{(}\PY{n+nb}{len}\PY{p}{(}\PY{n}{train\PYZus{}pos}\PY{p}{)}\PY{p}{,} \PY{l+m+mi}{1}\PY{p}{)}\PY{p}{)}\PY{p}{,} \PY{n}{np}\PY{o}{.}\PY{n}{zeros}\PY{p}{(}\PY{p}{(}\PY{n+nb}{len}\PY{p}{(}\PY{n}{train\PYZus{}neg}\PY{p}{)}\PY{p}{,} \PY{l+m+mi}{1}\PY{p}{)}\PY{p}{)}\PY{p}{,} \PY{n}{axis}\PY{o}{=}\PY{l+m+mi}{0}\PY{p}{)}
\PY{n}{test\PYZus{}y} \PY{o}{=} \PY{n}{np}\PY{o}{.}\PY{n}{append}\PY{p}{(}\PY{n}{np}\PY{o}{.}\PY{n}{ones}\PY{p}{(}\PY{p}{(}\PY{n+nb}{len}\PY{p}{(}\PY{n}{test\PYZus{}pos}\PY{p}{)}\PY{p}{,} \PY{l+m+mi}{1}\PY{p}{)}\PY{p}{)}\PY{p}{,} \PY{n}{np}\PY{o}{.}\PY{n}{zeros}\PY{p}{(}\PY{p}{(}\PY{n+nb}{len}\PY{p}{(}\PY{n}{test\PYZus{}neg}\PY{p}{)}\PY{p}{,} \PY{l+m+mi}{1}\PY{p}{)}\PY{p}{)}\PY{p}{,} \PY{n}{axis}\PY{o}{=}\PY{l+m+mi}{0}\PY{p}{)}
\end{Verbatim}
\end{tcolorbox}

    \begin{tcolorbox}[breakable, size=fbox, boxrule=1pt, pad at break*=1mm,colback=cellbackground, colframe=cellborder]
\prompt{In}{incolor}{7}{\boxspacing}
\begin{Verbatim}[commandchars=\\\{\}]
\PY{c+c1}{\PYZsh{} Print the shape train and test sets}
\PY{n+nb}{print}\PY{p}{(}\PY{l+s+s2}{\PYZdq{}}\PY{l+s+s2}{train\PYZus{}y.shape = }\PY{l+s+s2}{\PYZdq{}} \PY{o}{+} \PY{n+nb}{str}\PY{p}{(}\PY{n}{train\PYZus{}y}\PY{o}{.}\PY{n}{shape}\PY{p}{)}\PY{p}{)}
\PY{n+nb}{print}\PY{p}{(}\PY{l+s+s2}{\PYZdq{}}\PY{l+s+s2}{test\PYZus{}y.shape = }\PY{l+s+s2}{\PYZdq{}} \PY{o}{+} \PY{n+nb}{str}\PY{p}{(}\PY{n}{test\PYZus{}y}\PY{o}{.}\PY{n}{shape}\PY{p}{)}\PY{p}{)}
\end{Verbatim}
\end{tcolorbox}

    \begin{Verbatim}[commandchars=\\\{\}]
train\_y.shape = (8000, 1)
test\_y.shape = (2000, 1)
    \end{Verbatim}

    \begin{itemize}
\tightlist
\item
  Create the frequency dictionary using the imported build\_freqs
  function.

  \begin{itemize}
  \tightlist
  \item
    We highly recommend that you open utils.py and read the build\_freqs
    function to understand what it is doing.
  \item
    To view the file directory, go to the menu and click
    File-\textgreater Open.
  \end{itemize}
\end{itemize}

\begin{Shaded}
\begin{Highlighting}[]
    \ControlFlowTok{for}\NormalTok{ y,tweet }\KeywordTok{in} \BuiltInTok{zip}\NormalTok{(ys, tweets):}
        \ControlFlowTok{for}\NormalTok{ word }\KeywordTok{in}\NormalTok{ process\_tweet(tweet):}
\NormalTok{            pair }\OperatorTok{=}\NormalTok{ (word, y)}
            \ControlFlowTok{if}\NormalTok{ pair }\KeywordTok{in}\NormalTok{ freqs:}
\NormalTok{                freqs[pair] }\OperatorTok{+=} \DecValTok{1}
            \ControlFlowTok{else}\NormalTok{:}
\NormalTok{                freqs[pair] }\OperatorTok{=} \DecValTok{1}
\end{Highlighting}
\end{Shaded}

\begin{itemize}
\tightlist
\item
  Notice how the outer for loop goes through each tweet, and the inner
  for loop steps through each word in a tweet.
\item
  The `freqs' dictionary is the frequency dictionary that's being built.
\item
  The key is the tuple (word, label), such as (``happy'',1) or
  (``happy'',0). The value stored for each key is the count of how many
  times the word ``happy'' was associated with a positive label, or how
  many times ``happy'' was associated with a negative label.
\end{itemize}

    \begin{tcolorbox}[breakable, size=fbox, boxrule=1pt, pad at break*=1mm,colback=cellbackground, colframe=cellborder]
\prompt{In}{incolor}{8}{\boxspacing}
\begin{Verbatim}[commandchars=\\\{\}]
\PY{c+c1}{\PYZsh{} create frequency dictionary}
\PY{n}{freqs} \PY{o}{=} \PY{n}{build\PYZus{}freqs}\PY{p}{(}\PY{n}{train\PYZus{}x}\PY{p}{,} \PY{n}{train\PYZus{}y}\PY{p}{)}

\PY{c+c1}{\PYZsh{} check the output}
\PY{n+nb}{print}\PY{p}{(}\PY{l+s+s2}{\PYZdq{}}\PY{l+s+s2}{type(freqs) = }\PY{l+s+s2}{\PYZdq{}} \PY{o}{+} \PY{n+nb}{str}\PY{p}{(}\PY{n+nb}{type}\PY{p}{(}\PY{n}{freqs}\PY{p}{)}\PY{p}{)}\PY{p}{)}
\PY{n+nb}{print}\PY{p}{(}\PY{l+s+s2}{\PYZdq{}}\PY{l+s+s2}{len(freqs) = }\PY{l+s+s2}{\PYZdq{}} \PY{o}{+} \PY{n+nb}{str}\PY{p}{(}\PY{n+nb}{len}\PY{p}{(}\PY{n}{freqs}\PY{o}{.}\PY{n}{keys}\PY{p}{(}\PY{p}{)}\PY{p}{)}\PY{p}{)}\PY{p}{)}
\end{Verbatim}
\end{tcolorbox}

    \begin{Verbatim}[commandchars=\\\{\}]
type(freqs) = <class 'dict'>
len(freqs) = 11436
    \end{Verbatim}

    \hypertarget{expected-output}{%
\paragraph{Expected output}\label{expected-output}}

\begin{verbatim}
type(freqs) = <class 'dict'>
len(freqs) = 11436
\end{verbatim}

    \hypertarget{process-tweet}{%
\subsubsection{Process Tweet}\label{process-tweet}}

The given function `process\_tweet' tokenizes the tweet into individual
words, removes stop words and applies stemming.

    \begin{tcolorbox}[breakable, size=fbox, boxrule=1pt, pad at break*=1mm,colback=cellbackground, colframe=cellborder]
\prompt{In}{incolor}{9}{\boxspacing}
\begin{Verbatim}[commandchars=\\\{\}]
\PY{c+c1}{\PYZsh{} test the function below}
\PY{n+nb}{print}\PY{p}{(}\PY{l+s+s1}{\PYZsq{}}\PY{l+s+s1}{This is an example of a positive tweet: }\PY{l+s+se}{\PYZbs{}n}\PY{l+s+s1}{\PYZsq{}}\PY{p}{,} \PY{n}{train\PYZus{}x}\PY{p}{[}\PY{l+m+mi}{0}\PY{p}{]}\PY{p}{)}
\PY{n+nb}{print}\PY{p}{(}\PY{l+s+s1}{\PYZsq{}}\PY{l+s+se}{\PYZbs{}n}\PY{l+s+s1}{This is an example of the processed version of the tweet: }\PY{l+s+se}{\PYZbs{}n}\PY{l+s+s1}{\PYZsq{}}\PY{p}{,} \PY{n}{process\PYZus{}tweet}\PY{p}{(}\PY{n}{train\PYZus{}x}\PY{p}{[}\PY{l+m+mi}{0}\PY{p}{]}\PY{p}{)}\PY{p}{)}
\end{Verbatim}
\end{tcolorbox}

    \begin{Verbatim}[commandchars=\\\{\}]
This is an example of a positive tweet:
 \#FollowFriday @France\_Inte @PKuchly57 @Milipol\_Paris for being top engaged
members in my community this week :)

This is an example of the processed version of the tweet:
 ['followfriday', 'top', 'engag', 'member', 'commun', 'week', ':)']
    \end{Verbatim}

    \hypertarget{expected-output}{%
\paragraph{Expected output}\label{expected-output}}

\begin{verbatim}
This is an example of a positive tweet: 
 #FollowFriday @France_Inte @PKuchly57 @Milipol_Paris for being top engaged members in my community this week :)
 
This is an example of the processes version: 
 ['followfriday', 'top', 'engag', 'member', 'commun', 'week', ':)']
\end{verbatim}

    \#\# 1 - Logistic Regression

\#\#\# 1.1 - Sigmoid You will learn to use logistic regression for text
classification. * The sigmoid function is defined as:

\[ h(z) = \frac{1}{1+\exp^{-z}} \tag{1}\]

It maps the input `z' to a value that ranges between 0 and 1, and so it
can be treated as a probability.

Figure 1

    \#\#\# Exercise 1 - sigmoid Implement the sigmoid function. * You will
want this function to work if z is a scalar as well as if it is an
array.

    Hints

numpy.exp

    \begin{tcolorbox}[breakable, size=fbox, boxrule=1pt, pad at break*=1mm,colback=cellbackground, colframe=cellborder]
\prompt{In}{incolor}{14}{\boxspacing}
\begin{Verbatim}[commandchars=\\\{\}]
\PY{c+c1}{\PYZsh{} UNQ\PYZus{}C1 GRADED FUNCTION: sigmoid}
\PY{k}{def} \PY{n+nf}{sigmoid}\PY{p}{(}\PY{n}{z}\PY{p}{)}\PY{p}{:} 
    \PY{l+s+sd}{\PYZsq{}\PYZsq{}\PYZsq{}}
\PY{l+s+sd}{    Input:}
\PY{l+s+sd}{        z: is the input (can be a scalar or an array)}
\PY{l+s+sd}{    Output:}
\PY{l+s+sd}{        h: the sigmoid of z}
\PY{l+s+sd}{    \PYZsq{}\PYZsq{}\PYZsq{}}
    
    \PY{c+c1}{\PYZsh{}\PYZsh{}\PYZsh{} START CODE HERE \PYZsh{}\PYZsh{}\PYZsh{}}
    \PY{c+c1}{\PYZsh{} calculate the sigmoid of z}
    \PY{n}{h} \PY{o}{=} \PY{l+m+mi}{1}\PY{o}{/}\PY{p}{(}\PY{l+m+mi}{1}\PY{o}{+}\PY{n}{np}\PY{o}{.}\PY{n}{exp}\PY{p}{(}\PY{o}{\PYZhy{}}\PY{n}{z}\PY{p}{)}\PY{p}{)}
    \PY{c+c1}{\PYZsh{}\PYZsh{}\PYZsh{} END CODE HERE \PYZsh{}\PYZsh{}\PYZsh{}}
    
    \PY{k}{return} \PY{n}{h}
\end{Verbatim}
\end{tcolorbox}

    \begin{tcolorbox}[breakable, size=fbox, boxrule=1pt, pad at break*=1mm,colback=cellbackground, colframe=cellborder]
\prompt{In}{incolor}{15}{\boxspacing}
\begin{Verbatim}[commandchars=\\\{\}]
\PY{c+c1}{\PYZsh{} Testing your function }
\PY{k}{if} \PY{p}{(}\PY{n}{sigmoid}\PY{p}{(}\PY{l+m+mi}{0}\PY{p}{)} \PY{o}{==} \PY{l+m+mf}{0.5}\PY{p}{)}\PY{p}{:}
    \PY{n+nb}{print}\PY{p}{(}\PY{l+s+s1}{\PYZsq{}}\PY{l+s+s1}{SUCCESS!}\PY{l+s+s1}{\PYZsq{}}\PY{p}{)}
\PY{k}{else}\PY{p}{:}
    \PY{n+nb}{print}\PY{p}{(}\PY{l+s+s1}{\PYZsq{}}\PY{l+s+s1}{Oops!}\PY{l+s+s1}{\PYZsq{}}\PY{p}{)}

\PY{k}{if} \PY{p}{(}\PY{n}{sigmoid}\PY{p}{(}\PY{l+m+mf}{4.92}\PY{p}{)} \PY{o}{==} \PY{l+m+mf}{0.9927537604041685}\PY{p}{)}\PY{p}{:}
    \PY{n+nb}{print}\PY{p}{(}\PY{l+s+s1}{\PYZsq{}}\PY{l+s+s1}{CORRECT!}\PY{l+s+s1}{\PYZsq{}}\PY{p}{)}
\PY{k}{else}\PY{p}{:}
    \PY{n+nb}{print}\PY{p}{(}\PY{l+s+s1}{\PYZsq{}}\PY{l+s+s1}{Oops again!}\PY{l+s+s1}{\PYZsq{}}\PY{p}{)}
\end{Verbatim}
\end{tcolorbox}

    \begin{Verbatim}[commandchars=\\\{\}]
SUCCESS!
CORRECT!
    \end{Verbatim}

    \begin{tcolorbox}[breakable, size=fbox, boxrule=1pt, pad at break*=1mm,colback=cellbackground, colframe=cellborder]
\prompt{In}{incolor}{16}{\boxspacing}
\begin{Verbatim}[commandchars=\\\{\}]
\PY{c+c1}{\PYZsh{} Test your function}
\PY{n}{w1\PYZus{}unittest}\PY{o}{.}\PY{n}{test\PYZus{}sigmoid}\PY{p}{(}\PY{n}{sigmoid}\PY{p}{)}
\end{Verbatim}
\end{tcolorbox}

    \begin{Verbatim}[commandchars=\\\{\}]
\textcolor{ansi-green-intense}{ All tests passed}
    \end{Verbatim}

    \hypertarget{logistic-regression-regression-and-a-sigmoid}{%
\paragraph{Logistic Regression: Regression and a
Sigmoid}\label{logistic-regression-regression-and-a-sigmoid}}

Logistic regression takes a regular linear regression, and applies a
sigmoid to the output of the linear regression.

Regression:
\[z = \theta_0 x_0 + \theta_1 x_1 + \theta_2 x_2 + ... \theta_N x_N\]
Note that the \(\theta\) values are ``weights''. If you took the deep
learning specialization, we referred to the weights with the `w' vector.
In this course, we're using a different variable \(\theta\) to refer to
the weights.

Logistic regression \[ h(z) = \frac{1}{1+\exp^{-z}}\]
\[z = \theta_0 x_0 + \theta_1 x_1 + \theta_2 x_2 + ... \theta_N x_N\] We
will refer to `z' as the `logits'.

    \#\#\# 1.2 - Cost function and Gradient

The cost function used for logistic regression is the average of the log
loss across all training examples:

\[J(\theta) = -\frac{1}{m} \sum_{i=1}^m y^{(i)}\log (h(z(\theta)^{(i)})) + (1-y^{(i)})\log (1-h(z(\theta)^{(i)}))\tag{5} \]
* \(m\) is the number of training examples * \(y^{(i)}\) is the actual
label of training example `i'. * \(h(z^{(i)})\) is the model's
prediction for the training example `i'.

The loss function for a single training example is
\[ Loss = -1 \times \left( y^{(i)}\log (h(z(\theta)^{(i)})) + (1-y^{(i)})\log (1-h(z(\theta)^{(i)})) \right)\]

\begin{itemize}
\tightlist
\item
  All the \(h\) values are between 0 and 1, so the logs will be
  negative. That is the reason for the factor of -1 applied to the sum
  of the two loss terms.
\item
  Note that when the model predicts 1 (\(h(z(\theta)) = 1\)) and the
  label `y' is also 1, the loss for that training example is 0.
\item
  Similarly, when the model predicts 0 (\(h(z(\theta)) = 0\)) and the
  actual label is also 0, the loss for that training example is 0.
\item
  However, when the model prediction is close to 1
  (\(h(z(\theta)) = 0.9999\)) and the label is 0, the second term of the
  log loss becomes a large negative number, which is then multiplied by
  the overall factor of -1 to convert it to a positive loss value.
  \(-1 \times (1 - 0) \times log(1 - 0.9999) \approx 9.2\) The closer
  the model prediction gets to 1, the larger the loss.
\end{itemize}

    \begin{tcolorbox}[breakable, size=fbox, boxrule=1pt, pad at break*=1mm,colback=cellbackground, colframe=cellborder]
\prompt{In}{incolor}{17}{\boxspacing}
\begin{Verbatim}[commandchars=\\\{\}]
\PY{c+c1}{\PYZsh{} verify that when the model predicts close to 1, but the actual label is 0, the loss is a large positive value}
\PY{o}{\PYZhy{}}\PY{l+m+mi}{1} \PY{o}{*} \PY{p}{(}\PY{l+m+mi}{1} \PY{o}{\PYZhy{}} \PY{l+m+mi}{0}\PY{p}{)} \PY{o}{*} \PY{n}{np}\PY{o}{.}\PY{n}{log}\PY{p}{(}\PY{l+m+mi}{1} \PY{o}{\PYZhy{}} \PY{l+m+mf}{0.9999}\PY{p}{)} \PY{c+c1}{\PYZsh{} loss is about 9.2}
\end{Verbatim}
\end{tcolorbox}

            \begin{tcolorbox}[breakable, size=fbox, boxrule=.5pt, pad at break*=1mm, opacityfill=0]
\prompt{Out}{outcolor}{17}{\boxspacing}
\begin{Verbatim}[commandchars=\\\{\}]
9.210340371976294
\end{Verbatim}
\end{tcolorbox}
        
    \begin{itemize}
\tightlist
\item
  Likewise, if the model predicts close to 0 (\(h(z) = 0.0001\)) but the
  actual label is 1, the first term in the loss function becomes a large
  number: \(-1 \times log(0.0001) \approx 9.2\). The closer the
  prediction is to zero, the larger the loss.
\end{itemize}

    \begin{tcolorbox}[breakable, size=fbox, boxrule=1pt, pad at break*=1mm,colback=cellbackground, colframe=cellborder]
\prompt{In}{incolor}{18}{\boxspacing}
\begin{Verbatim}[commandchars=\\\{\}]
\PY{c+c1}{\PYZsh{} verify that when the model predicts close to 0 but the actual label is 1, the loss is a large positive value}
\PY{o}{\PYZhy{}}\PY{l+m+mi}{1} \PY{o}{*} \PY{n}{np}\PY{o}{.}\PY{n}{log}\PY{p}{(}\PY{l+m+mf}{0.0001}\PY{p}{)} \PY{c+c1}{\PYZsh{} loss is about 9.2}
\end{Verbatim}
\end{tcolorbox}

            \begin{tcolorbox}[breakable, size=fbox, boxrule=.5pt, pad at break*=1mm, opacityfill=0]
\prompt{Out}{outcolor}{18}{\boxspacing}
\begin{Verbatim}[commandchars=\\\{\}]
9.210340371976182
\end{Verbatim}
\end{tcolorbox}
        
    \hypertarget{update-the-weights}{%
\paragraph{Update the weights}\label{update-the-weights}}

To update your weight vector \(\theta\), you will apply gradient descent
to iteratively improve your model's predictions.\\
The gradient of the cost function \(J\) with respect to one of the
weights \(\theta_j\) is:

\[\nabla_{\theta_j}J(\theta) = \frac{1}{m} \sum_{i=1}^m(h^{(i)}-y^{(i)})x^{(i)}_j \tag{5}\]
* `i' is the index across all `m' training examples. * `j' is the index
of the weight \(\theta_j\), so \(x^{(i)}_j\) is the feature associated
with weight \(\theta_j\)

\begin{itemize}
\tightlist
\item
  To update the weight \(\theta_j\), we adjust it by subtracting a
  fraction of the gradient determined by \(\alpha\):
  \[\theta_j = \theta_j - \alpha \times \nabla_{\theta_j}J(\theta) \]
\item
  The learning rate \(\alpha\) is a value that we choose to control how
  big a single update will be.
\end{itemize}

    \#\#\# Exercise 2 - gradientDescent Implement gradient descent function.
* The number of iterations `num\_iters" is the number of times that
you'll use the entire training set. * For each iteration, you'll
calculate the cost function using all training examples (there are 'm'
training examples), and for all features. * Instead of updating a single
weight \(\theta_i\) at a time, we can update all the weights in the
column vector:\\
\[\mathbf{\theta} = \begin{pmatrix}
\theta_0
\\
\theta_1
\\ 
\theta_2 
\\ 
\vdots
\\ 
\theta_n
\end{pmatrix}\] * \(\mathbf{\theta}\) has dimensions (n+1, 1), where `n'
is the number of features, and there is one more element for the bias
term \(\theta_0\) (note that the corresponding feature value
\(\mathbf{x_0}\) is 1). * The `logits', `z', are calculated by
multiplying the feature matrix `x' with the weight vector `theta'.
\(z = \mathbf{x}\mathbf{\theta}\) * \(\mathbf{x}\) has dimensions (m,
n+1) * \(\mathbf{\theta}\): has dimensions (n+1, 1) * \(\mathbf{z}\):
has dimensions (m, 1) * The prediction `h', is calculated by applying
the sigmoid to each element in `z': \(h(z) = sigmoid(z)\), and has
dimensions (m,1). * The cost function \(J\) is calculated by taking the
dot product of the vectors `y' and `log(h)'. Since both `y' and `h' are
column vectors (m,1), transpose the vector to the left, so that matrix
multiplication of a row vector with column vector performs the dot
product.
\[J = \frac{-1}{m} \times \left(\mathbf{y}^T \cdot log(\mathbf{h}) + \mathbf{(1-y)}^T \cdot log(\mathbf{1-h}) \right)\]
* The update of theta is also vectorized. Because the dimensions of
\(\mathbf{x}\) are (m, n+1), and both \(\mathbf{h}\) and \(\mathbf{y}\)
are (m, 1), we need to transpose the \(\mathbf{x}\) and place it on the
left in order to perform matrix multiplication, which then yields the
(n+1, 1) answer we need:
\[\mathbf{\theta} = \mathbf{\theta} - \frac{\alpha}{m} \times \left( \mathbf{x}^T \cdot \left( \mathbf{h-y} \right) \right)\]

    Hints

use numpy.dot for matrix multiplication.

To ensure that the fraction -1/m is a decimal value, cast either the
numerator or denominator (or both), like \texttt{float(1)}, or write
\texttt{1.} for the float version of 1.

    \begin{tcolorbox}[breakable, size=fbox, boxrule=1pt, pad at break*=1mm,colback=cellbackground, colframe=cellborder]
\prompt{In}{incolor}{21}{\boxspacing}
\begin{Verbatim}[commandchars=\\\{\}]
\PY{c+c1}{\PYZsh{} UNQ\PYZus{}C2 GRADED FUNCTION: gradientDescent}
\PY{k}{def} \PY{n+nf}{gradientDescent}\PY{p}{(}\PY{n}{x}\PY{p}{,} \PY{n}{y}\PY{p}{,} \PY{n}{theta}\PY{p}{,} \PY{n}{alpha}\PY{p}{,} \PY{n}{num\PYZus{}iters}\PY{p}{)}\PY{p}{:}
    \PY{l+s+sd}{\PYZsq{}\PYZsq{}\PYZsq{}}
\PY{l+s+sd}{    Input:}
\PY{l+s+sd}{        x: matrix of features which is (m,n+1)}
\PY{l+s+sd}{        y: corresponding labels of the input matrix x, dimensions (m,1)}
\PY{l+s+sd}{        theta: weight vector of dimension (n+1,1)}
\PY{l+s+sd}{        alpha: learning rate}
\PY{l+s+sd}{        num\PYZus{}iters: number of iterations you want to train your model for}
\PY{l+s+sd}{    Output:}
\PY{l+s+sd}{        J: the final cost}
\PY{l+s+sd}{        theta: your final weight vector}
\PY{l+s+sd}{    Hint: you might want to print the cost to make sure that it is going down.}
\PY{l+s+sd}{    \PYZsq{}\PYZsq{}\PYZsq{}}
    \PY{c+c1}{\PYZsh{}\PYZsh{}\PYZsh{} START CODE HERE \PYZsh{}\PYZsh{}\PYZsh{}}
    \PY{c+c1}{\PYZsh{} get \PYZsq{}m\PYZsq{}, the number of rows in matrix x}
    \PY{n}{m} \PY{o}{=} \PY{n}{np}\PY{o}{.}\PY{n}{shape}\PY{p}{(}\PY{n}{x}\PY{p}{)}\PY{p}{[}\PY{l+m+mi}{0}\PY{p}{]}
    
    \PY{k}{for} \PY{n}{i} \PY{o+ow}{in} \PY{n+nb}{range}\PY{p}{(}\PY{l+m+mi}{0}\PY{p}{,} \PY{n}{num\PYZus{}iters}\PY{p}{)}\PY{p}{:}
        
        \PY{c+c1}{\PYZsh{} get z, the dot product of x and theta}
        \PY{n}{z} \PY{o}{=} \PY{n}{np}\PY{o}{.}\PY{n}{dot}\PY{p}{(}\PY{n}{x}\PY{p}{,}\PY{n}{theta}\PY{p}{)}
        
        \PY{c+c1}{\PYZsh{} get the sigmoid of z}
        \PY{n}{h} \PY{o}{=} \PY{n}{sigmoid}\PY{p}{(}\PY{n}{z}\PY{p}{)}
        
        \PY{c+c1}{\PYZsh{} calculate the cost function}
        \PY{n}{J} \PY{o}{=} \PY{p}{(}\PY{n}{np}\PY{o}{.}\PY{n}{dot}\PY{p}{(}\PY{n}{np}\PY{o}{.}\PY{n}{transpose}\PY{p}{(}\PY{n}{y}\PY{p}{)}\PY{p}{,}\PY{n}{np}\PY{o}{.}\PY{n}{log}\PY{p}{(}\PY{n}{h}\PY{p}{)}\PY{p}{)} \PY{o}{+} \PY{n}{np}\PY{o}{.}\PY{n}{dot}\PY{p}{(}\PY{n}{np}\PY{o}{.}\PY{n}{transpose}\PY{p}{(}\PY{l+m+mi}{1}\PY{o}{\PYZhy{}}\PY{n}{y}\PY{p}{)}\PY{p}{,}\PY{n}{np}\PY{o}{.}\PY{n}{log}\PY{p}{(}\PY{l+m+mi}{1}\PY{o}{\PYZhy{}}\PY{n}{h}\PY{p}{)}\PY{p}{)}\PY{p}{)}\PY{o}{/}\PY{p}{(}\PY{o}{\PYZhy{}}\PY{n}{m}\PY{p}{)}

        \PY{c+c1}{\PYZsh{} update the weights theta}
        \PY{n}{theta} \PY{o}{=} \PY{n}{theta} \PY{o}{\PYZhy{}} \PY{p}{(}\PY{n}{alpha}\PY{o}{/}\PY{n}{m}\PY{p}{)} \PY{o}{*} \PY{p}{(}\PY{n}{np}\PY{o}{.}\PY{n}{dot}\PY{p}{(}\PY{n}{np}\PY{o}{.}\PY{n}{transpose}\PY{p}{(}\PY{n}{x}\PY{p}{)}\PY{p}{,}\PY{p}{(}\PY{n}{h}\PY{o}{\PYZhy{}}\PY{n}{y}\PY{p}{)}\PY{p}{)}\PY{p}{)}
        
    \PY{c+c1}{\PYZsh{}\PYZsh{}\PYZsh{} END CODE HERE \PYZsh{}\PYZsh{}\PYZsh{}}
    \PY{n}{J} \PY{o}{=} \PY{n+nb}{float}\PY{p}{(}\PY{n}{J}\PY{p}{)}
    \PY{k}{return} \PY{n}{J}\PY{p}{,} \PY{n}{theta}
\end{Verbatim}
\end{tcolorbox}

    \begin{tcolorbox}[breakable, size=fbox, boxrule=1pt, pad at break*=1mm,colback=cellbackground, colframe=cellborder]
\prompt{In}{incolor}{22}{\boxspacing}
\begin{Verbatim}[commandchars=\\\{\}]
\PY{c+c1}{\PYZsh{} Check the function}
\PY{c+c1}{\PYZsh{} Construct a synthetic test case using numpy PRNG functions}
\PY{n}{np}\PY{o}{.}\PY{n}{random}\PY{o}{.}\PY{n}{seed}\PY{p}{(}\PY{l+m+mi}{1}\PY{p}{)}
\PY{c+c1}{\PYZsh{} X input is 10 x 3 with ones for the bias terms}
\PY{n}{tmp\PYZus{}X} \PY{o}{=} \PY{n}{np}\PY{o}{.}\PY{n}{append}\PY{p}{(}\PY{n}{np}\PY{o}{.}\PY{n}{ones}\PY{p}{(}\PY{p}{(}\PY{l+m+mi}{10}\PY{p}{,} \PY{l+m+mi}{1}\PY{p}{)}\PY{p}{)}\PY{p}{,} \PY{n}{np}\PY{o}{.}\PY{n}{random}\PY{o}{.}\PY{n}{rand}\PY{p}{(}\PY{l+m+mi}{10}\PY{p}{,} \PY{l+m+mi}{2}\PY{p}{)} \PY{o}{*} \PY{l+m+mi}{2000}\PY{p}{,} \PY{n}{axis}\PY{o}{=}\PY{l+m+mi}{1}\PY{p}{)}
\PY{c+c1}{\PYZsh{} Y Labels are 10 x 1}
\PY{n}{tmp\PYZus{}Y} \PY{o}{=} \PY{p}{(}\PY{n}{np}\PY{o}{.}\PY{n}{random}\PY{o}{.}\PY{n}{rand}\PY{p}{(}\PY{l+m+mi}{10}\PY{p}{,} \PY{l+m+mi}{1}\PY{p}{)} \PY{o}{\PYZgt{}} \PY{l+m+mf}{0.35}\PY{p}{)}\PY{o}{.}\PY{n}{astype}\PY{p}{(}\PY{n+nb}{float}\PY{p}{)}

\PY{c+c1}{\PYZsh{} Apply gradient descent}
\PY{n}{tmp\PYZus{}J}\PY{p}{,} \PY{n}{tmp\PYZus{}theta} \PY{o}{=} \PY{n}{gradientDescent}\PY{p}{(}\PY{n}{tmp\PYZus{}X}\PY{p}{,} \PY{n}{tmp\PYZus{}Y}\PY{p}{,} \PY{n}{np}\PY{o}{.}\PY{n}{zeros}\PY{p}{(}\PY{p}{(}\PY{l+m+mi}{3}\PY{p}{,} \PY{l+m+mi}{1}\PY{p}{)}\PY{p}{)}\PY{p}{,} \PY{l+m+mf}{1e\PYZhy{}8}\PY{p}{,} \PY{l+m+mi}{700}\PY{p}{)}
\PY{n+nb}{print}\PY{p}{(}\PY{l+s+sa}{f}\PY{l+s+s2}{\PYZdq{}}\PY{l+s+s2}{The cost after training is }\PY{l+s+si}{\PYZob{}}\PY{n}{tmp\PYZus{}J}\PY{l+s+si}{:}\PY{l+s+s2}{.8f}\PY{l+s+si}{\PYZcb{}}\PY{l+s+s2}{.}\PY{l+s+s2}{\PYZdq{}}\PY{p}{)}
\PY{n+nb}{print}\PY{p}{(}\PY{l+s+sa}{f}\PY{l+s+s2}{\PYZdq{}}\PY{l+s+s2}{The resulting vector of weights is }\PY{l+s+si}{\PYZob{}}\PY{p}{[}\PY{n+nb}{round}\PY{p}{(}\PY{n}{t}\PY{p}{,} \PY{l+m+mi}{8}\PY{p}{)} \PY{k}{for} \PY{n}{t} \PY{o+ow}{in} \PY{n}{np}\PY{o}{.}\PY{n}{squeeze}\PY{p}{(}\PY{n}{tmp\PYZus{}theta}\PY{p}{)}\PY{p}{]}\PY{l+s+si}{\PYZcb{}}\PY{l+s+s2}{\PYZdq{}}\PY{p}{)}
\end{Verbatim}
\end{tcolorbox}

    \begin{Verbatim}[commandchars=\\\{\}]
The cost after training is 0.67094970.
The resulting vector of weights is [4.1e-07, 0.00035658, 7.309e-05]
    \end{Verbatim}

    \hypertarget{expected-output}{%
\paragraph{Expected output}\label{expected-output}}

\begin{verbatim}
The cost after training is 0.67094970.
The resulting vector of weights is [4.1e-07, 0.00035658, 7.309e-05]
\end{verbatim}

    \begin{tcolorbox}[breakable, size=fbox, boxrule=1pt, pad at break*=1mm,colback=cellbackground, colframe=cellborder]
\prompt{In}{incolor}{23}{\boxspacing}
\begin{Verbatim}[commandchars=\\\{\}]
\PY{c+c1}{\PYZsh{} Test your function}
\PY{n}{w1\PYZus{}unittest}\PY{o}{.}\PY{n}{test\PYZus{}gradientDescent}\PY{p}{(}\PY{n}{gradientDescent}\PY{p}{)}
\end{Verbatim}
\end{tcolorbox}

    \begin{Verbatim}[commandchars=\\\{\}]
\textcolor{ansi-green-intense}{ All tests passed}
    \end{Verbatim}

    \#\# 2 - Extracting the Features

\begin{itemize}
\tightlist
\item
  Given a list of tweets, extract the features and store them in a
  matrix. You will extract two features.

  \begin{itemize}
  \tightlist
  \item
    The first feature is the number of positive words in a tweet.
  \item
    The second feature is the number of negative words in a tweet.
  \end{itemize}
\item
  Then train your logistic regression classifier on these features.
\item
  Test the classifier on a validation set.
\end{itemize}

\#\#\# Exercise 3 - extract\_features Implement the extract\_features
function. * This function takes in a single tweet. * Process the tweet
using the imported \texttt{process\_tweet} function and save the list of
tweet words. * Loop through each word in the list of processed words *
For each word, check the `freqs' dictionary for the count when that word
has a positive `1' label. (Check for the key (word, 1.0) * Do the same
for the count for when the word is associated with the negative label
`0'. (Check for the key (word, 0.0).)

\textbf{Note:} In the implementation instructions provided above, the
prediction of being positive or negative depends on feature vector which
counts-in duplicate words - this is different from what you have seen in
the lecture videos

    Hints

Make sure you handle cases when the (word, label) key is not found in
the dictionary.

Search the web for hints about using the `get' function of a Python
dictionary. Here is an example

    \begin{tcolorbox}[breakable, size=fbox, boxrule=1pt, pad at break*=1mm,colback=cellbackground, colframe=cellborder]
\prompt{In}{incolor}{28}{\boxspacing}
\begin{Verbatim}[commandchars=\\\{\}]
\PY{c+c1}{\PYZsh{} UNQ\PYZus{}C3 GRADED FUNCTION: extract\PYZus{}features}
\PY{k}{def} \PY{n+nf}{extract\PYZus{}features}\PY{p}{(}\PY{n}{tweet}\PY{p}{,} \PY{n}{freqs}\PY{p}{,} \PY{n}{process\PYZus{}tweet}\PY{o}{=}\PY{n}{process\PYZus{}tweet}\PY{p}{)}\PY{p}{:}
    \PY{l+s+sd}{\PYZsq{}\PYZsq{}\PYZsq{}}
\PY{l+s+sd}{    Input: }
\PY{l+s+sd}{        tweet: a string containing one tweet}
\PY{l+s+sd}{        freqs: a dictionary corresponding to the frequencies of each tuple (word, label)}
\PY{l+s+sd}{    Output: }
\PY{l+s+sd}{        x: a feature vector of dimension (1,3)}
\PY{l+s+sd}{    \PYZsq{}\PYZsq{}\PYZsq{}}
    \PY{c+c1}{\PYZsh{} process\PYZus{}tweet tokenizes, stems, and removes stopwords}
    \PY{n}{word\PYZus{}l} \PY{o}{=} \PY{n}{process\PYZus{}tweet}\PY{p}{(}\PY{n}{tweet}\PY{p}{)}
    
    \PY{c+c1}{\PYZsh{} 3 elements for [bias, positive, negative] counts}
    \PY{n}{x} \PY{o}{=} \PY{n}{np}\PY{o}{.}\PY{n}{zeros}\PY{p}{(}\PY{l+m+mi}{3}\PY{p}{)} 
    
    \PY{c+c1}{\PYZsh{} bias term is set to 1}
    \PY{n}{x}\PY{p}{[}\PY{l+m+mi}{0}\PY{p}{]} \PY{o}{=} \PY{l+m+mi}{1} 
    
    \PY{c+c1}{\PYZsh{}\PYZsh{}\PYZsh{} START CODE HERE \PYZsh{}\PYZsh{}\PYZsh{}}
    
    \PY{c+c1}{\PYZsh{} loop through each word in the list of words}
    \PY{k}{for} \PY{n}{word} \PY{o+ow}{in} \PY{n}{word\PYZus{}l}\PY{p}{:}
        
        \PY{c+c1}{\PYZsh{} increment the word count for the positive label 1}
        \PY{n}{x}\PY{p}{[}\PY{l+m+mi}{1}\PY{p}{]} \PY{o}{+}\PY{o}{=} \PY{n}{freqs}\PY{o}{.}\PY{n}{get}\PY{p}{(}\PY{p}{(}\PY{n}{word}\PY{p}{,} \PY{l+m+mf}{1.0}\PY{p}{)}\PY{p}{,}\PY{l+m+mi}{0}\PY{p}{)}
        
        \PY{c+c1}{\PYZsh{} increment the word count for the negative label 0}
        \PY{n}{x}\PY{p}{[}\PY{l+m+mi}{2}\PY{p}{]} \PY{o}{+}\PY{o}{=} \PY{n}{freqs}\PY{o}{.}\PY{n}{get}\PY{p}{(}\PY{p}{(}\PY{n}{word}\PY{p}{,} \PY{l+m+mf}{0.0}\PY{p}{)}\PY{p}{,}\PY{l+m+mi}{0}\PY{p}{)}
        
    \PY{c+c1}{\PYZsh{}\PYZsh{}\PYZsh{} END CODE HERE \PYZsh{}\PYZsh{}\PYZsh{}}
    
    \PY{n}{x} \PY{o}{=} \PY{n}{x}\PY{p}{[}\PY{k+kc}{None}\PY{p}{,} \PY{p}{:}\PY{p}{]}  \PY{c+c1}{\PYZsh{} adding batch dimension for further processing}
    \PY{k}{assert}\PY{p}{(}\PY{n}{x}\PY{o}{.}\PY{n}{shape} \PY{o}{==} \PY{p}{(}\PY{l+m+mi}{1}\PY{p}{,} \PY{l+m+mi}{3}\PY{p}{)}\PY{p}{)}
    \PY{k}{return} \PY{n}{x}
\end{Verbatim}
\end{tcolorbox}

    \begin{tcolorbox}[breakable, size=fbox, boxrule=1pt, pad at break*=1mm,colback=cellbackground, colframe=cellborder]
\prompt{In}{incolor}{29}{\boxspacing}
\begin{Verbatim}[commandchars=\\\{\}]
\PY{c+c1}{\PYZsh{} Check your function}
\PY{c+c1}{\PYZsh{} test 1}
\PY{c+c1}{\PYZsh{} test on training data}
\PY{n}{tmp1} \PY{o}{=} \PY{n}{extract\PYZus{}features}\PY{p}{(}\PY{n}{train\PYZus{}x}\PY{p}{[}\PY{l+m+mi}{0}\PY{p}{]}\PY{p}{,} \PY{n}{freqs}\PY{p}{)}
\PY{n+nb}{print}\PY{p}{(}\PY{n}{tmp1}\PY{p}{)}
\end{Verbatim}
\end{tcolorbox}

    \begin{Verbatim}[commandchars=\\\{\}]
[[1.000e+00 3.133e+03 6.100e+01]]
    \end{Verbatim}

    \hypertarget{expected-output}{%
\paragraph{Expected output}\label{expected-output}}

\begin{verbatim}
[[1.000e+00 3.133e+03 6.100e+01]]
\end{verbatim}

    \begin{tcolorbox}[breakable, size=fbox, boxrule=1pt, pad at break*=1mm,colback=cellbackground, colframe=cellborder]
\prompt{In}{incolor}{30}{\boxspacing}
\begin{Verbatim}[commandchars=\\\{\}]
\PY{c+c1}{\PYZsh{} test 2:}
\PY{c+c1}{\PYZsh{} check for when the words are not in the freqs dictionary}
\PY{n}{tmp2} \PY{o}{=} \PY{n}{extract\PYZus{}features}\PY{p}{(}\PY{l+s+s1}{\PYZsq{}}\PY{l+s+s1}{blorb bleeeeb bloooob}\PY{l+s+s1}{\PYZsq{}}\PY{p}{,} \PY{n}{freqs}\PY{p}{)}
\PY{n+nb}{print}\PY{p}{(}\PY{n}{tmp2}\PY{p}{)}
\end{Verbatim}
\end{tcolorbox}

    \begin{Verbatim}[commandchars=\\\{\}]
[[1. 0. 0.]]
    \end{Verbatim}

    \hypertarget{expected-output}{%
\paragraph{Expected output}\label{expected-output}}

\begin{verbatim}
[[1. 0. 0.]]
\end{verbatim}

    \begin{tcolorbox}[breakable, size=fbox, boxrule=1pt, pad at break*=1mm,colback=cellbackground, colframe=cellborder]
\prompt{In}{incolor}{31}{\boxspacing}
\begin{Verbatim}[commandchars=\\\{\}]
\PY{c+c1}{\PYZsh{} Test your function}
\PY{n}{w1\PYZus{}unittest}\PY{o}{.}\PY{n}{test\PYZus{}extract\PYZus{}features}\PY{p}{(}\PY{n}{extract\PYZus{}features}\PY{p}{,} \PY{n}{freqs}\PY{p}{)}
\end{Verbatim}
\end{tcolorbox}

    \begin{Verbatim}[commandchars=\\\{\}]
\textcolor{ansi-green-intense}{ All tests passed}
    \end{Verbatim}

    \#\# 3 - Training Your Model

To train the model: * Stack the features for all training examples into
a matrix X. * Call \texttt{gradientDescent}, which you've implemented
above.

This section is given to you. Please read it for understanding and run
the cell.

    \begin{tcolorbox}[breakable, size=fbox, boxrule=1pt, pad at break*=1mm,colback=cellbackground, colframe=cellborder]
\prompt{In}{incolor}{32}{\boxspacing}
\begin{Verbatim}[commandchars=\\\{\}]
\PY{c+c1}{\PYZsh{} collect the features \PYZsq{}x\PYZsq{} and stack them into a matrix \PYZsq{}X\PYZsq{}}
\PY{n}{X} \PY{o}{=} \PY{n}{np}\PY{o}{.}\PY{n}{zeros}\PY{p}{(}\PY{p}{(}\PY{n+nb}{len}\PY{p}{(}\PY{n}{train\PYZus{}x}\PY{p}{)}\PY{p}{,} \PY{l+m+mi}{3}\PY{p}{)}\PY{p}{)}
\PY{k}{for} \PY{n}{i} \PY{o+ow}{in} \PY{n+nb}{range}\PY{p}{(}\PY{n+nb}{len}\PY{p}{(}\PY{n}{train\PYZus{}x}\PY{p}{)}\PY{p}{)}\PY{p}{:}
    \PY{n}{X}\PY{p}{[}\PY{n}{i}\PY{p}{,} \PY{p}{:}\PY{p}{]}\PY{o}{=} \PY{n}{extract\PYZus{}features}\PY{p}{(}\PY{n}{train\PYZus{}x}\PY{p}{[}\PY{n}{i}\PY{p}{]}\PY{p}{,} \PY{n}{freqs}\PY{p}{)}

\PY{c+c1}{\PYZsh{} training labels corresponding to X}
\PY{n}{Y} \PY{o}{=} \PY{n}{train\PYZus{}y}

\PY{c+c1}{\PYZsh{} Apply gradient descent}
\PY{n}{J}\PY{p}{,} \PY{n}{theta} \PY{o}{=} \PY{n}{gradientDescent}\PY{p}{(}\PY{n}{X}\PY{p}{,} \PY{n}{Y}\PY{p}{,} \PY{n}{np}\PY{o}{.}\PY{n}{zeros}\PY{p}{(}\PY{p}{(}\PY{l+m+mi}{3}\PY{p}{,} \PY{l+m+mi}{1}\PY{p}{)}\PY{p}{)}\PY{p}{,} \PY{l+m+mf}{1e\PYZhy{}9}\PY{p}{,} \PY{l+m+mi}{1500}\PY{p}{)}
\PY{n+nb}{print}\PY{p}{(}\PY{l+s+sa}{f}\PY{l+s+s2}{\PYZdq{}}\PY{l+s+s2}{The cost after training is }\PY{l+s+si}{\PYZob{}}\PY{n}{J}\PY{l+s+si}{:}\PY{l+s+s2}{.8f}\PY{l+s+si}{\PYZcb{}}\PY{l+s+s2}{.}\PY{l+s+s2}{\PYZdq{}}\PY{p}{)}
\PY{n+nb}{print}\PY{p}{(}\PY{l+s+sa}{f}\PY{l+s+s2}{\PYZdq{}}\PY{l+s+s2}{The resulting vector of weights is }\PY{l+s+si}{\PYZob{}}\PY{p}{[}\PY{n+nb}{round}\PY{p}{(}\PY{n}{t}\PY{p}{,} \PY{l+m+mi}{8}\PY{p}{)} \PY{k}{for} \PY{n}{t} \PY{o+ow}{in} \PY{n}{np}\PY{o}{.}\PY{n}{squeeze}\PY{p}{(}\PY{n}{theta}\PY{p}{)}\PY{p}{]}\PY{l+s+si}{\PYZcb{}}\PY{l+s+s2}{\PYZdq{}}\PY{p}{)}
\end{Verbatim}
\end{tcolorbox}

    \begin{Verbatim}[commandchars=\\\{\}]
The cost after training is 0.22522315.
The resulting vector of weights is [6e-08, 0.00053818, -0.0005583]
    \end{Verbatim}

    \textbf{Expected Output}:

\begin{verbatim}
The cost after training is 0.22522315.
The resulting vector of weights is [6e-08, 0.00053818, -0.0005583]
\end{verbatim}

    \#\# 4 - Test your Logistic Regression

It is time for you to test your logistic regression function on some new
input that your model has not seen before. \#\#\# Exercise 4 -
predict\_tweet Implement \texttt{predict\_tweet}. Predict whether a
tweet is positive or negative.

\begin{itemize}
\tightlist
\item
  Given a tweet, process it, then extract the features.
\item
  Apply the model's learned weights on the features to get the logits.
\item
  Apply the sigmoid to the logits to get the prediction (a value between
  0 and 1).
\end{itemize}

\[y_{pred} = sigmoid(\mathbf{x} \cdot \theta)\]

    \begin{tcolorbox}[breakable, size=fbox, boxrule=1pt, pad at break*=1mm,colback=cellbackground, colframe=cellborder]
\prompt{In}{incolor}{33}{\boxspacing}
\begin{Verbatim}[commandchars=\\\{\}]
\PY{c+c1}{\PYZsh{} UNQ\PYZus{}C4 GRADED FUNCTION: predict\PYZus{}tweet}
\PY{k}{def} \PY{n+nf}{predict\PYZus{}tweet}\PY{p}{(}\PY{n}{tweet}\PY{p}{,} \PY{n}{freqs}\PY{p}{,} \PY{n}{theta}\PY{p}{)}\PY{p}{:}
    \PY{l+s+sd}{\PYZsq{}\PYZsq{}\PYZsq{}}
\PY{l+s+sd}{    Input: }
\PY{l+s+sd}{        tweet: a string}
\PY{l+s+sd}{        freqs: a dictionary corresponding to the frequencies of each tuple (word, label)}
\PY{l+s+sd}{        theta: (3,1) vector of weights}
\PY{l+s+sd}{    Output: }
\PY{l+s+sd}{        y\PYZus{}pred: the probability of a tweet being positive or negative}
\PY{l+s+sd}{    \PYZsq{}\PYZsq{}\PYZsq{}}
    \PY{c+c1}{\PYZsh{}\PYZsh{}\PYZsh{} START CODE HERE \PYZsh{}\PYZsh{}\PYZsh{}}
    
    \PY{c+c1}{\PYZsh{} extract the features of the tweet and store it into x}
    \PY{n}{x} \PY{o}{=} \PY{n}{extract\PYZus{}features}\PY{p}{(}\PY{n}{tweet}\PY{p}{,} \PY{n}{freqs}\PY{p}{,} \PY{n}{process\PYZus{}tweet}\PY{o}{=}\PY{n}{process\PYZus{}tweet}\PY{p}{)}
    
    \PY{c+c1}{\PYZsh{} make the prediction using x and theta}
    \PY{n}{y\PYZus{}pred} \PY{o}{=} \PY{n}{sigmoid}\PY{p}{(}\PY{n}{np}\PY{o}{.}\PY{n}{dot}\PY{p}{(}\PY{n}{x}\PY{p}{,}\PY{n}{theta}\PY{p}{)}\PY{p}{)}
    
    \PY{c+c1}{\PYZsh{}\PYZsh{}\PYZsh{} END CODE HERE \PYZsh{}\PYZsh{}\PYZsh{}}
    
    \PY{k}{return} \PY{n}{y\PYZus{}pred}
\end{Verbatim}
\end{tcolorbox}

    \begin{tcolorbox}[breakable, size=fbox, boxrule=1pt, pad at break*=1mm,colback=cellbackground, colframe=cellborder]
\prompt{In}{incolor}{34}{\boxspacing}
\begin{Verbatim}[commandchars=\\\{\}]
\PY{c+c1}{\PYZsh{} Run this cell to test your function}
\PY{k}{for} \PY{n}{tweet} \PY{o+ow}{in} \PY{p}{[}\PY{l+s+s1}{\PYZsq{}}\PY{l+s+s1}{I am happy}\PY{l+s+s1}{\PYZsq{}}\PY{p}{,} \PY{l+s+s1}{\PYZsq{}}\PY{l+s+s1}{I am bad}\PY{l+s+s1}{\PYZsq{}}\PY{p}{,} \PY{l+s+s1}{\PYZsq{}}\PY{l+s+s1}{this movie should have been great.}\PY{l+s+s1}{\PYZsq{}}\PY{p}{,} \PY{l+s+s1}{\PYZsq{}}\PY{l+s+s1}{great}\PY{l+s+s1}{\PYZsq{}}\PY{p}{,} \PY{l+s+s1}{\PYZsq{}}\PY{l+s+s1}{great great}\PY{l+s+s1}{\PYZsq{}}\PY{p}{,} \PY{l+s+s1}{\PYZsq{}}\PY{l+s+s1}{great great great}\PY{l+s+s1}{\PYZsq{}}\PY{p}{,} \PY{l+s+s1}{\PYZsq{}}\PY{l+s+s1}{great great great great}\PY{l+s+s1}{\PYZsq{}}\PY{p}{]}\PY{p}{:}
    \PY{n+nb}{print}\PY{p}{(} \PY{l+s+s1}{\PYZsq{}}\PY{l+s+si}{\PYZpc{}s}\PY{l+s+s1}{ \PYZhy{}\PYZgt{} }\PY{l+s+si}{\PYZpc{}f}\PY{l+s+s1}{\PYZsq{}} \PY{o}{\PYZpc{}} \PY{p}{(}\PY{n}{tweet}\PY{p}{,} \PY{n}{predict\PYZus{}tweet}\PY{p}{(}\PY{n}{tweet}\PY{p}{,} \PY{n}{freqs}\PY{p}{,} \PY{n}{theta}\PY{p}{)}\PY{p}{)}\PY{p}{)}    
    
\end{Verbatim}
\end{tcolorbox}

    \begin{Verbatim}[commandchars=\\\{\}]
I am happy -> 0.519275
I am bad -> 0.494347
this movie should have been great. -> 0.515979
great -> 0.516065
great great -> 0.532096
great great great -> 0.548062
great great great great -> 0.563929
    \end{Verbatim}

    \textbf{Expected Output}:

\begin{verbatim}
I am happy -> 0.519275
I am bad -> 0.494347
this movie should have been great. -> 0.515979
great -> 0.516065
great great -> 0.532096
great great great -> 0.548062
great great great great -> 0.563929
\end{verbatim}

    \begin{tcolorbox}[breakable, size=fbox, boxrule=1pt, pad at break*=1mm,colback=cellbackground, colframe=cellborder]
\prompt{In}{incolor}{35}{\boxspacing}
\begin{Verbatim}[commandchars=\\\{\}]
\PY{c+c1}{\PYZsh{} Feel free to check the sentiment of your own tweet below}
\PY{n}{my\PYZus{}tweet} \PY{o}{=} \PY{l+s+s1}{\PYZsq{}}\PY{l+s+s1}{I am learning :)}\PY{l+s+s1}{\PYZsq{}}
\PY{n}{predict\PYZus{}tweet}\PY{p}{(}\PY{n}{my\PYZus{}tweet}\PY{p}{,} \PY{n}{freqs}\PY{p}{,} \PY{n}{theta}\PY{p}{)}
\end{Verbatim}
\end{tcolorbox}

            \begin{tcolorbox}[breakable, size=fbox, boxrule=.5pt, pad at break*=1mm, opacityfill=0]
\prompt{Out}{outcolor}{35}{\boxspacing}
\begin{Verbatim}[commandchars=\\\{\}]
array([[0.83110307]])
\end{Verbatim}
\end{tcolorbox}
        
    \begin{tcolorbox}[breakable, size=fbox, boxrule=1pt, pad at break*=1mm,colback=cellbackground, colframe=cellborder]
\prompt{In}{incolor}{36}{\boxspacing}
\begin{Verbatim}[commandchars=\\\{\}]
\PY{c+c1}{\PYZsh{} Test your function}
\PY{n}{w1\PYZus{}unittest}\PY{o}{.}\PY{n}{test\PYZus{}predict\PYZus{}tweet}\PY{p}{(}\PY{n}{predict\PYZus{}tweet}\PY{p}{,} \PY{n}{freqs}\PY{p}{,} \PY{n}{theta}\PY{p}{)}
\end{Verbatim}
\end{tcolorbox}

    \begin{Verbatim}[commandchars=\\\{\}]
\textcolor{ansi-green-intense}{ All tests passed}
    \end{Verbatim}

    \#\#\# 4.1 - Check the Performance using the Test Set After training
your model using the training set above, check how your model might
perform on real, unseen data, by testing it against the test set.

\#\#\# Exercise 5 - test\_logistic\_regression Implement
\texttt{test\_logistic\_regression}. * Given the test data and the
weights of your trained model, calculate the accuracy of your logistic
regression model. * Use your `predict\_tweet' function to make
predictions on each tweet in the test set. * If the prediction is
\textgreater{} 0.5, set the model's classification `y\_hat' to 1,
otherwise set the model's classification `y\_hat' to 0. * A prediction
is accurate when the y\_hat equals the test\_y. Sum up all the instances
when they are equal and divide by m.

    Hints

Use np.asarray() to convert a list to a numpy array

Use numpy.squeeze() to make an (m,1) dimensional array into an (m,)
array

    \begin{tcolorbox}[breakable, size=fbox, boxrule=1pt, pad at break*=1mm,colback=cellbackground, colframe=cellborder]
\prompt{In}{incolor}{57}{\boxspacing}
\begin{Verbatim}[commandchars=\\\{\}]
\PY{c+c1}{\PYZsh{} UNQ\PYZus{}C5 GRADED FUNCTION: test\PYZus{}logistic\PYZus{}regression}
\PY{k}{def} \PY{n+nf}{test\PYZus{}logistic\PYZus{}regression}\PY{p}{(}\PY{n}{test\PYZus{}x}\PY{p}{,} \PY{n}{test\PYZus{}y}\PY{p}{,} \PY{n}{freqs}\PY{p}{,} \PY{n}{theta}\PY{p}{,} \PY{n}{predict\PYZus{}tweet}\PY{o}{=}\PY{n}{predict\PYZus{}tweet}\PY{p}{)}\PY{p}{:}
    \PY{l+s+sd}{\PYZdq{}\PYZdq{}\PYZdq{}}
\PY{l+s+sd}{    Input: }
\PY{l+s+sd}{        test\PYZus{}x: a list of tweets}
\PY{l+s+sd}{        test\PYZus{}y: (m, 1) vector with the corresponding labels for the list of tweets}
\PY{l+s+sd}{        freqs: a dictionary with the frequency of each pair (or tuple)}
\PY{l+s+sd}{        theta: weight vector of dimension (3, 1)}
\PY{l+s+sd}{    Output: }
\PY{l+s+sd}{        accuracy: (\PYZsh{} of tweets classified correctly) / (total \PYZsh{} of tweets)}
\PY{l+s+sd}{    \PYZdq{}\PYZdq{}\PYZdq{}}
    
    \PY{c+c1}{\PYZsh{}\PYZsh{}\PYZsh{} START CODE HERE \PYZsh{}\PYZsh{}\PYZsh{}}
    
    \PY{c+c1}{\PYZsh{} the list for storing predictions}
    \PY{n}{y\PYZus{}hat} \PY{o}{=} \PY{n+nb}{list}\PY{p}{(}\PY{p}{)}
    
    \PY{k}{for} \PY{n}{tweet} \PY{o+ow}{in} \PY{n}{test\PYZus{}x}\PY{p}{:}
        \PY{c+c1}{\PYZsh{} get the label prediction for the tweet}
        \PY{n}{y\PYZus{}pred} \PY{o}{=} \PY{n}{predict\PYZus{}tweet}\PY{p}{(}\PY{n}{tweet}\PY{p}{,}\PY{n}{freqs}\PY{p}{,}\PY{n}{theta}\PY{p}{)}
        
        \PY{k}{if} \PY{n}{y\PYZus{}pred} \PY{o}{\PYZgt{}} \PY{l+m+mf}{0.5}\PY{p}{:}
            \PY{c+c1}{\PYZsh{} append 1.0 to the list}
            \PY{n}{y\PYZus{}hat}\PY{o}{.}\PY{n}{append}\PY{p}{(}\PY{l+m+mf}{1.0}\PY{p}{)}
        \PY{k}{else}\PY{p}{:}
            \PY{c+c1}{\PYZsh{} append 0 to the list}
            \PY{n}{y\PYZus{}hat}\PY{o}{.}\PY{n}{append}\PY{p}{(}\PY{l+m+mf}{0.0}\PY{p}{)}

    \PY{c+c1}{\PYZsh{} With the above implementation, y\PYZus{}hat is a list, but test\PYZus{}y is (m,1) array}
    \PY{c+c1}{\PYZsh{} convert both to one\PYZhy{}dimensional arrays in order to compare them using the \PYZsq{}==\PYZsq{} operator}
    \PY{n}{y\PYZus{}hat\PYZus{}array} \PY{o}{=} \PY{n}{np}\PY{o}{.}\PY{n}{squeeze}\PY{p}{(}\PY{n}{np}\PY{o}{.}\PY{n}{asarray}\PY{p}{(}\PY{n}{y\PYZus{}hat}\PY{p}{)}\PY{p}{)}
    \PY{n}{test\PYZus{}y\PYZus{}array} \PY{o}{=} \PY{n}{np}\PY{o}{.}\PY{n}{squeeze}\PY{p}{(}\PY{n}{test\PYZus{}y}\PY{p}{)}
    \PY{n}{m} \PY{o}{=} \PY{n}{np}\PY{o}{.}\PY{n}{shape}\PY{p}{(}\PY{n}{test\PYZus{}y}\PY{p}{)}\PY{p}{[}\PY{l+m+mi}{0}\PY{p}{]}
    \PY{n}{accuracy} \PY{o}{=} \PY{p}{(}\PY{p}{(}\PY{n}{y\PYZus{}hat} \PY{o}{==} \PY{n}{test\PYZus{}y\PYZus{}array}\PY{p}{)}\PY{o}{.}\PY{n}{sum}\PY{p}{(}\PY{p}{)}\PY{p}{)}\PY{o}{/}\PY{n}{m}

    \PY{c+c1}{\PYZsh{}\PYZsh{}\PYZsh{} END CODE HERE \PYZsh{}\PYZsh{}\PYZsh{}}
    
    \PY{k}{return} \PY{n}{accuracy}
\end{Verbatim}
\end{tcolorbox}

    \begin{tcolorbox}[breakable, size=fbox, boxrule=1pt, pad at break*=1mm,colback=cellbackground, colframe=cellborder]
\prompt{In}{incolor}{58}{\boxspacing}
\begin{Verbatim}[commandchars=\\\{\}]
\PY{n}{tmp\PYZus{}accuracy} \PY{o}{=} \PY{n}{test\PYZus{}logistic\PYZus{}regression}\PY{p}{(}\PY{n}{test\PYZus{}x}\PY{p}{,} \PY{n}{test\PYZus{}y}\PY{p}{,} \PY{n}{freqs}\PY{p}{,} \PY{n}{theta}\PY{p}{)}
\PY{n+nb}{print}\PY{p}{(}\PY{l+s+sa}{f}\PY{l+s+s2}{\PYZdq{}}\PY{l+s+s2}{Logistic regression model}\PY{l+s+s2}{\PYZsq{}}\PY{l+s+s2}{s accuracy = }\PY{l+s+si}{\PYZob{}}\PY{n}{tmp\PYZus{}accuracy}\PY{l+s+si}{:}\PY{l+s+s2}{.4f}\PY{l+s+si}{\PYZcb{}}\PY{l+s+s2}{\PYZdq{}}\PY{p}{)}
\end{Verbatim}
\end{tcolorbox}

    \begin{Verbatim}[commandchars=\\\{\}]
Logistic regression model's accuracy = 0.9950
    \end{Verbatim}

    \hypertarget{expected-output}{%
\paragraph{Expected Output:}\label{expected-output}}

\texttt{0.9950}~\\
Pretty good!

    \begin{tcolorbox}[breakable, size=fbox, boxrule=1pt, pad at break*=1mm,colback=cellbackground, colframe=cellborder]
\prompt{In}{incolor}{59}{\boxspacing}
\begin{Verbatim}[commandchars=\\\{\}]
\PY{c+c1}{\PYZsh{} Test your function}
\PY{n}{w1\PYZus{}unittest}\PY{o}{.}\PY{n}{unittest\PYZus{}test\PYZus{}logistic\PYZus{}regression}\PY{p}{(}\PY{n}{test\PYZus{}logistic\PYZus{}regression}\PY{p}{,} \PY{n}{freqs}\PY{p}{,} \PY{n}{theta}\PY{p}{)}
\end{Verbatim}
\end{tcolorbox}

    \begin{Verbatim}[commandchars=\\\{\}]
\textcolor{ansi-green-intense}{ All tests passed}
    \end{Verbatim}

    \#\# 5 - Error Analysis

In this part you will see some tweets that your model misclassified. Why
do you think the misclassifications happened? Specifically what kind of
tweets does your model misclassify?

    \begin{tcolorbox}[breakable, size=fbox, boxrule=1pt, pad at break*=1mm,colback=cellbackground, colframe=cellborder]
\prompt{In}{incolor}{60}{\boxspacing}
\begin{Verbatim}[commandchars=\\\{\}]
\PY{c+c1}{\PYZsh{} Some error analysis done for you}
\PY{n+nb}{print}\PY{p}{(}\PY{l+s+s1}{\PYZsq{}}\PY{l+s+s1}{Label Predicted Tweet}\PY{l+s+s1}{\PYZsq{}}\PY{p}{)}
\PY{k}{for} \PY{n}{x}\PY{p}{,}\PY{n}{y} \PY{o+ow}{in} \PY{n+nb}{zip}\PY{p}{(}\PY{n}{test\PYZus{}x}\PY{p}{,}\PY{n}{test\PYZus{}y}\PY{p}{)}\PY{p}{:}
    \PY{n}{y\PYZus{}hat} \PY{o}{=} \PY{n}{predict\PYZus{}tweet}\PY{p}{(}\PY{n}{x}\PY{p}{,} \PY{n}{freqs}\PY{p}{,} \PY{n}{theta}\PY{p}{)}

    \PY{k}{if} \PY{n}{np}\PY{o}{.}\PY{n}{abs}\PY{p}{(}\PY{n}{y} \PY{o}{\PYZhy{}} \PY{p}{(}\PY{n}{y\PYZus{}hat} \PY{o}{\PYZgt{}} \PY{l+m+mf}{0.5}\PY{p}{)}\PY{p}{)} \PY{o}{\PYZgt{}} \PY{l+m+mi}{0}\PY{p}{:}
        \PY{n+nb}{print}\PY{p}{(}\PY{l+s+s1}{\PYZsq{}}\PY{l+s+s1}{THE TWEET IS:}\PY{l+s+s1}{\PYZsq{}}\PY{p}{,} \PY{n}{x}\PY{p}{)}
        \PY{n+nb}{print}\PY{p}{(}\PY{l+s+s1}{\PYZsq{}}\PY{l+s+s1}{THE PROCESSED TWEET IS:}\PY{l+s+s1}{\PYZsq{}}\PY{p}{,} \PY{n}{process\PYZus{}tweet}\PY{p}{(}\PY{n}{x}\PY{p}{)}\PY{p}{)}
        \PY{n+nb}{print}\PY{p}{(}\PY{l+s+s1}{\PYZsq{}}\PY{l+s+si}{\PYZpc{}d}\PY{l+s+se}{\PYZbs{}t}\PY{l+s+si}{\PYZpc{}0.8f}\PY{l+s+se}{\PYZbs{}t}\PY{l+s+si}{\PYZpc{}s}\PY{l+s+s1}{\PYZsq{}} \PY{o}{\PYZpc{}} \PY{p}{(}\PY{n}{y}\PY{p}{,} \PY{n}{y\PYZus{}hat}\PY{p}{,} \PY{l+s+s1}{\PYZsq{}}\PY{l+s+s1}{ }\PY{l+s+s1}{\PYZsq{}}\PY{o}{.}\PY{n}{join}\PY{p}{(}\PY{n}{process\PYZus{}tweet}\PY{p}{(}\PY{n}{x}\PY{p}{)}\PY{p}{)}\PY{o}{.}\PY{n}{encode}\PY{p}{(}\PY{l+s+s1}{\PYZsq{}}\PY{l+s+s1}{ascii}\PY{l+s+s1}{\PYZsq{}}\PY{p}{,} \PY{l+s+s1}{\PYZsq{}}\PY{l+s+s1}{ignore}\PY{l+s+s1}{\PYZsq{}}\PY{p}{)}\PY{p}{)}\PY{p}{)}
\end{Verbatim}
\end{tcolorbox}

    \begin{Verbatim}[commandchars=\\\{\}]
Label Predicted Tweet
THE TWEET IS: @MarkBreech Not sure it would be good thing 4 my bottom daring 2
say 2 Miss B but Im gonna be so stubborn on mouth soaping ! \#NotHavingit :p
THE PROCESSED TWEET IS: ['sure', 'would', 'good', 'thing', '4', 'bottom',
'dare', '2', 'say', '2', 'miss', 'b', 'im', 'gonna', 'stubborn', 'mouth',
'soap', 'nothavingit', ':p']
1       0.48901497      b'sure would good thing 4 bottom dare 2 say 2 miss b im
gonna stubborn mouth soap nothavingit :p'
THE TWEET IS: I'm playing Brain Dots : ) \#BrainDots
http://t.co/UGQzOx0huu
THE PROCESSED TWEET IS: ["i'm", 'play', 'brain', 'dot', 'braindot']
1       0.48418949      b"i'm play brain dot braindot"
THE TWEET IS: I'm playing Brain Dots : ) \#BrainDots http://t.co/aOKldo3GMj
http://t.co/xWCM9qyRG5
THE PROCESSED TWEET IS: ["i'm", 'play', 'brain', 'dot', 'braindot']
1       0.48418949      b"i'm play brain dot braindot"
THE TWEET IS: I'm playing Brain Dots : ) \#BrainDots http://t.co/R2JBO8iNww
http://t.co/ow5BBwdEMY
THE PROCESSED TWEET IS: ["i'm", 'play', 'brain', 'dot', 'braindot']
1       0.48418949      b"i'm play brain dot braindot"
THE TWEET IS: off to the park to get some sunlight : )
THE PROCESSED TWEET IS: ['park', 'get', 'sunlight']
1       0.49636374      b'park get sunlight'
THE TWEET IS: @msarosh Uff Itna Miss karhy thy ap :p
THE PROCESSED TWEET IS: ['uff', 'itna', 'miss', 'karhi', 'thi', 'ap', ':p']
1       0.48237069      b'uff itna miss karhi thi ap :p'
THE TWEET IS: @phenomyoutube u probs had more fun with david than me : (
THE PROCESSED TWEET IS: ['u', 'prob', 'fun', 'david']
0       0.50988239      b'u prob fun david'
THE TWEET IS: pats jay : (
THE PROCESSED TWEET IS: ['pat', 'jay']
0       0.50040365      b'pat jay'
THE TWEET IS: my beloved grandmother : ( https://t.co/wt4oXq5xCf
THE PROCESSED TWEET IS: ['belov', 'grandmoth']
0       0.50000002      b'belov grandmoth'
THE TWEET IS: Sr. Financial Analyst - Expedia, Inc.: (\#Bellevue, WA)
http://t.co/ktknMhvwCI \#Finance \#ExpediaJobs \#Job \#Jobs \#Hiring
THE PROCESSED TWEET IS: ['sr', 'financi', 'analyst', 'expedia', 'inc',
'bellevu', 'wa', 'financ', 'expediajob', 'job', 'job', 'hire']
0       0.50648681      b'sr financi analyst expedia inc bellevu wa financ
expediajob job job hire'
    \end{Verbatim}

    Later in this specialization, we will see how we can use deeplearning to
improve the prediction performance.

    \#\# 6 - Predict with your own Tweet

    \begin{tcolorbox}[breakable, size=fbox, boxrule=1pt, pad at break*=1mm,colback=cellbackground, colframe=cellborder]
\prompt{In}{incolor}{70}{\boxspacing}
\begin{Verbatim}[commandchars=\\\{\}]
\PY{c+c1}{\PYZsh{} Feel free to change the tweet below}
\PY{n}{my\PYZus{}tweet} \PY{o}{=} \PY{l+s+s1}{\PYZsq{}}\PY{l+s+s1}{This is a tremendously cry movie. The plot was terrible and I was sad until the ending!}\PY{l+s+s1}{\PYZsq{}}
\PY{n+nb}{print}\PY{p}{(}\PY{n}{process\PYZus{}tweet}\PY{p}{(}\PY{n}{my\PYZus{}tweet}\PY{p}{)}\PY{p}{)}
\PY{n}{y\PYZus{}hat} \PY{o}{=} \PY{n}{predict\PYZus{}tweet}\PY{p}{(}\PY{n}{my\PYZus{}tweet}\PY{p}{,} \PY{n}{freqs}\PY{p}{,} \PY{n}{theta}\PY{p}{)}
\PY{n+nb}{print}\PY{p}{(}\PY{n}{y\PYZus{}hat}\PY{p}{)}
\PY{k}{if} \PY{n}{y\PYZus{}hat} \PY{o}{\PYZgt{}} \PY{l+m+mf}{0.5}\PY{p}{:}
    \PY{n+nb}{print}\PY{p}{(}\PY{l+s+s1}{\PYZsq{}}\PY{l+s+s1}{Positive sentiment}\PY{l+s+s1}{\PYZsq{}}\PY{p}{)}
\PY{k}{else}\PY{p}{:} 
    \PY{n+nb}{print}\PY{p}{(}\PY{l+s+s1}{\PYZsq{}}\PY{l+s+s1}{Negative sentiment}\PY{l+s+s1}{\PYZsq{}}\PY{p}{)}
\end{Verbatim}
\end{tcolorbox}

    \begin{Verbatim}[commandchars=\\\{\}]
['depress', 'cri', 'movi', 'plot', 'terribl', 'sad', 'end', 'xd']
[[0.47633275]]
Negative sentiment
    \end{Verbatim}

    \begin{tcolorbox}[breakable, size=fbox, boxrule=1pt, pad at break*=1mm,colback=cellbackground, colframe=cellborder]
\prompt{In}{incolor}{ }{\boxspacing}
\begin{Verbatim}[commandchars=\\\{\}]

\end{Verbatim}
\end{tcolorbox}


    % Add a bibliography block to the postdoc
    
    
    
\end{document}
