\documentclass[11pt]{article}

    \usepackage[breakable]{tcolorbox}
    \usepackage{parskip} % Stop auto-indenting (to mimic markdown behaviour)
    
    \usepackage{iftex}
    \ifPDFTeX
    	\usepackage[T1]{fontenc}
    	\usepackage{mathpazo}
    \else
    	\usepackage{fontspec}
    \fi

    % Basic figure setup, for now with no caption control since it's done
    % automatically by Pandoc (which extracts ![](path) syntax from Markdown).
    \usepackage{graphicx}
    % Maintain compatibility with old templates. Remove in nbconvert 6.0
    \let\Oldincludegraphics\includegraphics
    % Ensure that by default, figures have no caption (until we provide a
    % proper Figure object with a Caption API and a way to capture that
    % in the conversion process - todo).
    \usepackage{caption}
    \DeclareCaptionFormat{nocaption}{}
    \captionsetup{format=nocaption,aboveskip=0pt,belowskip=0pt}

    \usepackage[Export]{adjustbox} % Used to constrain images to a maximum size
    \adjustboxset{max size={0.9\linewidth}{0.9\paperheight}}
    \usepackage{float}
    \floatplacement{figure}{H} % forces figures to be placed at the correct location
    \usepackage{xcolor} % Allow colors to be defined
    \usepackage{enumerate} % Needed for markdown enumerations to work
    \usepackage{geometry} % Used to adjust the document margins
    \usepackage{amsmath} % Equations
    \usepackage{amssymb} % Equations
    \usepackage{textcomp} % defines textquotesingle
    % Hack from http://tex.stackexchange.com/a/47451/13684:
    \AtBeginDocument{%
        \def\PYZsq{\textquotesingle}% Upright quotes in Pygmentized code
    }
    \usepackage{upquote} % Upright quotes for verbatim code
    \usepackage{eurosym} % defines \euro
    \usepackage[mathletters]{ucs} % Extended unicode (utf-8) support
    \usepackage{fancyvrb} % verbatim replacement that allows latex
    \usepackage{grffile} % extends the file name processing of package graphics 
                         % to support a larger range
    \makeatletter % fix for grffile with XeLaTeX
    \def\Gread@@xetex#1{%
      \IfFileExists{"\Gin@base".bb}%
      {\Gread@eps{\Gin@base.bb}}%
      {\Gread@@xetex@aux#1}%
    }
    \makeatother

    % The hyperref package gives us a pdf with properly built
    % internal navigation ('pdf bookmarks' for the table of contents,
    % internal cross-reference links, web links for URLs, etc.)
    \usepackage{hyperref}
    % The default LaTeX title has an obnoxious amount of whitespace. By default,
    % titling removes some of it. It also provides customization options.
    \usepackage{titling}
    \usepackage{longtable} % longtable support required by pandoc >1.10
    \usepackage{booktabs}  % table support for pandoc > 1.12.2
    \usepackage[inline]{enumitem} % IRkernel/repr support (it uses the enumerate* environment)
    \usepackage[normalem]{ulem} % ulem is needed to support strikethroughs (\sout)
                                % normalem makes italics be italics, not underlines
    \usepackage{mathrsfs}
    

    
    % Colors for the hyperref package
    \definecolor{urlcolor}{rgb}{0,.145,.698}
    \definecolor{linkcolor}{rgb}{.71,0.21,0.01}
    \definecolor{citecolor}{rgb}{.12,.54,.11}

    % ANSI colors
    \definecolor{ansi-black}{HTML}{3E424D}
    \definecolor{ansi-black-intense}{HTML}{282C36}
    \definecolor{ansi-red}{HTML}{E75C58}
    \definecolor{ansi-red-intense}{HTML}{B22B31}
    \definecolor{ansi-green}{HTML}{00A250}
    \definecolor{ansi-green-intense}{HTML}{007427}
    \definecolor{ansi-yellow}{HTML}{DDB62B}
    \definecolor{ansi-yellow-intense}{HTML}{B27D12}
    \definecolor{ansi-blue}{HTML}{208FFB}
    \definecolor{ansi-blue-intense}{HTML}{0065CA}
    \definecolor{ansi-magenta}{HTML}{D160C4}
    \definecolor{ansi-magenta-intense}{HTML}{A03196}
    \definecolor{ansi-cyan}{HTML}{60C6C8}
    \definecolor{ansi-cyan-intense}{HTML}{258F8F}
    \definecolor{ansi-white}{HTML}{C5C1B4}
    \definecolor{ansi-white-intense}{HTML}{A1A6B2}
    \definecolor{ansi-default-inverse-fg}{HTML}{FFFFFF}
    \definecolor{ansi-default-inverse-bg}{HTML}{000000}

    % commands and environments needed by pandoc snippets
    % extracted from the output of `pandoc -s`
    \providecommand{\tightlist}{%
      \setlength{\itemsep}{0pt}\setlength{\parskip}{0pt}}
    \DefineVerbatimEnvironment{Highlighting}{Verbatim}{commandchars=\\\{\}}
    % Add ',fontsize=\small' for more characters per line
    \newenvironment{Shaded}{}{}
    \newcommand{\KeywordTok}[1]{\textcolor[rgb]{0.00,0.44,0.13}{\textbf{{#1}}}}
    \newcommand{\DataTypeTok}[1]{\textcolor[rgb]{0.56,0.13,0.00}{{#1}}}
    \newcommand{\DecValTok}[1]{\textcolor[rgb]{0.25,0.63,0.44}{{#1}}}
    \newcommand{\BaseNTok}[1]{\textcolor[rgb]{0.25,0.63,0.44}{{#1}}}
    \newcommand{\FloatTok}[1]{\textcolor[rgb]{0.25,0.63,0.44}{{#1}}}
    \newcommand{\CharTok}[1]{\textcolor[rgb]{0.25,0.44,0.63}{{#1}}}
    \newcommand{\StringTok}[1]{\textcolor[rgb]{0.25,0.44,0.63}{{#1}}}
    \newcommand{\CommentTok}[1]{\textcolor[rgb]{0.38,0.63,0.69}{\textit{{#1}}}}
    \newcommand{\OtherTok}[1]{\textcolor[rgb]{0.00,0.44,0.13}{{#1}}}
    \newcommand{\AlertTok}[1]{\textcolor[rgb]{1.00,0.00,0.00}{\textbf{{#1}}}}
    \newcommand{\FunctionTok}[1]{\textcolor[rgb]{0.02,0.16,0.49}{{#1}}}
    \newcommand{\RegionMarkerTok}[1]{{#1}}
    \newcommand{\ErrorTok}[1]{\textcolor[rgb]{1.00,0.00,0.00}{\textbf{{#1}}}}
    \newcommand{\NormalTok}[1]{{#1}}
    
    % Additional commands for more recent versions of Pandoc
    \newcommand{\ConstantTok}[1]{\textcolor[rgb]{0.53,0.00,0.00}{{#1}}}
    \newcommand{\SpecialCharTok}[1]{\textcolor[rgb]{0.25,0.44,0.63}{{#1}}}
    \newcommand{\VerbatimStringTok}[1]{\textcolor[rgb]{0.25,0.44,0.63}{{#1}}}
    \newcommand{\SpecialStringTok}[1]{\textcolor[rgb]{0.73,0.40,0.53}{{#1}}}
    \newcommand{\ImportTok}[1]{{#1}}
    \newcommand{\DocumentationTok}[1]{\textcolor[rgb]{0.73,0.13,0.13}{\textit{{#1}}}}
    \newcommand{\AnnotationTok}[1]{\textcolor[rgb]{0.38,0.63,0.69}{\textbf{\textit{{#1}}}}}
    \newcommand{\CommentVarTok}[1]{\textcolor[rgb]{0.38,0.63,0.69}{\textbf{\textit{{#1}}}}}
    \newcommand{\VariableTok}[1]{\textcolor[rgb]{0.10,0.09,0.49}{{#1}}}
    \newcommand{\ControlFlowTok}[1]{\textcolor[rgb]{0.00,0.44,0.13}{\textbf{{#1}}}}
    \newcommand{\OperatorTok}[1]{\textcolor[rgb]{0.40,0.40,0.40}{{#1}}}
    \newcommand{\BuiltInTok}[1]{{#1}}
    \newcommand{\ExtensionTok}[1]{{#1}}
    \newcommand{\PreprocessorTok}[1]{\textcolor[rgb]{0.74,0.48,0.00}{{#1}}}
    \newcommand{\AttributeTok}[1]{\textcolor[rgb]{0.49,0.56,0.16}{{#1}}}
    \newcommand{\InformationTok}[1]{\textcolor[rgb]{0.38,0.63,0.69}{\textbf{\textit{{#1}}}}}
    \newcommand{\WarningTok}[1]{\textcolor[rgb]{0.38,0.63,0.69}{\textbf{\textit{{#1}}}}}
    
    
    % Define a nice break command that doesn't care if a line doesn't already
    % exist.
    \def\br{\hspace*{\fill} \\* }
    % Math Jax compatibility definitions
    \def\gt{>}
    \def\lt{<}
    \let\Oldtex\TeX
    \let\Oldlatex\LaTeX
    \renewcommand{\TeX}{\textrm{\Oldtex}}
    \renewcommand{\LaTeX}{\textrm{\Oldlatex}}
    % Document parameters
    % Document title
    \title{C1\_W1\_lecture\_nb\_01\_preprocessing}
    
    
    
    
    
% Pygments definitions
\makeatletter
\def\PY@reset{\let\PY@it=\relax \let\PY@bf=\relax%
    \let\PY@ul=\relax \let\PY@tc=\relax%
    \let\PY@bc=\relax \let\PY@ff=\relax}
\def\PY@tok#1{\csname PY@tok@#1\endcsname}
\def\PY@toks#1+{\ifx\relax#1\empty\else%
    \PY@tok{#1}\expandafter\PY@toks\fi}
\def\PY@do#1{\PY@bc{\PY@tc{\PY@ul{%
    \PY@it{\PY@bf{\PY@ff{#1}}}}}}}
\def\PY#1#2{\PY@reset\PY@toks#1+\relax+\PY@do{#2}}

\expandafter\def\csname PY@tok@w\endcsname{\def\PY@tc##1{\textcolor[rgb]{0.73,0.73,0.73}{##1}}}
\expandafter\def\csname PY@tok@c\endcsname{\let\PY@it=\textit\def\PY@tc##1{\textcolor[rgb]{0.25,0.50,0.50}{##1}}}
\expandafter\def\csname PY@tok@cp\endcsname{\def\PY@tc##1{\textcolor[rgb]{0.74,0.48,0.00}{##1}}}
\expandafter\def\csname PY@tok@k\endcsname{\let\PY@bf=\textbf\def\PY@tc##1{\textcolor[rgb]{0.00,0.50,0.00}{##1}}}
\expandafter\def\csname PY@tok@kp\endcsname{\def\PY@tc##1{\textcolor[rgb]{0.00,0.50,0.00}{##1}}}
\expandafter\def\csname PY@tok@kt\endcsname{\def\PY@tc##1{\textcolor[rgb]{0.69,0.00,0.25}{##1}}}
\expandafter\def\csname PY@tok@o\endcsname{\def\PY@tc##1{\textcolor[rgb]{0.40,0.40,0.40}{##1}}}
\expandafter\def\csname PY@tok@ow\endcsname{\let\PY@bf=\textbf\def\PY@tc##1{\textcolor[rgb]{0.67,0.13,1.00}{##1}}}
\expandafter\def\csname PY@tok@nb\endcsname{\def\PY@tc##1{\textcolor[rgb]{0.00,0.50,0.00}{##1}}}
\expandafter\def\csname PY@tok@nf\endcsname{\def\PY@tc##1{\textcolor[rgb]{0.00,0.00,1.00}{##1}}}
\expandafter\def\csname PY@tok@nc\endcsname{\let\PY@bf=\textbf\def\PY@tc##1{\textcolor[rgb]{0.00,0.00,1.00}{##1}}}
\expandafter\def\csname PY@tok@nn\endcsname{\let\PY@bf=\textbf\def\PY@tc##1{\textcolor[rgb]{0.00,0.00,1.00}{##1}}}
\expandafter\def\csname PY@tok@ne\endcsname{\let\PY@bf=\textbf\def\PY@tc##1{\textcolor[rgb]{0.82,0.25,0.23}{##1}}}
\expandafter\def\csname PY@tok@nv\endcsname{\def\PY@tc##1{\textcolor[rgb]{0.10,0.09,0.49}{##1}}}
\expandafter\def\csname PY@tok@no\endcsname{\def\PY@tc##1{\textcolor[rgb]{0.53,0.00,0.00}{##1}}}
\expandafter\def\csname PY@tok@nl\endcsname{\def\PY@tc##1{\textcolor[rgb]{0.63,0.63,0.00}{##1}}}
\expandafter\def\csname PY@tok@ni\endcsname{\let\PY@bf=\textbf\def\PY@tc##1{\textcolor[rgb]{0.60,0.60,0.60}{##1}}}
\expandafter\def\csname PY@tok@na\endcsname{\def\PY@tc##1{\textcolor[rgb]{0.49,0.56,0.16}{##1}}}
\expandafter\def\csname PY@tok@nt\endcsname{\let\PY@bf=\textbf\def\PY@tc##1{\textcolor[rgb]{0.00,0.50,0.00}{##1}}}
\expandafter\def\csname PY@tok@nd\endcsname{\def\PY@tc##1{\textcolor[rgb]{0.67,0.13,1.00}{##1}}}
\expandafter\def\csname PY@tok@s\endcsname{\def\PY@tc##1{\textcolor[rgb]{0.73,0.13,0.13}{##1}}}
\expandafter\def\csname PY@tok@sd\endcsname{\let\PY@it=\textit\def\PY@tc##1{\textcolor[rgb]{0.73,0.13,0.13}{##1}}}
\expandafter\def\csname PY@tok@si\endcsname{\let\PY@bf=\textbf\def\PY@tc##1{\textcolor[rgb]{0.73,0.40,0.53}{##1}}}
\expandafter\def\csname PY@tok@se\endcsname{\let\PY@bf=\textbf\def\PY@tc##1{\textcolor[rgb]{0.73,0.40,0.13}{##1}}}
\expandafter\def\csname PY@tok@sr\endcsname{\def\PY@tc##1{\textcolor[rgb]{0.73,0.40,0.53}{##1}}}
\expandafter\def\csname PY@tok@ss\endcsname{\def\PY@tc##1{\textcolor[rgb]{0.10,0.09,0.49}{##1}}}
\expandafter\def\csname PY@tok@sx\endcsname{\def\PY@tc##1{\textcolor[rgb]{0.00,0.50,0.00}{##1}}}
\expandafter\def\csname PY@tok@m\endcsname{\def\PY@tc##1{\textcolor[rgb]{0.40,0.40,0.40}{##1}}}
\expandafter\def\csname PY@tok@gh\endcsname{\let\PY@bf=\textbf\def\PY@tc##1{\textcolor[rgb]{0.00,0.00,0.50}{##1}}}
\expandafter\def\csname PY@tok@gu\endcsname{\let\PY@bf=\textbf\def\PY@tc##1{\textcolor[rgb]{0.50,0.00,0.50}{##1}}}
\expandafter\def\csname PY@tok@gd\endcsname{\def\PY@tc##1{\textcolor[rgb]{0.63,0.00,0.00}{##1}}}
\expandafter\def\csname PY@tok@gi\endcsname{\def\PY@tc##1{\textcolor[rgb]{0.00,0.63,0.00}{##1}}}
\expandafter\def\csname PY@tok@gr\endcsname{\def\PY@tc##1{\textcolor[rgb]{1.00,0.00,0.00}{##1}}}
\expandafter\def\csname PY@tok@ge\endcsname{\let\PY@it=\textit}
\expandafter\def\csname PY@tok@gs\endcsname{\let\PY@bf=\textbf}
\expandafter\def\csname PY@tok@gp\endcsname{\let\PY@bf=\textbf\def\PY@tc##1{\textcolor[rgb]{0.00,0.00,0.50}{##1}}}
\expandafter\def\csname PY@tok@go\endcsname{\def\PY@tc##1{\textcolor[rgb]{0.53,0.53,0.53}{##1}}}
\expandafter\def\csname PY@tok@gt\endcsname{\def\PY@tc##1{\textcolor[rgb]{0.00,0.27,0.87}{##1}}}
\expandafter\def\csname PY@tok@err\endcsname{\def\PY@bc##1{\setlength{\fboxsep}{0pt}\fcolorbox[rgb]{1.00,0.00,0.00}{1,1,1}{\strut ##1}}}
\expandafter\def\csname PY@tok@kc\endcsname{\let\PY@bf=\textbf\def\PY@tc##1{\textcolor[rgb]{0.00,0.50,0.00}{##1}}}
\expandafter\def\csname PY@tok@kd\endcsname{\let\PY@bf=\textbf\def\PY@tc##1{\textcolor[rgb]{0.00,0.50,0.00}{##1}}}
\expandafter\def\csname PY@tok@kn\endcsname{\let\PY@bf=\textbf\def\PY@tc##1{\textcolor[rgb]{0.00,0.50,0.00}{##1}}}
\expandafter\def\csname PY@tok@kr\endcsname{\let\PY@bf=\textbf\def\PY@tc##1{\textcolor[rgb]{0.00,0.50,0.00}{##1}}}
\expandafter\def\csname PY@tok@bp\endcsname{\def\PY@tc##1{\textcolor[rgb]{0.00,0.50,0.00}{##1}}}
\expandafter\def\csname PY@tok@fm\endcsname{\def\PY@tc##1{\textcolor[rgb]{0.00,0.00,1.00}{##1}}}
\expandafter\def\csname PY@tok@vc\endcsname{\def\PY@tc##1{\textcolor[rgb]{0.10,0.09,0.49}{##1}}}
\expandafter\def\csname PY@tok@vg\endcsname{\def\PY@tc##1{\textcolor[rgb]{0.10,0.09,0.49}{##1}}}
\expandafter\def\csname PY@tok@vi\endcsname{\def\PY@tc##1{\textcolor[rgb]{0.10,0.09,0.49}{##1}}}
\expandafter\def\csname PY@tok@vm\endcsname{\def\PY@tc##1{\textcolor[rgb]{0.10,0.09,0.49}{##1}}}
\expandafter\def\csname PY@tok@sa\endcsname{\def\PY@tc##1{\textcolor[rgb]{0.73,0.13,0.13}{##1}}}
\expandafter\def\csname PY@tok@sb\endcsname{\def\PY@tc##1{\textcolor[rgb]{0.73,0.13,0.13}{##1}}}
\expandafter\def\csname PY@tok@sc\endcsname{\def\PY@tc##1{\textcolor[rgb]{0.73,0.13,0.13}{##1}}}
\expandafter\def\csname PY@tok@dl\endcsname{\def\PY@tc##1{\textcolor[rgb]{0.73,0.13,0.13}{##1}}}
\expandafter\def\csname PY@tok@s2\endcsname{\def\PY@tc##1{\textcolor[rgb]{0.73,0.13,0.13}{##1}}}
\expandafter\def\csname PY@tok@sh\endcsname{\def\PY@tc##1{\textcolor[rgb]{0.73,0.13,0.13}{##1}}}
\expandafter\def\csname PY@tok@s1\endcsname{\def\PY@tc##1{\textcolor[rgb]{0.73,0.13,0.13}{##1}}}
\expandafter\def\csname PY@tok@mb\endcsname{\def\PY@tc##1{\textcolor[rgb]{0.40,0.40,0.40}{##1}}}
\expandafter\def\csname PY@tok@mf\endcsname{\def\PY@tc##1{\textcolor[rgb]{0.40,0.40,0.40}{##1}}}
\expandafter\def\csname PY@tok@mh\endcsname{\def\PY@tc##1{\textcolor[rgb]{0.40,0.40,0.40}{##1}}}
\expandafter\def\csname PY@tok@mi\endcsname{\def\PY@tc##1{\textcolor[rgb]{0.40,0.40,0.40}{##1}}}
\expandafter\def\csname PY@tok@il\endcsname{\def\PY@tc##1{\textcolor[rgb]{0.40,0.40,0.40}{##1}}}
\expandafter\def\csname PY@tok@mo\endcsname{\def\PY@tc##1{\textcolor[rgb]{0.40,0.40,0.40}{##1}}}
\expandafter\def\csname PY@tok@ch\endcsname{\let\PY@it=\textit\def\PY@tc##1{\textcolor[rgb]{0.25,0.50,0.50}{##1}}}
\expandafter\def\csname PY@tok@cm\endcsname{\let\PY@it=\textit\def\PY@tc##1{\textcolor[rgb]{0.25,0.50,0.50}{##1}}}
\expandafter\def\csname PY@tok@cpf\endcsname{\let\PY@it=\textit\def\PY@tc##1{\textcolor[rgb]{0.25,0.50,0.50}{##1}}}
\expandafter\def\csname PY@tok@c1\endcsname{\let\PY@it=\textit\def\PY@tc##1{\textcolor[rgb]{0.25,0.50,0.50}{##1}}}
\expandafter\def\csname PY@tok@cs\endcsname{\let\PY@it=\textit\def\PY@tc##1{\textcolor[rgb]{0.25,0.50,0.50}{##1}}}

\def\PYZbs{\char`\\}
\def\PYZus{\char`\_}
\def\PYZob{\char`\{}
\def\PYZcb{\char`\}}
\def\PYZca{\char`\^}
\def\PYZam{\char`\&}
\def\PYZlt{\char`\<}
\def\PYZgt{\char`\>}
\def\PYZsh{\char`\#}
\def\PYZpc{\char`\%}
\def\PYZdl{\char`\$}
\def\PYZhy{\char`\-}
\def\PYZsq{\char`\'}
\def\PYZdq{\char`\"}
\def\PYZti{\char`\~}
% for compatibility with earlier versions
\def\PYZat{@}
\def\PYZlb{[}
\def\PYZrb{]}
\makeatother


    % For linebreaks inside Verbatim environment from package fancyvrb. 
    \makeatletter
        \newbox\Wrappedcontinuationbox 
        \newbox\Wrappedvisiblespacebox 
        \newcommand*\Wrappedvisiblespace {\textcolor{red}{\textvisiblespace}} 
        \newcommand*\Wrappedcontinuationsymbol {\textcolor{red}{\llap{\tiny$\m@th\hookrightarrow$}}} 
        \newcommand*\Wrappedcontinuationindent {3ex } 
        \newcommand*\Wrappedafterbreak {\kern\Wrappedcontinuationindent\copy\Wrappedcontinuationbox} 
        % Take advantage of the already applied Pygments mark-up to insert 
        % potential linebreaks for TeX processing. 
        %        {, <, #, %, $, ' and ": go to next line. 
        %        _, }, ^, &, >, - and ~: stay at end of broken line. 
        % Use of \textquotesingle for straight quote. 
        \newcommand*\Wrappedbreaksatspecials {% 
            \def\PYGZus{\discretionary{\char`\_}{\Wrappedafterbreak}{\char`\_}}% 
            \def\PYGZob{\discretionary{}{\Wrappedafterbreak\char`\{}{\char`\{}}% 
            \def\PYGZcb{\discretionary{\char`\}}{\Wrappedafterbreak}{\char`\}}}% 
            \def\PYGZca{\discretionary{\char`\^}{\Wrappedafterbreak}{\char`\^}}% 
            \def\PYGZam{\discretionary{\char`\&}{\Wrappedafterbreak}{\char`\&}}% 
            \def\PYGZlt{\discretionary{}{\Wrappedafterbreak\char`\<}{\char`\<}}% 
            \def\PYGZgt{\discretionary{\char`\>}{\Wrappedafterbreak}{\char`\>}}% 
            \def\PYGZsh{\discretionary{}{\Wrappedafterbreak\char`\#}{\char`\#}}% 
            \def\PYGZpc{\discretionary{}{\Wrappedafterbreak\char`\%}{\char`\%}}% 
            \def\PYGZdl{\discretionary{}{\Wrappedafterbreak\char`\$}{\char`\$}}% 
            \def\PYGZhy{\discretionary{\char`\-}{\Wrappedafterbreak}{\char`\-}}% 
            \def\PYGZsq{\discretionary{}{\Wrappedafterbreak\textquotesingle}{\textquotesingle}}% 
            \def\PYGZdq{\discretionary{}{\Wrappedafterbreak\char`\"}{\char`\"}}% 
            \def\PYGZti{\discretionary{\char`\~}{\Wrappedafterbreak}{\char`\~}}% 
        } 
        % Some characters . , ; ? ! / are not pygmentized. 
        % This macro makes them "active" and they will insert potential linebreaks 
        \newcommand*\Wrappedbreaksatpunct {% 
            \lccode`\~`\.\lowercase{\def~}{\discretionary{\hbox{\char`\.}}{\Wrappedafterbreak}{\hbox{\char`\.}}}% 
            \lccode`\~`\,\lowercase{\def~}{\discretionary{\hbox{\char`\,}}{\Wrappedafterbreak}{\hbox{\char`\,}}}% 
            \lccode`\~`\;\lowercase{\def~}{\discretionary{\hbox{\char`\;}}{\Wrappedafterbreak}{\hbox{\char`\;}}}% 
            \lccode`\~`\:\lowercase{\def~}{\discretionary{\hbox{\char`\:}}{\Wrappedafterbreak}{\hbox{\char`\:}}}% 
            \lccode`\~`\?\lowercase{\def~}{\discretionary{\hbox{\char`\?}}{\Wrappedafterbreak}{\hbox{\char`\?}}}% 
            \lccode`\~`\!\lowercase{\def~}{\discretionary{\hbox{\char`\!}}{\Wrappedafterbreak}{\hbox{\char`\!}}}% 
            \lccode`\~`\/\lowercase{\def~}{\discretionary{\hbox{\char`\/}}{\Wrappedafterbreak}{\hbox{\char`\/}}}% 
            \catcode`\.\active
            \catcode`\,\active 
            \catcode`\;\active
            \catcode`\:\active
            \catcode`\?\active
            \catcode`\!\active
            \catcode`\/\active 
            \lccode`\~`\~ 	
        }
    \makeatother

    \let\OriginalVerbatim=\Verbatim
    \makeatletter
    \renewcommand{\Verbatim}[1][1]{%
        %\parskip\z@skip
        \sbox\Wrappedcontinuationbox {\Wrappedcontinuationsymbol}%
        \sbox\Wrappedvisiblespacebox {\FV@SetupFont\Wrappedvisiblespace}%
        \def\FancyVerbFormatLine ##1{\hsize\linewidth
            \vtop{\raggedright\hyphenpenalty\z@\exhyphenpenalty\z@
                \doublehyphendemerits\z@\finalhyphendemerits\z@
                \strut ##1\strut}%
        }%
        % If the linebreak is at a space, the latter will be displayed as visible
        % space at end of first line, and a continuation symbol starts next line.
        % Stretch/shrink are however usually zero for typewriter font.
        \def\FV@Space {%
            \nobreak\hskip\z@ plus\fontdimen3\font minus\fontdimen4\font
            \discretionary{\copy\Wrappedvisiblespacebox}{\Wrappedafterbreak}
            {\kern\fontdimen2\font}%
        }%
        
        % Allow breaks at special characters using \PYG... macros.
        \Wrappedbreaksatspecials
        % Breaks at punctuation characters . , ; ? ! and / need catcode=\active 	
        \OriginalVerbatim[#1,codes*=\Wrappedbreaksatpunct]%
    }
    \makeatother

    % Exact colors from NB
    \definecolor{incolor}{HTML}{303F9F}
    \definecolor{outcolor}{HTML}{D84315}
    \definecolor{cellborder}{HTML}{CFCFCF}
    \definecolor{cellbackground}{HTML}{F7F7F7}
    
    % prompt
    \makeatletter
    \newcommand{\boxspacing}{\kern\kvtcb@left@rule\kern\kvtcb@boxsep}
    \makeatother
    \newcommand{\prompt}[4]{
        \ttfamily\llap{{\color{#2}[#3]:\hspace{3pt}#4}}\vspace{-\baselineskip}
    }
    

    
    % Prevent overflowing lines due to hard-to-break entities
    \sloppy 
    % Setup hyperref package
    \hypersetup{
      breaklinks=true,  % so long urls are correctly broken across lines
      colorlinks=true,
      urlcolor=urlcolor,
      linkcolor=linkcolor,
      citecolor=citecolor,
      }
    % Slightly bigger margins than the latex defaults
    
    \geometry{verbose,tmargin=1in,bmargin=1in,lmargin=1in,rmargin=1in}
    
    

\begin{document}
    
    \maketitle
    
    

    
    \hypertarget{preprocessing}{%
\section{Preprocessing}\label{preprocessing}}

In this lab, we will be exploring how to preprocess tweets for sentiment
analysis. We will provide a function for preprocessing tweets during
this week's assignment, but it is still good to know what is going on
under the hood. By the end of this lecture, you will see how to use the
\href{http://www.nltk.org}{NLTK} package to perform a preprocessing
pipeline for Twitter datasets.

    \hypertarget{setup}{%
\subsection{Setup}\label{setup}}

You will be doing sentiment analysis on tweets in the first two weeks of
this course. To help with that, we will be using the
\href{http://www.nltk.org/howto/twitter.html}{Natural Language Toolkit
(NLTK)} package, an open-source Python library for natural language
processing. It has modules for collecting, handling, and processing
Twitter data, and you will be acquainted with them as we move along the
course.

For this exercise, we will use a Twitter dataset that comes with NLTK.
This dataset has been manually annotated and serves to establish
baselines for models quickly. Let us import them now as well as a few
other libraries we will be using.

    \begin{tcolorbox}[breakable, size=fbox, boxrule=1pt, pad at break*=1mm,colback=cellbackground, colframe=cellborder]
\prompt{In}{incolor}{1}{\boxspacing}
\begin{Verbatim}[commandchars=\\\{\}]
\PY{k+kn}{import} \PY{n+nn}{nltk}                                \PY{c+c1}{\PYZsh{} Python library for NLP}
\PY{k+kn}{from} \PY{n+nn}{nltk}\PY{n+nn}{.}\PY{n+nn}{corpus} \PY{k+kn}{import} \PY{n}{twitter\PYZus{}samples}    \PY{c+c1}{\PYZsh{} sample Twitter dataset from NLTK}
\PY{k+kn}{import} \PY{n+nn}{matplotlib}\PY{n+nn}{.}\PY{n+nn}{pyplot} \PY{k}{as} \PY{n+nn}{plt}            \PY{c+c1}{\PYZsh{} library for visualization}
\PY{k+kn}{import} \PY{n+nn}{random}                              \PY{c+c1}{\PYZsh{} pseudo\PYZhy{}random number generator}
\end{Verbatim}
\end{tcolorbox}

    \hypertarget{about-the-twitter-dataset}{%
\subsection{About the Twitter dataset}\label{about-the-twitter-dataset}}

The sample dataset from NLTK is separated into positive and negative
tweets. It contains 5000 positive tweets and 5000 negative tweets
exactly. The exact match between these classes is not a coincidence. The
intention is to have a balanced dataset. That does not reflect the real
distributions of positive and negative classes in live Twitter streams.
It is just because balanced datasets simplify the design of most
computational methods that are required for sentiment analysis. However,
it is better to be aware that this balance of classes is artificial.

You can download the dataset in your workspace (or in your local
computer) by doing:

    \begin{tcolorbox}[breakable, size=fbox, boxrule=1pt, pad at break*=1mm,colback=cellbackground, colframe=cellborder]
\prompt{In}{incolor}{2}{\boxspacing}
\begin{Verbatim}[commandchars=\\\{\}]
\PY{c+c1}{\PYZsh{} downloads sample twitter dataset.}
\PY{n}{nltk}\PY{o}{.}\PY{n}{download}\PY{p}{(}\PY{l+s+s1}{\PYZsq{}}\PY{l+s+s1}{twitter\PYZus{}samples}\PY{l+s+s1}{\PYZsq{}}\PY{p}{)}
\end{Verbatim}
\end{tcolorbox}

    \begin{Verbatim}[commandchars=\\\{\}]
[nltk\_data] Downloading package twitter\_samples to
[nltk\_data]     /home/jovyan/nltk\_data{\ldots}
[nltk\_data]   Unzipping corpora/twitter\_samples.zip.
    \end{Verbatim}

            \begin{tcolorbox}[breakable, size=fbox, boxrule=.5pt, pad at break*=1mm, opacityfill=0]
\prompt{Out}{outcolor}{2}{\boxspacing}
\begin{Verbatim}[commandchars=\\\{\}]
True
\end{Verbatim}
\end{tcolorbox}
        
    We can load the text fields of the positive and negative tweets by using
the module's \texttt{strings()} method like this:

    \begin{tcolorbox}[breakable, size=fbox, boxrule=1pt, pad at break*=1mm,colback=cellbackground, colframe=cellborder]
\prompt{In}{incolor}{3}{\boxspacing}
\begin{Verbatim}[commandchars=\\\{\}]
\PY{c+c1}{\PYZsh{} select the set of positive and negative tweets}
\PY{n}{all\PYZus{}positive\PYZus{}tweets} \PY{o}{=} \PY{n}{twitter\PYZus{}samples}\PY{o}{.}\PY{n}{strings}\PY{p}{(}\PY{l+s+s1}{\PYZsq{}}\PY{l+s+s1}{positive\PYZus{}tweets.json}\PY{l+s+s1}{\PYZsq{}}\PY{p}{)}
\PY{n}{all\PYZus{}negative\PYZus{}tweets} \PY{o}{=} \PY{n}{twitter\PYZus{}samples}\PY{o}{.}\PY{n}{strings}\PY{p}{(}\PY{l+s+s1}{\PYZsq{}}\PY{l+s+s1}{negative\PYZus{}tweets.json}\PY{l+s+s1}{\PYZsq{}}\PY{p}{)}
\end{Verbatim}
\end{tcolorbox}

    Next, we'll print a report with the number of positive and negative
tweets. It is also essential to know the data structure of the datasets

    \begin{tcolorbox}[breakable, size=fbox, boxrule=1pt, pad at break*=1mm,colback=cellbackground, colframe=cellborder]
\prompt{In}{incolor}{4}{\boxspacing}
\begin{Verbatim}[commandchars=\\\{\}]
\PY{n+nb}{print}\PY{p}{(}\PY{l+s+s1}{\PYZsq{}}\PY{l+s+s1}{Number of positive tweets: }\PY{l+s+s1}{\PYZsq{}}\PY{p}{,} \PY{n+nb}{len}\PY{p}{(}\PY{n}{all\PYZus{}positive\PYZus{}tweets}\PY{p}{)}\PY{p}{)}
\PY{n+nb}{print}\PY{p}{(}\PY{l+s+s1}{\PYZsq{}}\PY{l+s+s1}{Number of negative tweets: }\PY{l+s+s1}{\PYZsq{}}\PY{p}{,} \PY{n+nb}{len}\PY{p}{(}\PY{n}{all\PYZus{}negative\PYZus{}tweets}\PY{p}{)}\PY{p}{)}

\PY{n+nb}{print}\PY{p}{(}\PY{l+s+s1}{\PYZsq{}}\PY{l+s+se}{\PYZbs{}n}\PY{l+s+s1}{The type of all\PYZus{}positive\PYZus{}tweets is: }\PY{l+s+s1}{\PYZsq{}}\PY{p}{,} \PY{n+nb}{type}\PY{p}{(}\PY{n}{all\PYZus{}positive\PYZus{}tweets}\PY{p}{)}\PY{p}{)}
\PY{n+nb}{print}\PY{p}{(}\PY{l+s+s1}{\PYZsq{}}\PY{l+s+s1}{The type of a tweet entry is: }\PY{l+s+s1}{\PYZsq{}}\PY{p}{,} \PY{n+nb}{type}\PY{p}{(}\PY{n}{all\PYZus{}negative\PYZus{}tweets}\PY{p}{[}\PY{l+m+mi}{0}\PY{p}{]}\PY{p}{)}\PY{p}{)}
\end{Verbatim}
\end{tcolorbox}

    \begin{Verbatim}[commandchars=\\\{\}]
Number of positive tweets:  5000
Number of negative tweets:  5000

The type of all\_positive\_tweets is:  <class 'list'>
The type of a tweet entry is:  <class 'str'>
    \end{Verbatim}

    We can see that the data is stored in a list and as you might expect,
individual tweets are stored as strings.

You can make a more visually appealing report by using Matplotlib's
\href{https://matplotlib.org/tutorials/introductory/pyplot.html}{pyplot}
library. Let us see how to create a
\href{https://matplotlib.org/3.2.1/gallery/pie_and_polar_charts/pie_features.html\#sphx-glr-gallery-pie-and-polar-charts-pie-features-py}{pie
chart} to show the same information as above. This simple snippet will
serve you in future visualizations of this kind of data.

    \begin{tcolorbox}[breakable, size=fbox, boxrule=1pt, pad at break*=1mm,colback=cellbackground, colframe=cellborder]
\prompt{In}{incolor}{5}{\boxspacing}
\begin{Verbatim}[commandchars=\\\{\}]
\PY{c+c1}{\PYZsh{} Declare a figure with a custom size}
\PY{n}{fig} \PY{o}{=} \PY{n}{plt}\PY{o}{.}\PY{n}{figure}\PY{p}{(}\PY{n}{figsize}\PY{o}{=}\PY{p}{(}\PY{l+m+mi}{5}\PY{p}{,} \PY{l+m+mi}{5}\PY{p}{)}\PY{p}{)}

\PY{c+c1}{\PYZsh{} labels for the two classes}
\PY{n}{labels} \PY{o}{=} \PY{l+s+s1}{\PYZsq{}}\PY{l+s+s1}{Positives}\PY{l+s+s1}{\PYZsq{}}\PY{p}{,} \PY{l+s+s1}{\PYZsq{}}\PY{l+s+s1}{Negative}\PY{l+s+s1}{\PYZsq{}}

\PY{c+c1}{\PYZsh{} Sizes for each slide}
\PY{n}{sizes} \PY{o}{=} \PY{p}{[}\PY{n+nb}{len}\PY{p}{(}\PY{n}{all\PYZus{}positive\PYZus{}tweets}\PY{p}{)}\PY{p}{,} \PY{n+nb}{len}\PY{p}{(}\PY{n}{all\PYZus{}negative\PYZus{}tweets}\PY{p}{)}\PY{p}{]} 

\PY{c+c1}{\PYZsh{} Declare pie chart, where the slices will be ordered and plotted counter\PYZhy{}clockwise:}
\PY{n}{plt}\PY{o}{.}\PY{n}{pie}\PY{p}{(}\PY{n}{sizes}\PY{p}{,} \PY{n}{labels}\PY{o}{=}\PY{n}{labels}\PY{p}{,} \PY{n}{autopct}\PY{o}{=}\PY{l+s+s1}{\PYZsq{}}\PY{l+s+si}{\PYZpc{}1.1f}\PY{l+s+si}{\PYZpc{}\PYZpc{}}\PY{l+s+s1}{\PYZsq{}}\PY{p}{,}
        \PY{n}{shadow}\PY{o}{=}\PY{k+kc}{True}\PY{p}{,} \PY{n}{startangle}\PY{o}{=}\PY{l+m+mi}{90}\PY{p}{)}

\PY{c+c1}{\PYZsh{} Equal aspect ratio ensures that pie is drawn as a circle.}
\PY{n}{plt}\PY{o}{.}\PY{n}{axis}\PY{p}{(}\PY{l+s+s1}{\PYZsq{}}\PY{l+s+s1}{equal}\PY{l+s+s1}{\PYZsq{}}\PY{p}{)}  

\PY{c+c1}{\PYZsh{} Display the chart}
\PY{n}{plt}\PY{o}{.}\PY{n}{show}\PY{p}{(}\PY{p}{)}
\end{Verbatim}
\end{tcolorbox}

    \begin{center}
    \adjustimage{max size={0.9\linewidth}{0.9\paperheight}}{output_10_0.png}
    \end{center}
    { \hspace*{\fill} \\}
    
    \hypertarget{looking-at-raw-texts}{%
\subsection{Looking at raw texts}\label{looking-at-raw-texts}}

Before anything else, we can print a couple of tweets from the dataset
to see how they look. Understanding the data is responsible for 80\% of
the success or failure in data science projects. We can use this time to
observe aspects we'd like to consider when preprocessing our data.

Below, you will print one random positive and one random negative tweet.
We have added a color mark at the beginning of the string to further
distinguish the two. (Warning: This is taken from a public dataset of
real tweets and a very small portion has explicit content.)

    \begin{tcolorbox}[breakable, size=fbox, boxrule=1pt, pad at break*=1mm,colback=cellbackground, colframe=cellborder]
\prompt{In}{incolor}{6}{\boxspacing}
\begin{Verbatim}[commandchars=\\\{\}]
\PY{c+c1}{\PYZsh{} print positive in greeen}
\PY{n+nb}{print}\PY{p}{(}\PY{l+s+s1}{\PYZsq{}}\PY{l+s+se}{\PYZbs{}033}\PY{l+s+s1}{[92m}\PY{l+s+s1}{\PYZsq{}} \PY{o}{+} \PY{n}{all\PYZus{}positive\PYZus{}tweets}\PY{p}{[}\PY{n}{random}\PY{o}{.}\PY{n}{randint}\PY{p}{(}\PY{l+m+mi}{0}\PY{p}{,}\PY{l+m+mi}{5000}\PY{p}{)}\PY{p}{]}\PY{p}{)}

\PY{c+c1}{\PYZsh{} print negative in red}
\PY{n+nb}{print}\PY{p}{(}\PY{l+s+s1}{\PYZsq{}}\PY{l+s+se}{\PYZbs{}033}\PY{l+s+s1}{[91m}\PY{l+s+s1}{\PYZsq{}} \PY{o}{+} \PY{n}{all\PYZus{}negative\PYZus{}tweets}\PY{p}{[}\PY{n}{random}\PY{o}{.}\PY{n}{randint}\PY{p}{(}\PY{l+m+mi}{0}\PY{p}{,}\PY{l+m+mi}{5000}\PY{p}{)}\PY{p}{]}\PY{p}{)}
\end{Verbatim}
\end{tcolorbox}

    \begin{Verbatim}[commandchars=\\\{\}]
\textcolor{ansi-green-intense}{"Good morning, beautiful :)" That's all it takes.
}\textcolor{ansi-red-intense}{i didnt wanna end up here :(}
    \end{Verbatim}

    One observation you may have is the presence of
\href{https://en.wikipedia.org/wiki/Emoticon}{emoticons} and URLs in
many of the tweets. This info will come in handy in the next steps.

    \hypertarget{preprocess-raw-text-for-sentiment-analysis}{%
\subsection{Preprocess raw text for Sentiment
analysis}\label{preprocess-raw-text-for-sentiment-analysis}}

    Data preprocessing is one of the critical steps in any machine learning
project. It includes cleaning and formatting the data before feeding
into a machine learning algorithm. For NLP, the preprocessing steps are
comprised of the following tasks:

\begin{itemize}
\tightlist
\item
  Tokenizing the string
\item
  Lowercasing
\item
  Removing stop words and punctuation
\item
  Stemming
\end{itemize}

The videos explained each of these steps and why they are important.
Let's see how we can do these to a given tweet. We will choose just one
and see how this is transformed by each preprocessing step.

    \begin{tcolorbox}[breakable, size=fbox, boxrule=1pt, pad at break*=1mm,colback=cellbackground, colframe=cellborder]
\prompt{In}{incolor}{7}{\boxspacing}
\begin{Verbatim}[commandchars=\\\{\}]
\PY{c+c1}{\PYZsh{} Our selected sample. Complex enough to exemplify each step}
\PY{n}{tweet} \PY{o}{=} \PY{n}{all\PYZus{}positive\PYZus{}tweets}\PY{p}{[}\PY{l+m+mi}{2277}\PY{p}{]}
\PY{n+nb}{print}\PY{p}{(}\PY{n}{tweet}\PY{p}{)}
\end{Verbatim}
\end{tcolorbox}

    \begin{Verbatim}[commandchars=\\\{\}]
My beautiful sunflowers on a sunny Friday morning off :) \#sunflowers \#favourites
\#happy \#Friday off… https://t.co/3tfYom0N1i
    \end{Verbatim}

    Let's import a few more libraries for this purpose.

    \begin{tcolorbox}[breakable, size=fbox, boxrule=1pt, pad at break*=1mm,colback=cellbackground, colframe=cellborder]
\prompt{In}{incolor}{8}{\boxspacing}
\begin{Verbatim}[commandchars=\\\{\}]
\PY{c+c1}{\PYZsh{} download the stopwords from NLTK}
\PY{n}{nltk}\PY{o}{.}\PY{n}{download}\PY{p}{(}\PY{l+s+s1}{\PYZsq{}}\PY{l+s+s1}{stopwords}\PY{l+s+s1}{\PYZsq{}}\PY{p}{)}
\end{Verbatim}
\end{tcolorbox}

    \begin{Verbatim}[commandchars=\\\{\}]
[nltk\_data] Downloading package stopwords to /home/jovyan/nltk\_data{\ldots}
[nltk\_data]   Unzipping corpora/stopwords.zip.
    \end{Verbatim}

            \begin{tcolorbox}[breakable, size=fbox, boxrule=.5pt, pad at break*=1mm, opacityfill=0]
\prompt{Out}{outcolor}{8}{\boxspacing}
\begin{Verbatim}[commandchars=\\\{\}]
True
\end{Verbatim}
\end{tcolorbox}
        
    \begin{tcolorbox}[breakable, size=fbox, boxrule=1pt, pad at break*=1mm,colback=cellbackground, colframe=cellborder]
\prompt{In}{incolor}{10}{\boxspacing}
\begin{Verbatim}[commandchars=\\\{\}]
\PY{k+kn}{import} \PY{n+nn}{re}                                  \PY{c+c1}{\PYZsh{} library for regular expression operations}
\PY{k+kn}{import} \PY{n+nn}{string}                              \PY{c+c1}{\PYZsh{} for string operations}

\PY{k+kn}{from} \PY{n+nn}{nltk}\PY{n+nn}{.}\PY{n+nn}{corpus} \PY{k+kn}{import} \PY{n}{stopwords}          \PY{c+c1}{\PYZsh{} module for stop words that come with NLTK}
\PY{k+kn}{from} \PY{n+nn}{nltk}\PY{n+nn}{.}\PY{n+nn}{stem} \PY{k+kn}{import} \PY{n}{PorterStemmer}        \PY{c+c1}{\PYZsh{} module for stemming}
\PY{k+kn}{from} \PY{n+nn}{nltk}\PY{n+nn}{.}\PY{n+nn}{tokenize} \PY{k+kn}{import} \PY{n}{TweetTokenizer}   \PY{c+c1}{\PYZsh{} module for tokenizing strings}
\end{Verbatim}
\end{tcolorbox}

    \hypertarget{remove-hyperlinks-twitter-marks-and-styles}{%
\subsubsection{Remove hyperlinks, Twitter marks and
styles}\label{remove-hyperlinks-twitter-marks-and-styles}}

Since we have a Twitter dataset, we'd like to remove some substrings
commonly used on the platform like the hashtag, retweet marks, and
hyperlinks. We'll use the
\href{https://docs.python.org/3/library/re.html}{re} library to perform
regular expression operations on our tweet. We'll define our search
pattern and use the \texttt{sub()} method to remove matches by
substituting with an empty character
(i.e.~\texttt{\textquotesingle{}\textquotesingle{}})

    \begin{tcolorbox}[breakable, size=fbox, boxrule=1pt, pad at break*=1mm,colback=cellbackground, colframe=cellborder]
\prompt{In}{incolor}{11}{\boxspacing}
\begin{Verbatim}[commandchars=\\\{\}]
\PY{n+nb}{print}\PY{p}{(}\PY{l+s+s1}{\PYZsq{}}\PY{l+s+se}{\PYZbs{}033}\PY{l+s+s1}{[92m}\PY{l+s+s1}{\PYZsq{}} \PY{o}{+} \PY{n}{tweet}\PY{p}{)}
\PY{n+nb}{print}\PY{p}{(}\PY{l+s+s1}{\PYZsq{}}\PY{l+s+se}{\PYZbs{}033}\PY{l+s+s1}{[94m}\PY{l+s+s1}{\PYZsq{}}\PY{p}{)}

\PY{c+c1}{\PYZsh{} remove old style retweet text \PYZdq{}RT\PYZdq{}}
\PY{n}{tweet2} \PY{o}{=} \PY{n}{re}\PY{o}{.}\PY{n}{sub}\PY{p}{(}\PY{l+s+sa}{r}\PY{l+s+s1}{\PYZsq{}}\PY{l+s+s1}{\PYZca{}RT[}\PY{l+s+s1}{\PYZbs{}}\PY{l+s+s1}{s]+}\PY{l+s+s1}{\PYZsq{}}\PY{p}{,} \PY{l+s+s1}{\PYZsq{}}\PY{l+s+s1}{\PYZsq{}}\PY{p}{,} \PY{n}{tweet}\PY{p}{)}

\PY{c+c1}{\PYZsh{} remove hyperlinks}
\PY{n}{tweet2} \PY{o}{=} \PY{n}{re}\PY{o}{.}\PY{n}{sub}\PY{p}{(}\PY{l+s+sa}{r}\PY{l+s+s1}{\PYZsq{}}\PY{l+s+s1}{https?://[\PYZca{}}\PY{l+s+s1}{\PYZbs{}}\PY{l+s+s1}{s}\PY{l+s+s1}{\PYZbs{}}\PY{l+s+s1}{n}\PY{l+s+s1}{\PYZbs{}}\PY{l+s+s1}{r]+}\PY{l+s+s1}{\PYZsq{}}\PY{p}{,} \PY{l+s+s1}{\PYZsq{}}\PY{l+s+s1}{\PYZsq{}}\PY{p}{,} \PY{n}{tweet2}\PY{p}{)}

\PY{c+c1}{\PYZsh{} remove hashtags}
\PY{c+c1}{\PYZsh{} only removing the hash \PYZsh{} sign from the word}
\PY{n}{tweet2} \PY{o}{=} \PY{n}{re}\PY{o}{.}\PY{n}{sub}\PY{p}{(}\PY{l+s+sa}{r}\PY{l+s+s1}{\PYZsq{}}\PY{l+s+s1}{\PYZsh{}}\PY{l+s+s1}{\PYZsq{}}\PY{p}{,} \PY{l+s+s1}{\PYZsq{}}\PY{l+s+s1}{\PYZsq{}}\PY{p}{,} \PY{n}{tweet2}\PY{p}{)}

\PY{n+nb}{print}\PY{p}{(}\PY{n}{tweet2}\PY{p}{)}
\end{Verbatim}
\end{tcolorbox}

    \begin{Verbatim}[commandchars=\\\{\}]
\textcolor{ansi-green-intense}{My beautiful sunflowers on a sunny Friday morning off :) \#sunflowers
\#favourites \#happy \#Friday off… https://t.co/3tfYom0N1i
}\textcolor{ansi-blue-intense}{
My beautiful sunflowers on a sunny Friday morning off :) sunflowers favourites
happy Friday off…}
    \end{Verbatim}

    \hypertarget{tokenize-the-string}{%
\subsubsection{Tokenize the string}\label{tokenize-the-string}}

To tokenize means to split the strings into individual words without
blanks or tabs. In this same step, we will also convert each word in the
string to lower case. The
\href{https://www.nltk.org/api/nltk.tokenize.html\#module-nltk.tokenize.casual}{tokenize}
module from NLTK allows us to do these easily:

    \begin{tcolorbox}[breakable, size=fbox, boxrule=1pt, pad at break*=1mm,colback=cellbackground, colframe=cellborder]
\prompt{In}{incolor}{12}{\boxspacing}
\begin{Verbatim}[commandchars=\\\{\}]
\PY{n+nb}{print}\PY{p}{(}\PY{p}{)}
\PY{n+nb}{print}\PY{p}{(}\PY{l+s+s1}{\PYZsq{}}\PY{l+s+se}{\PYZbs{}033}\PY{l+s+s1}{[92m}\PY{l+s+s1}{\PYZsq{}} \PY{o}{+} \PY{n}{tweet2}\PY{p}{)}
\PY{n+nb}{print}\PY{p}{(}\PY{l+s+s1}{\PYZsq{}}\PY{l+s+se}{\PYZbs{}033}\PY{l+s+s1}{[94m}\PY{l+s+s1}{\PYZsq{}}\PY{p}{)}

\PY{c+c1}{\PYZsh{} instantiate tokenizer class}
\PY{n}{tokenizer} \PY{o}{=} \PY{n}{TweetTokenizer}\PY{p}{(}\PY{n}{preserve\PYZus{}case}\PY{o}{=}\PY{k+kc}{False}\PY{p}{,} \PY{n}{strip\PYZus{}handles}\PY{o}{=}\PY{k+kc}{True}\PY{p}{,}
                               \PY{n}{reduce\PYZus{}len}\PY{o}{=}\PY{k+kc}{True}\PY{p}{)}

\PY{c+c1}{\PYZsh{} tokenize tweets}
\PY{n}{tweet\PYZus{}tokens} \PY{o}{=} \PY{n}{tokenizer}\PY{o}{.}\PY{n}{tokenize}\PY{p}{(}\PY{n}{tweet2}\PY{p}{)}

\PY{n+nb}{print}\PY{p}{(}\PY{p}{)}
\PY{n+nb}{print}\PY{p}{(}\PY{l+s+s1}{\PYZsq{}}\PY{l+s+s1}{Tokenized string:}\PY{l+s+s1}{\PYZsq{}}\PY{p}{)}
\PY{n+nb}{print}\PY{p}{(}\PY{n}{tweet\PYZus{}tokens}\PY{p}{)}
\end{Verbatim}
\end{tcolorbox}

    \begin{Verbatim}[commandchars=\\\{\}]

\textcolor{ansi-green-intense}{My beautiful sunflowers on a sunny Friday morning off :) sunflowers
favourites happy Friday off…
}\textcolor{ansi-blue-intense}{

Tokenized string:
['my', 'beautiful', 'sunflowers', 'on', 'a', 'sunny', 'friday', 'morning',
'off', ':)', 'sunflowers', 'favourites', 'happy', 'friday', 'off', '…']}
    \end{Verbatim}

    \hypertarget{remove-stop-words-and-punctuations}{%
\subsubsection{Remove stop words and
punctuations}\label{remove-stop-words-and-punctuations}}

The next step is to remove stop words and punctuation. Stop words are
words that don't add significant meaning to the text. You'll see the
list provided by NLTK when you run the cells below.

    \begin{tcolorbox}[breakable, size=fbox, boxrule=1pt, pad at break*=1mm,colback=cellbackground, colframe=cellborder]
\prompt{In}{incolor}{13}{\boxspacing}
\begin{Verbatim}[commandchars=\\\{\}]
\PY{c+c1}{\PYZsh{}Import the english stop words list from NLTK}
\PY{n}{stopwords\PYZus{}english} \PY{o}{=} \PY{n}{stopwords}\PY{o}{.}\PY{n}{words}\PY{p}{(}\PY{l+s+s1}{\PYZsq{}}\PY{l+s+s1}{english}\PY{l+s+s1}{\PYZsq{}}\PY{p}{)} 

\PY{n+nb}{print}\PY{p}{(}\PY{l+s+s1}{\PYZsq{}}\PY{l+s+s1}{Stop words}\PY{l+s+se}{\PYZbs{}n}\PY{l+s+s1}{\PYZsq{}}\PY{p}{)}
\PY{n+nb}{print}\PY{p}{(}\PY{n}{stopwords\PYZus{}english}\PY{p}{)}

\PY{n+nb}{print}\PY{p}{(}\PY{l+s+s1}{\PYZsq{}}\PY{l+s+se}{\PYZbs{}n}\PY{l+s+s1}{Punctuation}\PY{l+s+se}{\PYZbs{}n}\PY{l+s+s1}{\PYZsq{}}\PY{p}{)}
\PY{n+nb}{print}\PY{p}{(}\PY{n}{string}\PY{o}{.}\PY{n}{punctuation}\PY{p}{)}
\end{Verbatim}
\end{tcolorbox}

    \begin{Verbatim}[commandchars=\\\{\}]
Stop words

['i', 'me', 'my', 'myself', 'we', 'our', 'ours', 'ourselves', 'you', "you're",
"you've", "you'll", "you'd", 'your', 'yours', 'yourself', 'yourselves', 'he',
'him', 'his', 'himself', 'she', "she's", 'her', 'hers', 'herself', 'it', "it's",
'its', 'itself', 'they', 'them', 'their', 'theirs', 'themselves', 'what',
'which', 'who', 'whom', 'this', 'that', "that'll", 'these', 'those', 'am', 'is',
'are', 'was', 'were', 'be', 'been', 'being', 'have', 'has', 'had', 'having',
'do', 'does', 'did', 'doing', 'a', 'an', 'the', 'and', 'but', 'if', 'or',
'because', 'as', 'until', 'while', 'of', 'at', 'by', 'for', 'with', 'about',
'against', 'between', 'into', 'through', 'during', 'before', 'after', 'above',
'below', 'to', 'from', 'up', 'down', 'in', 'out', 'on', 'off', 'over', 'under',
'again', 'further', 'then', 'once', 'here', 'there', 'when', 'where', 'why',
'how', 'all', 'any', 'both', 'each', 'few', 'more', 'most', 'other', 'some',
'such', 'no', 'nor', 'not', 'only', 'own', 'same', 'so', 'than', 'too', 'very',
's', 't', 'can', 'will', 'just', 'don', "don't", 'should', "should've", 'now',
'd', 'll', 'm', 'o', 're', 've', 'y', 'ain', 'aren', "aren't", 'couldn',
"couldn't", 'didn', "didn't", 'doesn', "doesn't", 'hadn', "hadn't", 'hasn',
"hasn't", 'haven', "haven't", 'isn', "isn't", 'ma', 'mightn', "mightn't",
'mustn', "mustn't", 'needn', "needn't", 'shan', "shan't", 'shouldn',
"shouldn't", 'wasn', "wasn't", 'weren', "weren't", 'won', "won't", 'wouldn',
"wouldn't"]

Punctuation

!"\#\$\%\&'()*+,-./:;<=>?@[\textbackslash{}]\^{}\_`\{|\}\textasciitilde{}
    \end{Verbatim}

    We can see that the stop words list above contains some words that could
be important in some contexts. These could be words like \emph{i, not,
between, because, won, against}. You might need to customize the stop
words list for some applications. For our exercise, we will use the
entire list.

For the punctuation, we saw earlier that certain groupings like `:)' and
`\ldots{}' should be retained when dealing with tweets because they are
used to express emotions. In other contexts, like medical analysis,
these should also be removed.

Time to clean up our tokenized tweet!

    \begin{tcolorbox}[breakable, size=fbox, boxrule=1pt, pad at break*=1mm,colback=cellbackground, colframe=cellborder]
\prompt{In}{incolor}{14}{\boxspacing}
\begin{Verbatim}[commandchars=\\\{\}]
\PY{n+nb}{print}\PY{p}{(}\PY{p}{)}
\PY{n+nb}{print}\PY{p}{(}\PY{l+s+s1}{\PYZsq{}}\PY{l+s+se}{\PYZbs{}033}\PY{l+s+s1}{[92m}\PY{l+s+s1}{\PYZsq{}}\PY{p}{)}
\PY{n+nb}{print}\PY{p}{(}\PY{n}{tweet\PYZus{}tokens}\PY{p}{)}
\PY{n+nb}{print}\PY{p}{(}\PY{l+s+s1}{\PYZsq{}}\PY{l+s+se}{\PYZbs{}033}\PY{l+s+s1}{[94m}\PY{l+s+s1}{\PYZsq{}}\PY{p}{)}

\PY{n}{tweets\PYZus{}clean} \PY{o}{=} \PY{p}{[}\PY{p}{]}

\PY{k}{for} \PY{n}{word} \PY{o+ow}{in} \PY{n}{tweet\PYZus{}tokens}\PY{p}{:} \PY{c+c1}{\PYZsh{} Go through every word in your tokens list}
    \PY{k}{if} \PY{p}{(}\PY{n}{word} \PY{o+ow}{not} \PY{o+ow}{in} \PY{n}{stopwords\PYZus{}english} \PY{o+ow}{and}  \PY{c+c1}{\PYZsh{} remove stopwords}
        \PY{n}{word} \PY{o+ow}{not} \PY{o+ow}{in} \PY{n}{string}\PY{o}{.}\PY{n}{punctuation}\PY{p}{)}\PY{p}{:}  \PY{c+c1}{\PYZsh{} remove punctuation}
        \PY{n}{tweets\PYZus{}clean}\PY{o}{.}\PY{n}{append}\PY{p}{(}\PY{n}{word}\PY{p}{)}

\PY{n+nb}{print}\PY{p}{(}\PY{l+s+s1}{\PYZsq{}}\PY{l+s+s1}{removed stop words and punctuation:}\PY{l+s+s1}{\PYZsq{}}\PY{p}{)}
\PY{n+nb}{print}\PY{p}{(}\PY{n}{tweets\PYZus{}clean}\PY{p}{)}
\end{Verbatim}
\end{tcolorbox}

    \begin{Verbatim}[commandchars=\\\{\}]

\textcolor{ansi-green-intense}{
['my', 'beautiful', 'sunflowers', 'on', 'a', 'sunny', 'friday', 'morning',
'off', ':)', 'sunflowers', 'favourites', 'happy', 'friday', 'off', '…']
}\textcolor{ansi-blue-intense}{
removed stop words and punctuation:
['beautiful', 'sunflowers', 'sunny', 'friday', 'morning', ':)', 'sunflowers',
'favourites', 'happy', 'friday', '…']}
    \end{Verbatim}

    Please note that the words \textbf{happy} and \textbf{sunny} in this
list are correctly spelled.

    \hypertarget{stemming}{%
\subsubsection{Stemming}\label{stemming}}

Stemming is the process of converting a word to its most general form,
or stem. This helps in reducing the size of our vocabulary.

Consider the words: * \textbf{learn} * \textbf{learn}ing *
\textbf{learn}ed * \textbf{learn}t

All these words are stemmed from its common root \textbf{learn}.
However, in some cases, the stemming process produces words that are not
correct spellings of the root word. For example, \textbf{happi} and
\textbf{sunni}. That's because it chooses the most common stem for
related words. For example, we can look at the set of words that
comprises the different forms of happy:

\begin{itemize}
\tightlist
\item
  \textbf{happ}y
\item
  \textbf{happi}ness
\item
  \textbf{happi}er
\end{itemize}

We can see that the prefix \textbf{happi} is more commonly used. We
cannot choose \textbf{happ} because it is the stem of unrelated words
like \textbf{happen}.

NLTK has different modules for stemming and we will be using the
\href{https://www.nltk.org/api/nltk.stem.html\#module-nltk.stem.porter}{PorterStemmer}
module which uses the
\href{https://tartarus.org/martin/PorterStemmer/}{Porter Stemming
Algorithm}. Let's see how we can use it in the cell below.

    \begin{tcolorbox}[breakable, size=fbox, boxrule=1pt, pad at break*=1mm,colback=cellbackground, colframe=cellborder]
\prompt{In}{incolor}{15}{\boxspacing}
\begin{Verbatim}[commandchars=\\\{\}]
\PY{n+nb}{print}\PY{p}{(}\PY{p}{)}
\PY{n+nb}{print}\PY{p}{(}\PY{l+s+s1}{\PYZsq{}}\PY{l+s+se}{\PYZbs{}033}\PY{l+s+s1}{[92m}\PY{l+s+s1}{\PYZsq{}}\PY{p}{)}
\PY{n+nb}{print}\PY{p}{(}\PY{n}{tweets\PYZus{}clean}\PY{p}{)}
\PY{n+nb}{print}\PY{p}{(}\PY{l+s+s1}{\PYZsq{}}\PY{l+s+se}{\PYZbs{}033}\PY{l+s+s1}{[94m}\PY{l+s+s1}{\PYZsq{}}\PY{p}{)}

\PY{c+c1}{\PYZsh{} Instantiate stemming class}
\PY{n}{stemmer} \PY{o}{=} \PY{n}{PorterStemmer}\PY{p}{(}\PY{p}{)} 

\PY{c+c1}{\PYZsh{} Create an empty list to store the stems}
\PY{n}{tweets\PYZus{}stem} \PY{o}{=} \PY{p}{[}\PY{p}{]} 

\PY{k}{for} \PY{n}{word} \PY{o+ow}{in} \PY{n}{tweets\PYZus{}clean}\PY{p}{:}
    \PY{n}{stem\PYZus{}word} \PY{o}{=} \PY{n}{stemmer}\PY{o}{.}\PY{n}{stem}\PY{p}{(}\PY{n}{word}\PY{p}{)}  \PY{c+c1}{\PYZsh{} stemming word}
    \PY{n}{tweets\PYZus{}stem}\PY{o}{.}\PY{n}{append}\PY{p}{(}\PY{n}{stem\PYZus{}word}\PY{p}{)}  \PY{c+c1}{\PYZsh{} append to the list}

\PY{n+nb}{print}\PY{p}{(}\PY{l+s+s1}{\PYZsq{}}\PY{l+s+s1}{stemmed words:}\PY{l+s+s1}{\PYZsq{}}\PY{p}{)}
\PY{n+nb}{print}\PY{p}{(}\PY{n}{tweets\PYZus{}stem}\PY{p}{)}
\end{Verbatim}
\end{tcolorbox}

    \begin{Verbatim}[commandchars=\\\{\}]

\textcolor{ansi-green-intense}{
['beautiful', 'sunflowers', 'sunny', 'friday', 'morning', ':)', 'sunflowers',
'favourites', 'happy', 'friday', '…']
}\textcolor{ansi-blue-intense}{
stemmed words:
['beauti', 'sunflow', 'sunni', 'friday', 'morn', ':)', 'sunflow', 'favourit',
'happi', 'friday', '…']}
    \end{Verbatim}

    That's it! Now we have a set of words we can feed into to the next stage
of our machine learning project.

    \hypertarget{process_tweet}{%
\subsection{process\_tweet()}\label{process_tweet}}

As shown above, preprocessing consists of multiple steps before you
arrive at the final list of words. We will not ask you to replicate
these however. In the week's assignment, you will use the function
\texttt{process\_tweet(tweet)} available in \emph{utils.py}. We
encourage you to open the file and you'll see that this function's
implementation is very similar to the steps above.

To obtain the same result as in the previous code cells, you will only
need to call the function \texttt{process\_tweet()}. Let's do that in
the next cell.

    \begin{tcolorbox}[breakable, size=fbox, boxrule=1pt, pad at break*=1mm,colback=cellbackground, colframe=cellborder]
\prompt{In}{incolor}{16}{\boxspacing}
\begin{Verbatim}[commandchars=\\\{\}]
\PY{k+kn}{from} \PY{n+nn}{utils} \PY{k+kn}{import} \PY{n}{process\PYZus{}tweet} \PY{c+c1}{\PYZsh{} Import the process\PYZus{}tweet function}

\PY{c+c1}{\PYZsh{} choose the same tweet}
\PY{n}{tweet} \PY{o}{=} \PY{n}{all\PYZus{}positive\PYZus{}tweets}\PY{p}{[}\PY{l+m+mi}{2277}\PY{p}{]}

\PY{n+nb}{print}\PY{p}{(}\PY{p}{)}
\PY{n+nb}{print}\PY{p}{(}\PY{l+s+s1}{\PYZsq{}}\PY{l+s+se}{\PYZbs{}033}\PY{l+s+s1}{[92m}\PY{l+s+s1}{\PYZsq{}}\PY{p}{)}
\PY{n+nb}{print}\PY{p}{(}\PY{n}{tweet}\PY{p}{)}
\PY{n+nb}{print}\PY{p}{(}\PY{l+s+s1}{\PYZsq{}}\PY{l+s+se}{\PYZbs{}033}\PY{l+s+s1}{[94m}\PY{l+s+s1}{\PYZsq{}}\PY{p}{)}

\PY{c+c1}{\PYZsh{} call the imported function}
\PY{n}{tweets\PYZus{}stem} \PY{o}{=} \PY{n}{process\PYZus{}tweet}\PY{p}{(}\PY{n}{tweet}\PY{p}{)}\PY{p}{;} \PY{c+c1}{\PYZsh{} Preprocess a given tweet}

\PY{n+nb}{print}\PY{p}{(}\PY{l+s+s1}{\PYZsq{}}\PY{l+s+s1}{preprocessed tweet:}\PY{l+s+s1}{\PYZsq{}}\PY{p}{)}
\PY{n+nb}{print}\PY{p}{(}\PY{n}{tweets\PYZus{}stem}\PY{p}{)} \PY{c+c1}{\PYZsh{} Print the result}
\end{Verbatim}
\end{tcolorbox}

    \begin{Verbatim}[commandchars=\\\{\}]

\textcolor{ansi-green-intense}{
My beautiful sunflowers on a sunny Friday morning off :) \#sunflowers \#favourites
\#happy \#Friday off… https://t.co/3tfYom0N1i
}\textcolor{ansi-blue-intense}{
preprocessed tweet:
['beauti', 'sunflow', 'sunni', 'friday', 'morn', ':)', 'sunflow', 'favourit',
'happi', 'friday', '…']}
    \end{Verbatim}

    That's it for this lab! You now know what is going on when you call the
preprocessing helper function in this week's assignment. Hopefully, this
exercise has also given you some insights on how to tweak this for other
types of text datasets.


    % Add a bibliography block to the postdoc
    
    
    
\end{document}
