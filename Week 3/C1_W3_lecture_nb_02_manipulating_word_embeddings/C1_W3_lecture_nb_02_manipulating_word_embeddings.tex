\documentclass[11pt]{article}

    \usepackage[breakable]{tcolorbox}
    \usepackage{parskip} % Stop auto-indenting (to mimic markdown behaviour)
    
    \usepackage{iftex}
    \ifPDFTeX
    	\usepackage[T1]{fontenc}
    	\usepackage{mathpazo}
    \else
    	\usepackage{fontspec}
    \fi

    % Basic figure setup, for now with no caption control since it's done
    % automatically by Pandoc (which extracts ![](path) syntax from Markdown).
    \usepackage{graphicx}
    % Maintain compatibility with old templates. Remove in nbconvert 6.0
    \let\Oldincludegraphics\includegraphics
    % Ensure that by default, figures have no caption (until we provide a
    % proper Figure object with a Caption API and a way to capture that
    % in the conversion process - todo).
    \usepackage{caption}
    \DeclareCaptionFormat{nocaption}{}
    \captionsetup{format=nocaption,aboveskip=0pt,belowskip=0pt}

    \usepackage[Export]{adjustbox} % Used to constrain images to a maximum size
    \adjustboxset{max size={0.9\linewidth}{0.9\paperheight}}
    \usepackage{float}
    \floatplacement{figure}{H} % forces figures to be placed at the correct location
    \usepackage{xcolor} % Allow colors to be defined
    \usepackage{enumerate} % Needed for markdown enumerations to work
    \usepackage{geometry} % Used to adjust the document margins
    \usepackage{amsmath} % Equations
    \usepackage{amssymb} % Equations
    \usepackage{textcomp} % defines textquotesingle
    % Hack from http://tex.stackexchange.com/a/47451/13684:
    \AtBeginDocument{%
        \def\PYZsq{\textquotesingle}% Upright quotes in Pygmentized code
    }
    \usepackage{upquote} % Upright quotes for verbatim code
    \usepackage{eurosym} % defines \euro
    \usepackage[mathletters]{ucs} % Extended unicode (utf-8) support
    \usepackage{fancyvrb} % verbatim replacement that allows latex
    \usepackage{grffile} % extends the file name processing of package graphics 
                         % to support a larger range
    \makeatletter % fix for grffile with XeLaTeX
    \def\Gread@@xetex#1{%
      \IfFileExists{"\Gin@base".bb}%
      {\Gread@eps{\Gin@base.bb}}%
      {\Gread@@xetex@aux#1}%
    }
    \makeatother

    % The hyperref package gives us a pdf with properly built
    % internal navigation ('pdf bookmarks' for the table of contents,
    % internal cross-reference links, web links for URLs, etc.)
    \usepackage{hyperref}
    % The default LaTeX title has an obnoxious amount of whitespace. By default,
    % titling removes some of it. It also provides customization options.
    \usepackage{titling}
    \usepackage{longtable} % longtable support required by pandoc >1.10
    \usepackage{booktabs}  % table support for pandoc > 1.12.2
    \usepackage[inline]{enumitem} % IRkernel/repr support (it uses the enumerate* environment)
    \usepackage[normalem]{ulem} % ulem is needed to support strikethroughs (\sout)
                                % normalem makes italics be italics, not underlines
    \usepackage{mathrsfs}
    

    
    % Colors for the hyperref package
    \definecolor{urlcolor}{rgb}{0,.145,.698}
    \definecolor{linkcolor}{rgb}{.71,0.21,0.01}
    \definecolor{citecolor}{rgb}{.12,.54,.11}

    % ANSI colors
    \definecolor{ansi-black}{HTML}{3E424D}
    \definecolor{ansi-black-intense}{HTML}{282C36}
    \definecolor{ansi-red}{HTML}{E75C58}
    \definecolor{ansi-red-intense}{HTML}{B22B31}
    \definecolor{ansi-green}{HTML}{00A250}
    \definecolor{ansi-green-intense}{HTML}{007427}
    \definecolor{ansi-yellow}{HTML}{DDB62B}
    \definecolor{ansi-yellow-intense}{HTML}{B27D12}
    \definecolor{ansi-blue}{HTML}{208FFB}
    \definecolor{ansi-blue-intense}{HTML}{0065CA}
    \definecolor{ansi-magenta}{HTML}{D160C4}
    \definecolor{ansi-magenta-intense}{HTML}{A03196}
    \definecolor{ansi-cyan}{HTML}{60C6C8}
    \definecolor{ansi-cyan-intense}{HTML}{258F8F}
    \definecolor{ansi-white}{HTML}{C5C1B4}
    \definecolor{ansi-white-intense}{HTML}{A1A6B2}
    \definecolor{ansi-default-inverse-fg}{HTML}{FFFFFF}
    \definecolor{ansi-default-inverse-bg}{HTML}{000000}

    % commands and environments needed by pandoc snippets
    % extracted from the output of `pandoc -s`
    \providecommand{\tightlist}{%
      \setlength{\itemsep}{0pt}\setlength{\parskip}{0pt}}
    \DefineVerbatimEnvironment{Highlighting}{Verbatim}{commandchars=\\\{\}}
    % Add ',fontsize=\small' for more characters per line
    \newenvironment{Shaded}{}{}
    \newcommand{\KeywordTok}[1]{\textcolor[rgb]{0.00,0.44,0.13}{\textbf{{#1}}}}
    \newcommand{\DataTypeTok}[1]{\textcolor[rgb]{0.56,0.13,0.00}{{#1}}}
    \newcommand{\DecValTok}[1]{\textcolor[rgb]{0.25,0.63,0.44}{{#1}}}
    \newcommand{\BaseNTok}[1]{\textcolor[rgb]{0.25,0.63,0.44}{{#1}}}
    \newcommand{\FloatTok}[1]{\textcolor[rgb]{0.25,0.63,0.44}{{#1}}}
    \newcommand{\CharTok}[1]{\textcolor[rgb]{0.25,0.44,0.63}{{#1}}}
    \newcommand{\StringTok}[1]{\textcolor[rgb]{0.25,0.44,0.63}{{#1}}}
    \newcommand{\CommentTok}[1]{\textcolor[rgb]{0.38,0.63,0.69}{\textit{{#1}}}}
    \newcommand{\OtherTok}[1]{\textcolor[rgb]{0.00,0.44,0.13}{{#1}}}
    \newcommand{\AlertTok}[1]{\textcolor[rgb]{1.00,0.00,0.00}{\textbf{{#1}}}}
    \newcommand{\FunctionTok}[1]{\textcolor[rgb]{0.02,0.16,0.49}{{#1}}}
    \newcommand{\RegionMarkerTok}[1]{{#1}}
    \newcommand{\ErrorTok}[1]{\textcolor[rgb]{1.00,0.00,0.00}{\textbf{{#1}}}}
    \newcommand{\NormalTok}[1]{{#1}}
    
    % Additional commands for more recent versions of Pandoc
    \newcommand{\ConstantTok}[1]{\textcolor[rgb]{0.53,0.00,0.00}{{#1}}}
    \newcommand{\SpecialCharTok}[1]{\textcolor[rgb]{0.25,0.44,0.63}{{#1}}}
    \newcommand{\VerbatimStringTok}[1]{\textcolor[rgb]{0.25,0.44,0.63}{{#1}}}
    \newcommand{\SpecialStringTok}[1]{\textcolor[rgb]{0.73,0.40,0.53}{{#1}}}
    \newcommand{\ImportTok}[1]{{#1}}
    \newcommand{\DocumentationTok}[1]{\textcolor[rgb]{0.73,0.13,0.13}{\textit{{#1}}}}
    \newcommand{\AnnotationTok}[1]{\textcolor[rgb]{0.38,0.63,0.69}{\textbf{\textit{{#1}}}}}
    \newcommand{\CommentVarTok}[1]{\textcolor[rgb]{0.38,0.63,0.69}{\textbf{\textit{{#1}}}}}
    \newcommand{\VariableTok}[1]{\textcolor[rgb]{0.10,0.09,0.49}{{#1}}}
    \newcommand{\ControlFlowTok}[1]{\textcolor[rgb]{0.00,0.44,0.13}{\textbf{{#1}}}}
    \newcommand{\OperatorTok}[1]{\textcolor[rgb]{0.40,0.40,0.40}{{#1}}}
    \newcommand{\BuiltInTok}[1]{{#1}}
    \newcommand{\ExtensionTok}[1]{{#1}}
    \newcommand{\PreprocessorTok}[1]{\textcolor[rgb]{0.74,0.48,0.00}{{#1}}}
    \newcommand{\AttributeTok}[1]{\textcolor[rgb]{0.49,0.56,0.16}{{#1}}}
    \newcommand{\InformationTok}[1]{\textcolor[rgb]{0.38,0.63,0.69}{\textbf{\textit{{#1}}}}}
    \newcommand{\WarningTok}[1]{\textcolor[rgb]{0.38,0.63,0.69}{\textbf{\textit{{#1}}}}}
    
    
    % Define a nice break command that doesn't care if a line doesn't already
    % exist.
    \def\br{\hspace*{\fill} \\* }
    % Math Jax compatibility definitions
    \def\gt{>}
    \def\lt{<}
    \let\Oldtex\TeX
    \let\Oldlatex\LaTeX
    \renewcommand{\TeX}{\textrm{\Oldtex}}
    \renewcommand{\LaTeX}{\textrm{\Oldlatex}}
    % Document parameters
    % Document title
    \title{C1\_W3\_lecture\_nb\_02\_manipulating\_word\_embeddings}
    
    
    
    
    
% Pygments definitions
\makeatletter
\def\PY@reset{\let\PY@it=\relax \let\PY@bf=\relax%
    \let\PY@ul=\relax \let\PY@tc=\relax%
    \let\PY@bc=\relax \let\PY@ff=\relax}
\def\PY@tok#1{\csname PY@tok@#1\endcsname}
\def\PY@toks#1+{\ifx\relax#1\empty\else%
    \PY@tok{#1}\expandafter\PY@toks\fi}
\def\PY@do#1{\PY@bc{\PY@tc{\PY@ul{%
    \PY@it{\PY@bf{\PY@ff{#1}}}}}}}
\def\PY#1#2{\PY@reset\PY@toks#1+\relax+\PY@do{#2}}

\expandafter\def\csname PY@tok@w\endcsname{\def\PY@tc##1{\textcolor[rgb]{0.73,0.73,0.73}{##1}}}
\expandafter\def\csname PY@tok@c\endcsname{\let\PY@it=\textit\def\PY@tc##1{\textcolor[rgb]{0.25,0.50,0.50}{##1}}}
\expandafter\def\csname PY@tok@cp\endcsname{\def\PY@tc##1{\textcolor[rgb]{0.74,0.48,0.00}{##1}}}
\expandafter\def\csname PY@tok@k\endcsname{\let\PY@bf=\textbf\def\PY@tc##1{\textcolor[rgb]{0.00,0.50,0.00}{##1}}}
\expandafter\def\csname PY@tok@kp\endcsname{\def\PY@tc##1{\textcolor[rgb]{0.00,0.50,0.00}{##1}}}
\expandafter\def\csname PY@tok@kt\endcsname{\def\PY@tc##1{\textcolor[rgb]{0.69,0.00,0.25}{##1}}}
\expandafter\def\csname PY@tok@o\endcsname{\def\PY@tc##1{\textcolor[rgb]{0.40,0.40,0.40}{##1}}}
\expandafter\def\csname PY@tok@ow\endcsname{\let\PY@bf=\textbf\def\PY@tc##1{\textcolor[rgb]{0.67,0.13,1.00}{##1}}}
\expandafter\def\csname PY@tok@nb\endcsname{\def\PY@tc##1{\textcolor[rgb]{0.00,0.50,0.00}{##1}}}
\expandafter\def\csname PY@tok@nf\endcsname{\def\PY@tc##1{\textcolor[rgb]{0.00,0.00,1.00}{##1}}}
\expandafter\def\csname PY@tok@nc\endcsname{\let\PY@bf=\textbf\def\PY@tc##1{\textcolor[rgb]{0.00,0.00,1.00}{##1}}}
\expandafter\def\csname PY@tok@nn\endcsname{\let\PY@bf=\textbf\def\PY@tc##1{\textcolor[rgb]{0.00,0.00,1.00}{##1}}}
\expandafter\def\csname PY@tok@ne\endcsname{\let\PY@bf=\textbf\def\PY@tc##1{\textcolor[rgb]{0.82,0.25,0.23}{##1}}}
\expandafter\def\csname PY@tok@nv\endcsname{\def\PY@tc##1{\textcolor[rgb]{0.10,0.09,0.49}{##1}}}
\expandafter\def\csname PY@tok@no\endcsname{\def\PY@tc##1{\textcolor[rgb]{0.53,0.00,0.00}{##1}}}
\expandafter\def\csname PY@tok@nl\endcsname{\def\PY@tc##1{\textcolor[rgb]{0.63,0.63,0.00}{##1}}}
\expandafter\def\csname PY@tok@ni\endcsname{\let\PY@bf=\textbf\def\PY@tc##1{\textcolor[rgb]{0.60,0.60,0.60}{##1}}}
\expandafter\def\csname PY@tok@na\endcsname{\def\PY@tc##1{\textcolor[rgb]{0.49,0.56,0.16}{##1}}}
\expandafter\def\csname PY@tok@nt\endcsname{\let\PY@bf=\textbf\def\PY@tc##1{\textcolor[rgb]{0.00,0.50,0.00}{##1}}}
\expandafter\def\csname PY@tok@nd\endcsname{\def\PY@tc##1{\textcolor[rgb]{0.67,0.13,1.00}{##1}}}
\expandafter\def\csname PY@tok@s\endcsname{\def\PY@tc##1{\textcolor[rgb]{0.73,0.13,0.13}{##1}}}
\expandafter\def\csname PY@tok@sd\endcsname{\let\PY@it=\textit\def\PY@tc##1{\textcolor[rgb]{0.73,0.13,0.13}{##1}}}
\expandafter\def\csname PY@tok@si\endcsname{\let\PY@bf=\textbf\def\PY@tc##1{\textcolor[rgb]{0.73,0.40,0.53}{##1}}}
\expandafter\def\csname PY@tok@se\endcsname{\let\PY@bf=\textbf\def\PY@tc##1{\textcolor[rgb]{0.73,0.40,0.13}{##1}}}
\expandafter\def\csname PY@tok@sr\endcsname{\def\PY@tc##1{\textcolor[rgb]{0.73,0.40,0.53}{##1}}}
\expandafter\def\csname PY@tok@ss\endcsname{\def\PY@tc##1{\textcolor[rgb]{0.10,0.09,0.49}{##1}}}
\expandafter\def\csname PY@tok@sx\endcsname{\def\PY@tc##1{\textcolor[rgb]{0.00,0.50,0.00}{##1}}}
\expandafter\def\csname PY@tok@m\endcsname{\def\PY@tc##1{\textcolor[rgb]{0.40,0.40,0.40}{##1}}}
\expandafter\def\csname PY@tok@gh\endcsname{\let\PY@bf=\textbf\def\PY@tc##1{\textcolor[rgb]{0.00,0.00,0.50}{##1}}}
\expandafter\def\csname PY@tok@gu\endcsname{\let\PY@bf=\textbf\def\PY@tc##1{\textcolor[rgb]{0.50,0.00,0.50}{##1}}}
\expandafter\def\csname PY@tok@gd\endcsname{\def\PY@tc##1{\textcolor[rgb]{0.63,0.00,0.00}{##1}}}
\expandafter\def\csname PY@tok@gi\endcsname{\def\PY@tc##1{\textcolor[rgb]{0.00,0.63,0.00}{##1}}}
\expandafter\def\csname PY@tok@gr\endcsname{\def\PY@tc##1{\textcolor[rgb]{1.00,0.00,0.00}{##1}}}
\expandafter\def\csname PY@tok@ge\endcsname{\let\PY@it=\textit}
\expandafter\def\csname PY@tok@gs\endcsname{\let\PY@bf=\textbf}
\expandafter\def\csname PY@tok@gp\endcsname{\let\PY@bf=\textbf\def\PY@tc##1{\textcolor[rgb]{0.00,0.00,0.50}{##1}}}
\expandafter\def\csname PY@tok@go\endcsname{\def\PY@tc##1{\textcolor[rgb]{0.53,0.53,0.53}{##1}}}
\expandafter\def\csname PY@tok@gt\endcsname{\def\PY@tc##1{\textcolor[rgb]{0.00,0.27,0.87}{##1}}}
\expandafter\def\csname PY@tok@err\endcsname{\def\PY@bc##1{\setlength{\fboxsep}{0pt}\fcolorbox[rgb]{1.00,0.00,0.00}{1,1,1}{\strut ##1}}}
\expandafter\def\csname PY@tok@kc\endcsname{\let\PY@bf=\textbf\def\PY@tc##1{\textcolor[rgb]{0.00,0.50,0.00}{##1}}}
\expandafter\def\csname PY@tok@kd\endcsname{\let\PY@bf=\textbf\def\PY@tc##1{\textcolor[rgb]{0.00,0.50,0.00}{##1}}}
\expandafter\def\csname PY@tok@kn\endcsname{\let\PY@bf=\textbf\def\PY@tc##1{\textcolor[rgb]{0.00,0.50,0.00}{##1}}}
\expandafter\def\csname PY@tok@kr\endcsname{\let\PY@bf=\textbf\def\PY@tc##1{\textcolor[rgb]{0.00,0.50,0.00}{##1}}}
\expandafter\def\csname PY@tok@bp\endcsname{\def\PY@tc##1{\textcolor[rgb]{0.00,0.50,0.00}{##1}}}
\expandafter\def\csname PY@tok@fm\endcsname{\def\PY@tc##1{\textcolor[rgb]{0.00,0.00,1.00}{##1}}}
\expandafter\def\csname PY@tok@vc\endcsname{\def\PY@tc##1{\textcolor[rgb]{0.10,0.09,0.49}{##1}}}
\expandafter\def\csname PY@tok@vg\endcsname{\def\PY@tc##1{\textcolor[rgb]{0.10,0.09,0.49}{##1}}}
\expandafter\def\csname PY@tok@vi\endcsname{\def\PY@tc##1{\textcolor[rgb]{0.10,0.09,0.49}{##1}}}
\expandafter\def\csname PY@tok@vm\endcsname{\def\PY@tc##1{\textcolor[rgb]{0.10,0.09,0.49}{##1}}}
\expandafter\def\csname PY@tok@sa\endcsname{\def\PY@tc##1{\textcolor[rgb]{0.73,0.13,0.13}{##1}}}
\expandafter\def\csname PY@tok@sb\endcsname{\def\PY@tc##1{\textcolor[rgb]{0.73,0.13,0.13}{##1}}}
\expandafter\def\csname PY@tok@sc\endcsname{\def\PY@tc##1{\textcolor[rgb]{0.73,0.13,0.13}{##1}}}
\expandafter\def\csname PY@tok@dl\endcsname{\def\PY@tc##1{\textcolor[rgb]{0.73,0.13,0.13}{##1}}}
\expandafter\def\csname PY@tok@s2\endcsname{\def\PY@tc##1{\textcolor[rgb]{0.73,0.13,0.13}{##1}}}
\expandafter\def\csname PY@tok@sh\endcsname{\def\PY@tc##1{\textcolor[rgb]{0.73,0.13,0.13}{##1}}}
\expandafter\def\csname PY@tok@s1\endcsname{\def\PY@tc##1{\textcolor[rgb]{0.73,0.13,0.13}{##1}}}
\expandafter\def\csname PY@tok@mb\endcsname{\def\PY@tc##1{\textcolor[rgb]{0.40,0.40,0.40}{##1}}}
\expandafter\def\csname PY@tok@mf\endcsname{\def\PY@tc##1{\textcolor[rgb]{0.40,0.40,0.40}{##1}}}
\expandafter\def\csname PY@tok@mh\endcsname{\def\PY@tc##1{\textcolor[rgb]{0.40,0.40,0.40}{##1}}}
\expandafter\def\csname PY@tok@mi\endcsname{\def\PY@tc##1{\textcolor[rgb]{0.40,0.40,0.40}{##1}}}
\expandafter\def\csname PY@tok@il\endcsname{\def\PY@tc##1{\textcolor[rgb]{0.40,0.40,0.40}{##1}}}
\expandafter\def\csname PY@tok@mo\endcsname{\def\PY@tc##1{\textcolor[rgb]{0.40,0.40,0.40}{##1}}}
\expandafter\def\csname PY@tok@ch\endcsname{\let\PY@it=\textit\def\PY@tc##1{\textcolor[rgb]{0.25,0.50,0.50}{##1}}}
\expandafter\def\csname PY@tok@cm\endcsname{\let\PY@it=\textit\def\PY@tc##1{\textcolor[rgb]{0.25,0.50,0.50}{##1}}}
\expandafter\def\csname PY@tok@cpf\endcsname{\let\PY@it=\textit\def\PY@tc##1{\textcolor[rgb]{0.25,0.50,0.50}{##1}}}
\expandafter\def\csname PY@tok@c1\endcsname{\let\PY@it=\textit\def\PY@tc##1{\textcolor[rgb]{0.25,0.50,0.50}{##1}}}
\expandafter\def\csname PY@tok@cs\endcsname{\let\PY@it=\textit\def\PY@tc##1{\textcolor[rgb]{0.25,0.50,0.50}{##1}}}

\def\PYZbs{\char`\\}
\def\PYZus{\char`\_}
\def\PYZob{\char`\{}
\def\PYZcb{\char`\}}
\def\PYZca{\char`\^}
\def\PYZam{\char`\&}
\def\PYZlt{\char`\<}
\def\PYZgt{\char`\>}
\def\PYZsh{\char`\#}
\def\PYZpc{\char`\%}
\def\PYZdl{\char`\$}
\def\PYZhy{\char`\-}
\def\PYZsq{\char`\'}
\def\PYZdq{\char`\"}
\def\PYZti{\char`\~}
% for compatibility with earlier versions
\def\PYZat{@}
\def\PYZlb{[}
\def\PYZrb{]}
\makeatother


    % For linebreaks inside Verbatim environment from package fancyvrb. 
    \makeatletter
        \newbox\Wrappedcontinuationbox 
        \newbox\Wrappedvisiblespacebox 
        \newcommand*\Wrappedvisiblespace {\textcolor{red}{\textvisiblespace}} 
        \newcommand*\Wrappedcontinuationsymbol {\textcolor{red}{\llap{\tiny$\m@th\hookrightarrow$}}} 
        \newcommand*\Wrappedcontinuationindent {3ex } 
        \newcommand*\Wrappedafterbreak {\kern\Wrappedcontinuationindent\copy\Wrappedcontinuationbox} 
        % Take advantage of the already applied Pygments mark-up to insert 
        % potential linebreaks for TeX processing. 
        %        {, <, #, %, $, ' and ": go to next line. 
        %        _, }, ^, &, >, - and ~: stay at end of broken line. 
        % Use of \textquotesingle for straight quote. 
        \newcommand*\Wrappedbreaksatspecials {% 
            \def\PYGZus{\discretionary{\char`\_}{\Wrappedafterbreak}{\char`\_}}% 
            \def\PYGZob{\discretionary{}{\Wrappedafterbreak\char`\{}{\char`\{}}% 
            \def\PYGZcb{\discretionary{\char`\}}{\Wrappedafterbreak}{\char`\}}}% 
            \def\PYGZca{\discretionary{\char`\^}{\Wrappedafterbreak}{\char`\^}}% 
            \def\PYGZam{\discretionary{\char`\&}{\Wrappedafterbreak}{\char`\&}}% 
            \def\PYGZlt{\discretionary{}{\Wrappedafterbreak\char`\<}{\char`\<}}% 
            \def\PYGZgt{\discretionary{\char`\>}{\Wrappedafterbreak}{\char`\>}}% 
            \def\PYGZsh{\discretionary{}{\Wrappedafterbreak\char`\#}{\char`\#}}% 
            \def\PYGZpc{\discretionary{}{\Wrappedafterbreak\char`\%}{\char`\%}}% 
            \def\PYGZdl{\discretionary{}{\Wrappedafterbreak\char`\$}{\char`\$}}% 
            \def\PYGZhy{\discretionary{\char`\-}{\Wrappedafterbreak}{\char`\-}}% 
            \def\PYGZsq{\discretionary{}{\Wrappedafterbreak\textquotesingle}{\textquotesingle}}% 
            \def\PYGZdq{\discretionary{}{\Wrappedafterbreak\char`\"}{\char`\"}}% 
            \def\PYGZti{\discretionary{\char`\~}{\Wrappedafterbreak}{\char`\~}}% 
        } 
        % Some characters . , ; ? ! / are not pygmentized. 
        % This macro makes them "active" and they will insert potential linebreaks 
        \newcommand*\Wrappedbreaksatpunct {% 
            \lccode`\~`\.\lowercase{\def~}{\discretionary{\hbox{\char`\.}}{\Wrappedafterbreak}{\hbox{\char`\.}}}% 
            \lccode`\~`\,\lowercase{\def~}{\discretionary{\hbox{\char`\,}}{\Wrappedafterbreak}{\hbox{\char`\,}}}% 
            \lccode`\~`\;\lowercase{\def~}{\discretionary{\hbox{\char`\;}}{\Wrappedafterbreak}{\hbox{\char`\;}}}% 
            \lccode`\~`\:\lowercase{\def~}{\discretionary{\hbox{\char`\:}}{\Wrappedafterbreak}{\hbox{\char`\:}}}% 
            \lccode`\~`\?\lowercase{\def~}{\discretionary{\hbox{\char`\?}}{\Wrappedafterbreak}{\hbox{\char`\?}}}% 
            \lccode`\~`\!\lowercase{\def~}{\discretionary{\hbox{\char`\!}}{\Wrappedafterbreak}{\hbox{\char`\!}}}% 
            \lccode`\~`\/\lowercase{\def~}{\discretionary{\hbox{\char`\/}}{\Wrappedafterbreak}{\hbox{\char`\/}}}% 
            \catcode`\.\active
            \catcode`\,\active 
            \catcode`\;\active
            \catcode`\:\active
            \catcode`\?\active
            \catcode`\!\active
            \catcode`\/\active 
            \lccode`\~`\~ 	
        }
    \makeatother

    \let\OriginalVerbatim=\Verbatim
    \makeatletter
    \renewcommand{\Verbatim}[1][1]{%
        %\parskip\z@skip
        \sbox\Wrappedcontinuationbox {\Wrappedcontinuationsymbol}%
        \sbox\Wrappedvisiblespacebox {\FV@SetupFont\Wrappedvisiblespace}%
        \def\FancyVerbFormatLine ##1{\hsize\linewidth
            \vtop{\raggedright\hyphenpenalty\z@\exhyphenpenalty\z@
                \doublehyphendemerits\z@\finalhyphendemerits\z@
                \strut ##1\strut}%
        }%
        % If the linebreak is at a space, the latter will be displayed as visible
        % space at end of first line, and a continuation symbol starts next line.
        % Stretch/shrink are however usually zero for typewriter font.
        \def\FV@Space {%
            \nobreak\hskip\z@ plus\fontdimen3\font minus\fontdimen4\font
            \discretionary{\copy\Wrappedvisiblespacebox}{\Wrappedafterbreak}
            {\kern\fontdimen2\font}%
        }%
        
        % Allow breaks at special characters using \PYG... macros.
        \Wrappedbreaksatspecials
        % Breaks at punctuation characters . , ; ? ! and / need catcode=\active 	
        \OriginalVerbatim[#1,codes*=\Wrappedbreaksatpunct]%
    }
    \makeatother

    % Exact colors from NB
    \definecolor{incolor}{HTML}{303F9F}
    \definecolor{outcolor}{HTML}{D84315}
    \definecolor{cellborder}{HTML}{CFCFCF}
    \definecolor{cellbackground}{HTML}{F7F7F7}
    
    % prompt
    \makeatletter
    \newcommand{\boxspacing}{\kern\kvtcb@left@rule\kern\kvtcb@boxsep}
    \makeatother
    \newcommand{\prompt}[4]{
        \ttfamily\llap{{\color{#2}[#3]:\hspace{3pt}#4}}\vspace{-\baselineskip}
    }
    

    
    % Prevent overflowing lines due to hard-to-break entities
    \sloppy 
    % Setup hyperref package
    \hypersetup{
      breaklinks=true,  % so long urls are correctly broken across lines
      colorlinks=true,
      urlcolor=urlcolor,
      linkcolor=linkcolor,
      citecolor=citecolor,
      }
    % Slightly bigger margins than the latex defaults
    
    \geometry{verbose,tmargin=1in,bmargin=1in,lmargin=1in,rmargin=1in}
    
    

\begin{document}
    
    \maketitle
    
    

    
    \hypertarget{manipulating-word-embeddings}{%
\section{Manipulating word
embeddings}\label{manipulating-word-embeddings}}

In this week's assignment, you are going to use a pre-trained word
embedding for finding word analogies and equivalence. This exercise can
be used as an Intrinsic Evaluation for the word embedding performance.
In this notebook, you will apply linear algebra operations using NumPy
to find analogies between words manually. This will help you to prepare
for this week's assignment.

    \begin{tcolorbox}[breakable, size=fbox, boxrule=1pt, pad at break*=1mm,colback=cellbackground, colframe=cellborder]
\prompt{In}{incolor}{1}{\boxspacing}
\begin{Verbatim}[commandchars=\\\{\}]
\PY{k+kn}{import} \PY{n+nn}{pandas} \PY{k}{as} \PY{n+nn}{pd} \PY{c+c1}{\PYZsh{} Library for Dataframes }
\PY{k+kn}{import} \PY{n+nn}{numpy} \PY{k}{as} \PY{n+nn}{np} \PY{c+c1}{\PYZsh{} Library for math functions}
\PY{k+kn}{import} \PY{n+nn}{pickle} \PY{c+c1}{\PYZsh{} Python object serialization library. Not secure}

\PY{n}{word\PYZus{}embeddings} \PY{o}{=} \PY{n}{pickle}\PY{o}{.}\PY{n}{load}\PY{p}{(} \PY{n+nb}{open}\PY{p}{(} \PY{l+s+s2}{\PYZdq{}}\PY{l+s+s2}{./data/word\PYZus{}embeddings\PYZus{}subset.p}\PY{l+s+s2}{\PYZdq{}}\PY{p}{,} \PY{l+s+s2}{\PYZdq{}}\PY{l+s+s2}{rb}\PY{l+s+s2}{\PYZdq{}} \PY{p}{)} \PY{p}{)}
\PY{n+nb}{len}\PY{p}{(}\PY{n}{word\PYZus{}embeddings}\PY{p}{)} \PY{c+c1}{\PYZsh{} there should be 243 words that will be used in this assignment}
\end{Verbatim}
\end{tcolorbox}

            \begin{tcolorbox}[breakable, size=fbox, boxrule=.5pt, pad at break*=1mm, opacityfill=0]
\prompt{Out}{outcolor}{1}{\boxspacing}
\begin{Verbatim}[commandchars=\\\{\}]
243
\end{Verbatim}
\end{tcolorbox}
        
    Now that the model is loaded, we can take a look at the word
representations. First, note that \textbf{word\_embeddings} is a
dictionary. Each word is the key to the entry, and the value is its
corresponding vector presentation. Remember that square brackets allow
access to any entry if the key exists.

    \begin{tcolorbox}[breakable, size=fbox, boxrule=1pt, pad at break*=1mm,colback=cellbackground, colframe=cellborder]
\prompt{In}{incolor}{2}{\boxspacing}
\begin{Verbatim}[commandchars=\\\{\}]
\PY{n}{countryVector} \PY{o}{=} \PY{n}{word\PYZus{}embeddings}\PY{p}{[}\PY{l+s+s1}{\PYZsq{}}\PY{l+s+s1}{country}\PY{l+s+s1}{\PYZsq{}}\PY{p}{]} \PY{c+c1}{\PYZsh{} Get the vector representation for the word \PYZsq{}country\PYZsq{}}
\PY{n+nb}{print}\PY{p}{(}\PY{n+nb}{type}\PY{p}{(}\PY{n}{countryVector}\PY{p}{)}\PY{p}{)} \PY{c+c1}{\PYZsh{} Print the type of the vector. Note it is a numpy array}
\PY{n+nb}{print}\PY{p}{(}\PY{n}{countryVector}\PY{p}{)} \PY{c+c1}{\PYZsh{} Print the values of the vector.  }
\end{Verbatim}
\end{tcolorbox}

    \begin{Verbatim}[commandchars=\\\{\}]
<class 'numpy.ndarray'>
[-0.08007812  0.13378906  0.14355469  0.09472656 -0.04736328 -0.02355957
 -0.00854492 -0.18652344  0.04589844 -0.08154297 -0.03442383 -0.11621094
  0.21777344 -0.10351562 -0.06689453  0.15332031 -0.19335938  0.26367188
 -0.13671875 -0.05566406  0.07470703 -0.00070953  0.09375    -0.14453125
  0.04296875 -0.01916504 -0.22558594 -0.12695312 -0.0168457   0.05224609
  0.0625     -0.1484375  -0.01965332  0.17578125  0.10644531 -0.04760742
 -0.10253906 -0.28515625  0.10351562  0.20800781 -0.07617188 -0.04345703
  0.08642578  0.08740234  0.11767578  0.20996094 -0.07275391  0.1640625
 -0.01135254  0.0025177   0.05810547 -0.03222656  0.06884766  0.046875
  0.10107422  0.02148438 -0.16210938  0.07128906 -0.16210938  0.05981445
  0.05102539 -0.05566406  0.06787109 -0.03759766  0.04345703 -0.03173828
 -0.03417969 -0.01116943  0.06201172 -0.08007812 -0.14941406  0.11914062
  0.02575684  0.00302124  0.04711914 -0.17773438  0.04101562  0.05541992
  0.00598145  0.03027344 -0.07666016 -0.109375    0.02832031 -0.10498047
  0.0100708  -0.03149414 -0.22363281 -0.03125    -0.01147461  0.17285156
  0.08056641 -0.10888672 -0.09570312 -0.21777344 -0.07910156 -0.10009766
  0.06396484 -0.11962891  0.18652344 -0.02062988 -0.02172852  0.29296875
 -0.00793457  0.0324707  -0.15136719  0.00227356 -0.03540039 -0.13378906
  0.0546875  -0.03271484 -0.01855469 -0.10302734 -0.13378906  0.11425781
  0.16699219  0.01361084 -0.02722168 -0.2109375   0.07177734  0.08691406
 -0.09960938  0.01422119 -0.18261719  0.00741577  0.01965332  0.00738525
 -0.03271484 -0.15234375 -0.26367188 -0.14746094  0.03320312 -0.03344727
 -0.01000977  0.01855469  0.00183868 -0.10498047  0.09667969  0.07910156
  0.11181641  0.13085938 -0.08740234 -0.1328125   0.05004883  0.19824219
  0.0612793   0.16210938  0.06933594  0.01281738  0.01550293  0.01531982
  0.11474609  0.02758789  0.13769531 -0.08349609  0.01123047 -0.20507812
 -0.12988281 -0.16699219  0.20410156 -0.03588867 -0.10888672  0.0534668
  0.15820312 -0.20410156  0.14648438 -0.11572266  0.01855469 -0.13574219
  0.24121094  0.12304688 -0.14550781  0.17578125  0.11816406 -0.30859375
  0.10888672 -0.22363281  0.19335938 -0.15722656 -0.07666016 -0.09082031
 -0.19628906 -0.23144531 -0.09130859 -0.14160156  0.06347656  0.03344727
 -0.03369141  0.06591797  0.06201172  0.3046875   0.16796875 -0.11035156
 -0.03833008 -0.02563477 -0.09765625  0.04467773 -0.0534668   0.11621094
 -0.15039062 -0.16308594 -0.15527344  0.04638672  0.11572266 -0.06640625
 -0.04516602  0.02331543 -0.08105469 -0.0255127  -0.07714844  0.0016861
  0.15820312  0.00994873 -0.06445312  0.15722656 -0.03112793  0.10644531
 -0.140625    0.23535156 -0.11279297  0.16015625  0.00061798 -0.1484375
  0.02307129 -0.109375    0.05444336 -0.14160156  0.11621094  0.03710938
  0.14746094 -0.04199219 -0.01391602 -0.03881836  0.02783203  0.10205078
  0.07470703  0.20898438 -0.04223633 -0.04150391 -0.00588989 -0.14941406
 -0.04296875 -0.10107422 -0.06176758  0.09472656  0.22265625 -0.02307129
  0.04858398 -0.15527344 -0.02282715 -0.04174805  0.16699219 -0.09423828
  0.14453125  0.11132812  0.04223633 -0.16699219  0.10253906  0.16796875
  0.12597656 -0.11865234 -0.0213623  -0.08056641  0.24316406  0.15527344
  0.16503906  0.00854492 -0.12255859  0.08691406 -0.11914062 -0.02941895
  0.08349609 -0.03100586  0.13964844 -0.05151367  0.00765991 -0.04443359
 -0.04980469 -0.03222656 -0.00952148 -0.10888672 -0.10302734 -0.15722656
  0.19335938  0.04858398  0.015625   -0.08105469 -0.11621094 -0.01989746
  0.05737305  0.06103516 -0.14550781  0.06738281 -0.24414062 -0.07714844
  0.04760742 -0.07519531 -0.14941406 -0.04418945  0.09716797  0.06738281]
    \end{Verbatim}

    It is important to note that we store each vector as a NumPy array. It
allows us to use the linear algebra operations on it.

The vectors have a size of 300, while the vocabulary size of Google News
is around 3 million words!

    \begin{tcolorbox}[breakable, size=fbox, boxrule=1pt, pad at break*=1mm,colback=cellbackground, colframe=cellborder]
\prompt{In}{incolor}{3}{\boxspacing}
\begin{Verbatim}[commandchars=\\\{\}]
\PY{c+c1}{\PYZsh{}Get the vector for a given word:}
\PY{k}{def} \PY{n+nf}{vec}\PY{p}{(}\PY{n}{w}\PY{p}{)}\PY{p}{:}
    \PY{k}{return} \PY{n}{word\PYZus{}embeddings}\PY{p}{[}\PY{n}{w}\PY{p}{]}
\end{Verbatim}
\end{tcolorbox}

    \hypertarget{operating-on-word-embeddings}{%
\subsection{Operating on word
embeddings}\label{operating-on-word-embeddings}}

Remember that understanding the data is one of the most critical steps
in Data Science. Word embeddings are the result of machine learning
processes and will be part of the input for further processes. These
word embedding needs to be validated or at least understood because the
performance of the derived model will strongly depend on its quality.

Word embeddings are multidimensional arrays, usually with hundreds of
attributes that pose a challenge for its interpretation.

In this notebook, we will visually inspect the word embedding of some
words using a pair of attributes. Raw attributes are not the best option
for the creation of such charts but will allow us to illustrate the
mechanical part in Python.

In the next cell, we make a beautiful plot for the word embeddings of
some words. Even if plotting the dots gives an idea of the words, the
arrow representations help to visualize the vector's alignment as well.

    \begin{tcolorbox}[breakable, size=fbox, boxrule=1pt, pad at break*=1mm,colback=cellbackground, colframe=cellborder]
\prompt{In}{incolor}{4}{\boxspacing}
\begin{Verbatim}[commandchars=\\\{\}]
\PY{k+kn}{import} \PY{n+nn}{matplotlib}\PY{n+nn}{.}\PY{n+nn}{pyplot} \PY{k}{as} \PY{n+nn}{plt} \PY{c+c1}{\PYZsh{} Import matplotlib}
\PY{o}{\PYZpc{}}\PY{k}{matplotlib} inline

\PY{n}{words} \PY{o}{=} \PY{p}{[}\PY{l+s+s1}{\PYZsq{}}\PY{l+s+s1}{oil}\PY{l+s+s1}{\PYZsq{}}\PY{p}{,} \PY{l+s+s1}{\PYZsq{}}\PY{l+s+s1}{gas}\PY{l+s+s1}{\PYZsq{}}\PY{p}{,} \PY{l+s+s1}{\PYZsq{}}\PY{l+s+s1}{happy}\PY{l+s+s1}{\PYZsq{}}\PY{p}{,} \PY{l+s+s1}{\PYZsq{}}\PY{l+s+s1}{sad}\PY{l+s+s1}{\PYZsq{}}\PY{p}{,} \PY{l+s+s1}{\PYZsq{}}\PY{l+s+s1}{city}\PY{l+s+s1}{\PYZsq{}}\PY{p}{,} \PY{l+s+s1}{\PYZsq{}}\PY{l+s+s1}{town}\PY{l+s+s1}{\PYZsq{}}\PY{p}{,} \PY{l+s+s1}{\PYZsq{}}\PY{l+s+s1}{village}\PY{l+s+s1}{\PYZsq{}}\PY{p}{,} \PY{l+s+s1}{\PYZsq{}}\PY{l+s+s1}{country}\PY{l+s+s1}{\PYZsq{}}\PY{p}{,} \PY{l+s+s1}{\PYZsq{}}\PY{l+s+s1}{continent}\PY{l+s+s1}{\PYZsq{}}\PY{p}{,} \PY{l+s+s1}{\PYZsq{}}\PY{l+s+s1}{petroleum}\PY{l+s+s1}{\PYZsq{}}\PY{p}{,} \PY{l+s+s1}{\PYZsq{}}\PY{l+s+s1}{joyful}\PY{l+s+s1}{\PYZsq{}}\PY{p}{]}

\PY{n}{bag2d} \PY{o}{=} \PY{n}{np}\PY{o}{.}\PY{n}{array}\PY{p}{(}\PY{p}{[}\PY{n}{vec}\PY{p}{(}\PY{n}{word}\PY{p}{)} \PY{k}{for} \PY{n}{word} \PY{o+ow}{in} \PY{n}{words}\PY{p}{]}\PY{p}{)} \PY{c+c1}{\PYZsh{} Convert each word to its vector representation}

\PY{n}{fig}\PY{p}{,} \PY{n}{ax} \PY{o}{=} \PY{n}{plt}\PY{o}{.}\PY{n}{subplots}\PY{p}{(}\PY{n}{figsize} \PY{o}{=} \PY{p}{(}\PY{l+m+mi}{10}\PY{p}{,} \PY{l+m+mi}{10}\PY{p}{)}\PY{p}{)} \PY{c+c1}{\PYZsh{} Create custom size image}

\PY{n}{col1} \PY{o}{=} \PY{l+m+mi}{3} \PY{c+c1}{\PYZsh{} Select the column for the x axis}
\PY{n}{col2} \PY{o}{=} \PY{l+m+mi}{2} \PY{c+c1}{\PYZsh{} Select the column for the y axis}

\PY{c+c1}{\PYZsh{} Print an arrow for each word}
\PY{k}{for} \PY{n}{word} \PY{o+ow}{in} \PY{n}{bag2d}\PY{p}{:}
    \PY{n}{ax}\PY{o}{.}\PY{n}{arrow}\PY{p}{(}\PY{l+m+mi}{0}\PY{p}{,} \PY{l+m+mi}{0}\PY{p}{,} \PY{n}{word}\PY{p}{[}\PY{n}{col1}\PY{p}{]}\PY{p}{,} \PY{n}{word}\PY{p}{[}\PY{n}{col2}\PY{p}{]}\PY{p}{,} \PY{n}{head\PYZus{}width}\PY{o}{=}\PY{l+m+mf}{0.005}\PY{p}{,} \PY{n}{head\PYZus{}length}\PY{o}{=}\PY{l+m+mf}{0.005}\PY{p}{,} \PY{n}{fc}\PY{o}{=}\PY{l+s+s1}{\PYZsq{}}\PY{l+s+s1}{r}\PY{l+s+s1}{\PYZsq{}}\PY{p}{,} \PY{n}{ec}\PY{o}{=}\PY{l+s+s1}{\PYZsq{}}\PY{l+s+s1}{r}\PY{l+s+s1}{\PYZsq{}}\PY{p}{,} \PY{n}{width} \PY{o}{=} \PY{l+m+mf}{1e\PYZhy{}5}\PY{p}{)}

    
\PY{n}{ax}\PY{o}{.}\PY{n}{scatter}\PY{p}{(}\PY{n}{bag2d}\PY{p}{[}\PY{p}{:}\PY{p}{,} \PY{n}{col1}\PY{p}{]}\PY{p}{,} \PY{n}{bag2d}\PY{p}{[}\PY{p}{:}\PY{p}{,} \PY{n}{col2}\PY{p}{]}\PY{p}{)}\PY{p}{;} \PY{c+c1}{\PYZsh{} Plot a dot for each word}

\PY{c+c1}{\PYZsh{} Add the word label over each dot in the scatter plot}
\PY{k}{for} \PY{n}{i} \PY{o+ow}{in} \PY{n+nb}{range}\PY{p}{(}\PY{l+m+mi}{0}\PY{p}{,} \PY{n+nb}{len}\PY{p}{(}\PY{n}{words}\PY{p}{)}\PY{p}{)}\PY{p}{:}
    \PY{n}{ax}\PY{o}{.}\PY{n}{annotate}\PY{p}{(}\PY{n}{words}\PY{p}{[}\PY{n}{i}\PY{p}{]}\PY{p}{,} \PY{p}{(}\PY{n}{bag2d}\PY{p}{[}\PY{n}{i}\PY{p}{,} \PY{n}{col1}\PY{p}{]}\PY{p}{,} \PY{n}{bag2d}\PY{p}{[}\PY{n}{i}\PY{p}{,} \PY{n}{col2}\PY{p}{]}\PY{p}{)}\PY{p}{)}


\PY{n}{plt}\PY{o}{.}\PY{n}{show}\PY{p}{(}\PY{p}{)}
\end{Verbatim}
\end{tcolorbox}

    \begin{center}
    \adjustimage{max size={0.9\linewidth}{0.9\paperheight}}{output_7_0.png}
    \end{center}
    { \hspace*{\fill} \\}
    
    Note that similar words like `village' and `town' or `petroleum', `oil',
and `gas' tend to point in the same direction. Also, note that `sad' and
`happy' looks close to each other; however, the vectors point in
opposite directions.

In this chart, one can figure out the angles and distances between the
words. Some words are close in both kinds of distance metrics.

    \hypertarget{word-distance}{%
\subsection{Word distance}\label{word-distance}}

Now plot the words `sad', `happy', `town', and `village'. In this same
chart, display the vector from `village' to `town' and the vector from
`sad' to `happy'. Let us use NumPy for these linear algebra operations.

    \begin{tcolorbox}[breakable, size=fbox, boxrule=1pt, pad at break*=1mm,colback=cellbackground, colframe=cellborder]
\prompt{In}{incolor}{5}{\boxspacing}
\begin{Verbatim}[commandchars=\\\{\}]
\PY{n}{words} \PY{o}{=} \PY{p}{[}\PY{l+s+s1}{\PYZsq{}}\PY{l+s+s1}{sad}\PY{l+s+s1}{\PYZsq{}}\PY{p}{,} \PY{l+s+s1}{\PYZsq{}}\PY{l+s+s1}{happy}\PY{l+s+s1}{\PYZsq{}}\PY{p}{,} \PY{l+s+s1}{\PYZsq{}}\PY{l+s+s1}{town}\PY{l+s+s1}{\PYZsq{}}\PY{p}{,} \PY{l+s+s1}{\PYZsq{}}\PY{l+s+s1}{village}\PY{l+s+s1}{\PYZsq{}}\PY{p}{]}

\PY{n}{bag2d} \PY{o}{=} \PY{n}{np}\PY{o}{.}\PY{n}{array}\PY{p}{(}\PY{p}{[}\PY{n}{vec}\PY{p}{(}\PY{n}{word}\PY{p}{)} \PY{k}{for} \PY{n}{word} \PY{o+ow}{in} \PY{n}{words}\PY{p}{]}\PY{p}{)} \PY{c+c1}{\PYZsh{} Convert each word to its vector representation}

\PY{n}{fig}\PY{p}{,} \PY{n}{ax} \PY{o}{=} \PY{n}{plt}\PY{o}{.}\PY{n}{subplots}\PY{p}{(}\PY{n}{figsize} \PY{o}{=} \PY{p}{(}\PY{l+m+mi}{10}\PY{p}{,} \PY{l+m+mi}{10}\PY{p}{)}\PY{p}{)} \PY{c+c1}{\PYZsh{} Create custom size image}

\PY{n}{col1} \PY{o}{=} \PY{l+m+mi}{3} \PY{c+c1}{\PYZsh{} Select the column for the x axe}
\PY{n}{col2} \PY{o}{=} \PY{l+m+mi}{2} \PY{c+c1}{\PYZsh{} Select the column for the y axe}

\PY{c+c1}{\PYZsh{} Print an arrow for each word}
\PY{k}{for} \PY{n}{word} \PY{o+ow}{in} \PY{n}{bag2d}\PY{p}{:}
    \PY{n}{ax}\PY{o}{.}\PY{n}{arrow}\PY{p}{(}\PY{l+m+mi}{0}\PY{p}{,} \PY{l+m+mi}{0}\PY{p}{,} \PY{n}{word}\PY{p}{[}\PY{n}{col1}\PY{p}{]}\PY{p}{,} \PY{n}{word}\PY{p}{[}\PY{n}{col2}\PY{p}{]}\PY{p}{,} \PY{n}{head\PYZus{}width}\PY{o}{=}\PY{l+m+mf}{0.0005}\PY{p}{,} \PY{n}{head\PYZus{}length}\PY{o}{=}\PY{l+m+mf}{0.0005}\PY{p}{,} \PY{n}{fc}\PY{o}{=}\PY{l+s+s1}{\PYZsq{}}\PY{l+s+s1}{r}\PY{l+s+s1}{\PYZsq{}}\PY{p}{,} \PY{n}{ec}\PY{o}{=}\PY{l+s+s1}{\PYZsq{}}\PY{l+s+s1}{r}\PY{l+s+s1}{\PYZsq{}}\PY{p}{,} \PY{n}{width} \PY{o}{=} \PY{l+m+mf}{1e\PYZhy{}5}\PY{p}{)}
    
\PY{c+c1}{\PYZsh{} print the vector difference between village and town}
\PY{n}{village} \PY{o}{=} \PY{n}{vec}\PY{p}{(}\PY{l+s+s1}{\PYZsq{}}\PY{l+s+s1}{village}\PY{l+s+s1}{\PYZsq{}}\PY{p}{)}
\PY{n}{town} \PY{o}{=} \PY{n}{vec}\PY{p}{(}\PY{l+s+s1}{\PYZsq{}}\PY{l+s+s1}{town}\PY{l+s+s1}{\PYZsq{}}\PY{p}{)}
\PY{n}{diff} \PY{o}{=} \PY{n}{town} \PY{o}{\PYZhy{}} \PY{n}{village}
\PY{n}{ax}\PY{o}{.}\PY{n}{arrow}\PY{p}{(}\PY{n}{village}\PY{p}{[}\PY{n}{col1}\PY{p}{]}\PY{p}{,} \PY{n}{village}\PY{p}{[}\PY{n}{col2}\PY{p}{]}\PY{p}{,} \PY{n}{diff}\PY{p}{[}\PY{n}{col1}\PY{p}{]}\PY{p}{,} \PY{n}{diff}\PY{p}{[}\PY{n}{col2}\PY{p}{]}\PY{p}{,} \PY{n}{fc}\PY{o}{=}\PY{l+s+s1}{\PYZsq{}}\PY{l+s+s1}{b}\PY{l+s+s1}{\PYZsq{}}\PY{p}{,} \PY{n}{ec}\PY{o}{=}\PY{l+s+s1}{\PYZsq{}}\PY{l+s+s1}{b}\PY{l+s+s1}{\PYZsq{}}\PY{p}{,} \PY{n}{width} \PY{o}{=} \PY{l+m+mf}{1e\PYZhy{}5}\PY{p}{)}

\PY{c+c1}{\PYZsh{} print the vector difference between village and town}
\PY{n}{sad} \PY{o}{=} \PY{n}{vec}\PY{p}{(}\PY{l+s+s1}{\PYZsq{}}\PY{l+s+s1}{sad}\PY{l+s+s1}{\PYZsq{}}\PY{p}{)}
\PY{n}{happy} \PY{o}{=} \PY{n}{vec}\PY{p}{(}\PY{l+s+s1}{\PYZsq{}}\PY{l+s+s1}{happy}\PY{l+s+s1}{\PYZsq{}}\PY{p}{)}
\PY{n}{diff} \PY{o}{=} \PY{n}{happy} \PY{o}{\PYZhy{}} \PY{n}{sad}
\PY{n}{ax}\PY{o}{.}\PY{n}{arrow}\PY{p}{(}\PY{n}{sad}\PY{p}{[}\PY{n}{col1}\PY{p}{]}\PY{p}{,} \PY{n}{sad}\PY{p}{[}\PY{n}{col2}\PY{p}{]}\PY{p}{,} \PY{n}{diff}\PY{p}{[}\PY{n}{col1}\PY{p}{]}\PY{p}{,} \PY{n}{diff}\PY{p}{[}\PY{n}{col2}\PY{p}{]}\PY{p}{,} \PY{n}{fc}\PY{o}{=}\PY{l+s+s1}{\PYZsq{}}\PY{l+s+s1}{b}\PY{l+s+s1}{\PYZsq{}}\PY{p}{,} \PY{n}{ec}\PY{o}{=}\PY{l+s+s1}{\PYZsq{}}\PY{l+s+s1}{b}\PY{l+s+s1}{\PYZsq{}}\PY{p}{,} \PY{n}{width} \PY{o}{=} \PY{l+m+mf}{1e\PYZhy{}5}\PY{p}{)}


\PY{n}{ax}\PY{o}{.}\PY{n}{scatter}\PY{p}{(}\PY{n}{bag2d}\PY{p}{[}\PY{p}{:}\PY{p}{,} \PY{n}{col1}\PY{p}{]}\PY{p}{,} \PY{n}{bag2d}\PY{p}{[}\PY{p}{:}\PY{p}{,} \PY{n}{col2}\PY{p}{]}\PY{p}{)}\PY{p}{;} \PY{c+c1}{\PYZsh{} Plot a dot for each word}

\PY{c+c1}{\PYZsh{} Add the word label over each dot in the scatter plot}
\PY{k}{for} \PY{n}{i} \PY{o+ow}{in} \PY{n+nb}{range}\PY{p}{(}\PY{l+m+mi}{0}\PY{p}{,} \PY{n+nb}{len}\PY{p}{(}\PY{n}{words}\PY{p}{)}\PY{p}{)}\PY{p}{:}
    \PY{n}{ax}\PY{o}{.}\PY{n}{annotate}\PY{p}{(}\PY{n}{words}\PY{p}{[}\PY{n}{i}\PY{p}{]}\PY{p}{,} \PY{p}{(}\PY{n}{bag2d}\PY{p}{[}\PY{n}{i}\PY{p}{,} \PY{n}{col1}\PY{p}{]}\PY{p}{,} \PY{n}{bag2d}\PY{p}{[}\PY{n}{i}\PY{p}{,} \PY{n}{col2}\PY{p}{]}\PY{p}{)}\PY{p}{)}


\PY{n}{plt}\PY{o}{.}\PY{n}{show}\PY{p}{(}\PY{p}{)}
\end{Verbatim}
\end{tcolorbox}

    \begin{center}
    \adjustimage{max size={0.9\linewidth}{0.9\paperheight}}{output_10_0.png}
    \end{center}
    { \hspace*{\fill} \\}
    
    \hypertarget{linear-algebra-on-word-embeddings}{%
\subsection{Linear algebra on word
embeddings}\label{linear-algebra-on-word-embeddings}}

In the lectures, we saw the analogies between words using algebra on
word embeddings. Let us see how to do it in Python with Numpy.

To start, get the \textbf{norm} of a word in the word embedding.

    \begin{tcolorbox}[breakable, size=fbox, boxrule=1pt, pad at break*=1mm,colback=cellbackground, colframe=cellborder]
\prompt{In}{incolor}{6}{\boxspacing}
\begin{Verbatim}[commandchars=\\\{\}]
\PY{n+nb}{print}\PY{p}{(}\PY{n}{np}\PY{o}{.}\PY{n}{linalg}\PY{o}{.}\PY{n}{norm}\PY{p}{(}\PY{n}{vec}\PY{p}{(}\PY{l+s+s1}{\PYZsq{}}\PY{l+s+s1}{town}\PY{l+s+s1}{\PYZsq{}}\PY{p}{)}\PY{p}{)}\PY{p}{)} \PY{c+c1}{\PYZsh{} Print the norm of the word town}
\PY{n+nb}{print}\PY{p}{(}\PY{n}{np}\PY{o}{.}\PY{n}{linalg}\PY{o}{.}\PY{n}{norm}\PY{p}{(}\PY{n}{vec}\PY{p}{(}\PY{l+s+s1}{\PYZsq{}}\PY{l+s+s1}{sad}\PY{l+s+s1}{\PYZsq{}}\PY{p}{)}\PY{p}{)}\PY{p}{)} \PY{c+c1}{\PYZsh{} Print the norm of the word sad}
\end{Verbatim}
\end{tcolorbox}

    \begin{Verbatim}[commandchars=\\\{\}]
2.3858097
2.9004838
    \end{Verbatim}

    \hypertarget{predicting-capitals}{%
\subsection{Predicting capitals}\label{predicting-capitals}}

Now, applying vector difference and addition, one can create a vector
representation for a new word. For example, we can say that the vector
difference between `France' and `Paris' represents the concept of
Capital.

One can move from the city of Madrid in the direction of the concept of
Capital, and obtain something close to the corresponding country to
which Madrid is the Capital.

    \begin{tcolorbox}[breakable, size=fbox, boxrule=1pt, pad at break*=1mm,colback=cellbackground, colframe=cellborder]
\prompt{In}{incolor}{7}{\boxspacing}
\begin{Verbatim}[commandchars=\\\{\}]
\PY{n}{capital} \PY{o}{=} \PY{n}{vec}\PY{p}{(}\PY{l+s+s1}{\PYZsq{}}\PY{l+s+s1}{France}\PY{l+s+s1}{\PYZsq{}}\PY{p}{)} \PY{o}{\PYZhy{}} \PY{n}{vec}\PY{p}{(}\PY{l+s+s1}{\PYZsq{}}\PY{l+s+s1}{Paris}\PY{l+s+s1}{\PYZsq{}}\PY{p}{)}
\PY{n}{country} \PY{o}{=} \PY{n}{vec}\PY{p}{(}\PY{l+s+s1}{\PYZsq{}}\PY{l+s+s1}{Madrid}\PY{l+s+s1}{\PYZsq{}}\PY{p}{)} \PY{o}{+} \PY{n}{capital}

\PY{n+nb}{print}\PY{p}{(}\PY{n}{country}\PY{p}{[}\PY{l+m+mi}{0}\PY{p}{:}\PY{l+m+mi}{5}\PY{p}{]}\PY{p}{)} \PY{c+c1}{\PYZsh{} Print the first 5 values of the vector}
\end{Verbatim}
\end{tcolorbox}

    \begin{Verbatim}[commandchars=\\\{\}]
[-0.02905273 -0.2475586   0.53952026  0.20581055 -0.14862823]
    \end{Verbatim}

    We can observe that the vector `country' that we expected to be the same
as the vector for Spain is not exactly it.

    \begin{tcolorbox}[breakable, size=fbox, boxrule=1pt, pad at break*=1mm,colback=cellbackground, colframe=cellborder]
\prompt{In}{incolor}{8}{\boxspacing}
\begin{Verbatim}[commandchars=\\\{\}]
\PY{n}{diff} \PY{o}{=} \PY{n}{country} \PY{o}{\PYZhy{}} \PY{n}{vec}\PY{p}{(}\PY{l+s+s1}{\PYZsq{}}\PY{l+s+s1}{Spain}\PY{l+s+s1}{\PYZsq{}}\PY{p}{)}
\PY{n+nb}{print}\PY{p}{(}\PY{n}{diff}\PY{p}{[}\PY{l+m+mi}{0}\PY{p}{:}\PY{l+m+mi}{10}\PY{p}{]}\PY{p}{)}
\end{Verbatim}
\end{tcolorbox}

    \begin{Verbatim}[commandchars=\\\{\}]
[-0.06054688 -0.06494141  0.37643433  0.08129883 -0.13007355 -0.00952148
 -0.03417969 -0.00708008  0.09790039 -0.01867676]
    \end{Verbatim}

    So, we have to look for the closest words in the embedding that matches
the candidate country. If the word embedding works as expected, the most
similar word must be `Spain'. Let us define a function that helps us to
do it. We will store our word embedding as a DataFrame, which facilitate
the lookup operations based on the numerical vectors.

    \begin{tcolorbox}[breakable, size=fbox, boxrule=1pt, pad at break*=1mm,colback=cellbackground, colframe=cellborder]
\prompt{In}{incolor}{9}{\boxspacing}
\begin{Verbatim}[commandchars=\\\{\}]
\PY{c+c1}{\PYZsh{} Create a dataframe out of the dictionary embedding. This facilitate the algebraic operations}
\PY{n}{keys} \PY{o}{=} \PY{n}{word\PYZus{}embeddings}\PY{o}{.}\PY{n}{keys}\PY{p}{(}\PY{p}{)}
\PY{n}{data} \PY{o}{=} \PY{p}{[}\PY{p}{]}
\PY{k}{for} \PY{n}{key} \PY{o+ow}{in} \PY{n}{keys}\PY{p}{:}
    \PY{n}{data}\PY{o}{.}\PY{n}{append}\PY{p}{(}\PY{n}{word\PYZus{}embeddings}\PY{p}{[}\PY{n}{key}\PY{p}{]}\PY{p}{)}

\PY{n}{embedding} \PY{o}{=} \PY{n}{pd}\PY{o}{.}\PY{n}{DataFrame}\PY{p}{(}\PY{n}{data}\PY{o}{=}\PY{n}{data}\PY{p}{,} \PY{n}{index}\PY{o}{=}\PY{n}{keys}\PY{p}{)}
\PY{c+c1}{\PYZsh{} Define a function to find the closest word to a vector:}
\PY{k}{def} \PY{n+nf}{find\PYZus{}closest\PYZus{}word}\PY{p}{(}\PY{n}{v}\PY{p}{,} \PY{n}{k} \PY{o}{=} \PY{l+m+mi}{1}\PY{p}{)}\PY{p}{:}
    \PY{c+c1}{\PYZsh{} Calculate the vector difference from each word to the input vector}
    \PY{n}{diff} \PY{o}{=} \PY{n}{embedding}\PY{o}{.}\PY{n}{values} \PY{o}{\PYZhy{}} \PY{n}{v} 
    \PY{c+c1}{\PYZsh{} Get the squared L2 norm of each difference vector.}
    \PY{c+c1}{\PYZsh{} It means the squared euclidean distance from each word to the input vector}
    \PY{n}{delta} \PY{o}{=} \PY{n}{np}\PY{o}{.}\PY{n}{sum}\PY{p}{(}\PY{n}{diff} \PY{o}{*} \PY{n}{diff}\PY{p}{,} \PY{n}{axis}\PY{o}{=}\PY{l+m+mi}{1}\PY{p}{)}
    \PY{c+c1}{\PYZsh{} Find the index of the minimun distance in the array}
    \PY{n}{i} \PY{o}{=} \PY{n}{np}\PY{o}{.}\PY{n}{argmin}\PY{p}{(}\PY{n}{delta}\PY{p}{)}
    \PY{c+c1}{\PYZsh{} Return the row name for this item}
    \PY{k}{return} \PY{n}{embedding}\PY{o}{.}\PY{n}{iloc}\PY{p}{[}\PY{n}{i}\PY{p}{]}\PY{o}{.}\PY{n}{name}
\end{Verbatim}
\end{tcolorbox}

    \begin{tcolorbox}[breakable, size=fbox, boxrule=1pt, pad at break*=1mm,colback=cellbackground, colframe=cellborder]
\prompt{In}{incolor}{10}{\boxspacing}
\begin{Verbatim}[commandchars=\\\{\}]
\PY{c+c1}{\PYZsh{} Print some rows of the embedding as a Dataframe}
\PY{n}{embedding}\PY{o}{.}\PY{n}{head}\PY{p}{(}\PY{l+m+mi}{10}\PY{p}{)}
\end{Verbatim}
\end{tcolorbox}

            \begin{tcolorbox}[breakable, size=fbox, boxrule=.5pt, pad at break*=1mm, opacityfill=0]
\prompt{Out}{outcolor}{10}{\boxspacing}
\begin{Verbatim}[commandchars=\\\{\}]
                0         1         2         3         4         5    \textbackslash{}
country   -0.080078  0.133789  0.143555  0.094727 -0.047363 -0.023560
city      -0.010071  0.057373  0.183594 -0.040039 -0.029785 -0.079102
China     -0.073242  0.135742  0.108887  0.083008 -0.127930 -0.227539
Iraq       0.191406  0.125000 -0.065430  0.060059 -0.285156 -0.102539
oil       -0.139648  0.062256 -0.279297  0.063965  0.044434 -0.154297
town       0.123535  0.159180  0.030029 -0.161133  0.015625  0.111816
Canada    -0.136719 -0.154297  0.269531  0.273438  0.086914 -0.076172
London    -0.267578  0.092773 -0.238281  0.115234 -0.006836  0.221680
England   -0.198242  0.115234  0.062500 -0.058350  0.226562  0.045898
Australia  0.048828 -0.194336 -0.041504  0.084473 -0.114258 -0.208008

                6         7         8         9    {\ldots}       290       291  \textbackslash{}
country   -0.008545 -0.186523  0.045898 -0.081543  {\ldots} -0.145508  0.067383
city       0.071777  0.013306 -0.143555  0.011292  {\ldots}  0.024292 -0.168945
China      0.151367 -0.045654 -0.065430  0.034424  {\ldots}  0.140625  0.087402
Iraq       0.117188 -0.351562 -0.095215  0.200195  {\ldots} -0.100586 -0.077148
oil       -0.184570 -0.498047  0.047363  0.110840  {\ldots} -0.195312 -0.345703
town       0.039795 -0.196289 -0.039307  0.067871  {\ldots} -0.007935 -0.091797
Canada    -0.018677  0.006256  0.077637 -0.211914  {\ldots}  0.105469  0.030762
London    -0.251953 -0.055420  0.020020  0.149414  {\ldots} -0.008667 -0.008484
England   -0.062256 -0.202148  0.080566  0.021606  {\ldots}  0.135742  0.109375
Australia -0.164062 -0.269531  0.079102  0.275391  {\ldots}  0.021118  0.171875

                292       293       294       295       296       297  \textbackslash{}
country   -0.244141 -0.077148  0.047607 -0.075195 -0.149414 -0.044189
city      -0.062988  0.117188 -0.020508  0.030273 -0.247070 -0.122559
China      0.152344  0.079590  0.006348 -0.037842 -0.183594  0.137695
Iraq      -0.123047  0.193359 -0.153320  0.089355 -0.173828 -0.054688
oil        0.217773 -0.091797  0.051025  0.061279  0.194336  0.204102
town      -0.265625  0.029297  0.089844 -0.049805 -0.202148 -0.079590
Canada    -0.039307  0.183594 -0.117676  0.191406  0.074219  0.020996
London    -0.053223  0.197266 -0.296875  0.064453  0.091797  0.058350
England   -0.121582  0.008545 -0.171875  0.086914  0.070312  0.003281
Australia  0.042236  0.221680 -0.239258 -0.106934  0.030884  0.006622

                298       299
country    0.097168  0.067383
city       0.076172 -0.234375
China      0.093750 -0.079590
Iraq       0.302734  0.105957
oil        0.235352 -0.051025
town       0.068848 -0.164062
Canada     0.285156 -0.257812
London     0.022583 -0.101074
England    0.069336  0.056152
Australia  0.051270 -0.135742

[10 rows x 300 columns]
\end{Verbatim}
\end{tcolorbox}
        
    Now let us find the name that corresponds to our numerical country:

    \begin{tcolorbox}[breakable, size=fbox, boxrule=1pt, pad at break*=1mm,colback=cellbackground, colframe=cellborder]
\prompt{In}{incolor}{11}{\boxspacing}
\begin{Verbatim}[commandchars=\\\{\}]
\PY{n}{find\PYZus{}closest\PYZus{}word}\PY{p}{(}\PY{n}{country}\PY{p}{)}
\end{Verbatim}
\end{tcolorbox}

            \begin{tcolorbox}[breakable, size=fbox, boxrule=.5pt, pad at break*=1mm, opacityfill=0]
\prompt{Out}{outcolor}{11}{\boxspacing}
\begin{Verbatim}[commandchars=\\\{\}]
'Spain'
\end{Verbatim}
\end{tcolorbox}
        
    \hypertarget{predicting-other-countries}{%
\subsection{Predicting other
Countries}\label{predicting-other-countries}}

    \begin{tcolorbox}[breakable, size=fbox, boxrule=1pt, pad at break*=1mm,colback=cellbackground, colframe=cellborder]
\prompt{In}{incolor}{12}{\boxspacing}
\begin{Verbatim}[commandchars=\\\{\}]
\PY{n}{find\PYZus{}closest\PYZus{}word}\PY{p}{(}\PY{n}{vec}\PY{p}{(}\PY{l+s+s1}{\PYZsq{}}\PY{l+s+s1}{Italy}\PY{l+s+s1}{\PYZsq{}}\PY{p}{)} \PY{o}{\PYZhy{}} \PY{n}{vec}\PY{p}{(}\PY{l+s+s1}{\PYZsq{}}\PY{l+s+s1}{Rome}\PY{l+s+s1}{\PYZsq{}}\PY{p}{)} \PY{o}{+} \PY{n}{vec}\PY{p}{(}\PY{l+s+s1}{\PYZsq{}}\PY{l+s+s1}{Madrid}\PY{l+s+s1}{\PYZsq{}}\PY{p}{)}\PY{p}{)}
\end{Verbatim}
\end{tcolorbox}

            \begin{tcolorbox}[breakable, size=fbox, boxrule=.5pt, pad at break*=1mm, opacityfill=0]
\prompt{Out}{outcolor}{12}{\boxspacing}
\begin{Verbatim}[commandchars=\\\{\}]
'Spain'
\end{Verbatim}
\end{tcolorbox}
        
    \begin{tcolorbox}[breakable, size=fbox, boxrule=1pt, pad at break*=1mm,colback=cellbackground, colframe=cellborder]
\prompt{In}{incolor}{13}{\boxspacing}
\begin{Verbatim}[commandchars=\\\{\}]
\PY{n+nb}{print}\PY{p}{(}\PY{n}{find\PYZus{}closest\PYZus{}word}\PY{p}{(}\PY{n}{vec}\PY{p}{(}\PY{l+s+s1}{\PYZsq{}}\PY{l+s+s1}{Berlin}\PY{l+s+s1}{\PYZsq{}}\PY{p}{)} \PY{o}{+} \PY{n}{capital}\PY{p}{)}\PY{p}{)}
\PY{n+nb}{print}\PY{p}{(}\PY{n}{find\PYZus{}closest\PYZus{}word}\PY{p}{(}\PY{n}{vec}\PY{p}{(}\PY{l+s+s1}{\PYZsq{}}\PY{l+s+s1}{Beijing}\PY{l+s+s1}{\PYZsq{}}\PY{p}{)} \PY{o}{+} \PY{n}{capital}\PY{p}{)}\PY{p}{)}
\end{Verbatim}
\end{tcolorbox}

    \begin{Verbatim}[commandchars=\\\{\}]
Germany
China
    \end{Verbatim}

    However, it does not always work.

    \begin{tcolorbox}[breakable, size=fbox, boxrule=1pt, pad at break*=1mm,colback=cellbackground, colframe=cellborder]
\prompt{In}{incolor}{14}{\boxspacing}
\begin{Verbatim}[commandchars=\\\{\}]
\PY{n+nb}{print}\PY{p}{(}\PY{n}{find\PYZus{}closest\PYZus{}word}\PY{p}{(}\PY{n}{vec}\PY{p}{(}\PY{l+s+s1}{\PYZsq{}}\PY{l+s+s1}{Lisbon}\PY{l+s+s1}{\PYZsq{}}\PY{p}{)} \PY{o}{+} \PY{n}{capital}\PY{p}{)}\PY{p}{)}
\end{Verbatim}
\end{tcolorbox}

    \begin{Verbatim}[commandchars=\\\{\}]
Lisbon
    \end{Verbatim}

    \hypertarget{represent-a-sentence-as-a-vector}{%
\subsection{Represent a sentence as a
vector}\label{represent-a-sentence-as-a-vector}}

A whole sentence can be represented as a vector by summing all the word
vectors that conform to the sentence. Let us see.

    \begin{tcolorbox}[breakable, size=fbox, boxrule=1pt, pad at break*=1mm,colback=cellbackground, colframe=cellborder]
\prompt{In}{incolor}{15}{\boxspacing}
\begin{Verbatim}[commandchars=\\\{\}]
\PY{n}{doc} \PY{o}{=} \PY{l+s+s2}{\PYZdq{}}\PY{l+s+s2}{Spain petroleum city king}\PY{l+s+s2}{\PYZdq{}}
\PY{n}{vdoc} \PY{o}{=} \PY{p}{[}\PY{n}{vec}\PY{p}{(}\PY{n}{x}\PY{p}{)} \PY{k}{for} \PY{n}{x} \PY{o+ow}{in} \PY{n}{doc}\PY{o}{.}\PY{n}{split}\PY{p}{(}\PY{l+s+s2}{\PYZdq{}}\PY{l+s+s2}{ }\PY{l+s+s2}{\PYZdq{}}\PY{p}{)}\PY{p}{]}
\PY{n}{doc2vec} \PY{o}{=} \PY{n}{np}\PY{o}{.}\PY{n}{sum}\PY{p}{(}\PY{n}{vdoc}\PY{p}{,} \PY{n}{axis} \PY{o}{=} \PY{l+m+mi}{0}\PY{p}{)}
\PY{n}{doc2vec}
\end{Verbatim}
\end{tcolorbox}

            \begin{tcolorbox}[breakable, size=fbox, boxrule=.5pt, pad at break*=1mm, opacityfill=0]
\prompt{Out}{outcolor}{15}{\boxspacing}
\begin{Verbatim}[commandchars=\\\{\}]
array([ 2.87475586e-02,  1.03759766e-01,  1.32629395e-01,  3.33007812e-01,
       -2.61230469e-02, -5.95703125e-01, -1.25976562e-01, -1.01306152e+00,
       -2.18544006e-01,  6.60705566e-01, -2.58300781e-01, -2.09960938e-02,
       -7.71484375e-02, -3.07128906e-01, -5.94726562e-01,  2.00561523e-01,
       -1.04980469e-02, -1.10748291e-01,  4.82177734e-02,  6.38977051e-01,
        2.36083984e-01, -2.69775391e-01,  3.90625000e-02,  4.16503906e-01,
        2.83416748e-01, -7.25097656e-02, -3.12988281e-01,  1.05712891e-01,
        3.22265625e-02,  2.38403320e-01,  3.88183594e-01, -7.51953125e-02,
       -1.26281738e-01,  6.60644531e-01, -7.89794922e-01, -7.04345703e-02,
       -1.14379883e-01, -4.78515625e-02,  4.76318359e-01,  5.31127930e-01,
        8.10546875e-02, -1.17553711e-01,  1.02050781e+00,  5.59814453e-01,
       -1.17187500e-01,  1.21826172e-01, -5.51574707e-01,  1.44531250e-01,
       -7.66113281e-01,  5.36102295e-01, -2.80029297e-01,  3.85986328e-01,
       -2.39135742e-01, -2.86865234e-02, -5.10498047e-01,  2.59658813e-01,
       -7.52929688e-01,  4.32128906e-02, -7.17773438e-02, -1.26708984e-01,
        4.40673828e-02,  5.12939453e-01, -5.15808105e-01,  1.20117188e-01,
       -5.52978516e-02, -3.92089844e-01, -3.15917969e-01,  1.57226562e-01,
       -3.19702148e-01,  1.75170898e-01, -3.81835938e-01, -2.07031250e-01,
       -4.72717285e-02, -2.79296875e-01, -3.29040527e-01, -1.69067383e-01,
        1.61132812e-02,  1.71569824e-01,  5.73730469e-02, -2.44140625e-03,
        8.34960938e-02, -1.58203125e-01, -3.10119629e-01,  5.28564453e-02,
        8.60595703e-02,  5.12695312e-02, -7.22900391e-01,  4.97924805e-01,
       -5.85937500e-03,  4.49951172e-01,  3.82446289e-01, -2.80029297e-01,
       -3.28125000e-01, -6.27441406e-02, -4.81933594e-01,  1.93176270e-02,
       -1.69326782e-01, -4.28649902e-01,  5.39062500e-01, -1.28417969e-01,
       -8.83789062e-02,  5.13916016e-01,  9.13085938e-02, -1.60156250e-01,
        6.86035156e-02, -9.74121094e-02, -3.70712280e-01, -3.27270508e-01,
        1.77978516e-01, -4.65332031e-01,  1.70410156e-01,  9.08203125e-02,
        2.76857376e-01, -1.69677734e-01,  3.27728271e-01, -3.12500000e-02,
       -2.20809937e-01, -3.46679688e-01,  4.67407227e-01,  5.31860352e-01,
       -1.30615234e-01, -2.36816406e-02, -6.56250000e-01, -5.79589844e-01,
       -2.05810547e-01, -3.03222656e-01,  1.94259644e-01, -7.28515625e-01,
       -4.92522240e-01, -5.37109375e-01, -3.47656250e-01,  1.08642578e-01,
       -1.41601562e-01, -2.07031250e-01,  2.52441406e-01, -7.78808594e-02,
       -5.02441406e-01,  1.53808594e-02,  8.64257812e-02,  2.59765625e-01,
        6.64062500e-02, -7.12890625e-01, -1.45751953e-01,  7.56835938e-03,
        4.87792969e-01,  1.39160156e-01,  1.15722656e-01,  1.28662109e-01,
       -4.75585938e-01,  2.21191406e-01,  3.25317383e-01,  1.06323242e-01,
       -6.11083984e-01, -3.59619141e-01,  6.54296875e-02, -2.41699219e-01,
       -6.29882812e-02, -1.62109375e-01,  4.26269531e-01, -4.38354492e-01,
        1.93725586e-01,  4.89562988e-01,  5.31494141e-01, -7.29370117e-02,
        1.77246094e-01,  9.39941406e-02,  2.92236328e-01, -2.74047852e-01,
        2.63366699e-02,  4.36035156e-01, -3.76953125e-01,  3.10546875e-01,
        4.87304688e-01, -2.43041992e-01,  1.21612549e-02, -3.80371094e-01,
        3.80493164e-01, -6.22436523e-01, -3.98071289e-01,  1.24206543e-01,
       -8.20312500e-01, -2.72583008e-01, -6.21582031e-01, -4.87060547e-01,
        3.06671143e-01, -2.61230469e-01,  5.12451172e-01,  5.55694580e-01,
        5.66894531e-01,  7.33886719e-01, -1.75781250e-01,  4.13574219e-01,
       -2.54272461e-01,  1.32507324e-01, -4.78515625e-01,  4.63256836e-01,
       -6.21948242e-02, -1.80664062e-01, -5.46386719e-01, -6.31103516e-01,
       -1.47949219e-01, -3.15185547e-01, -7.12890625e-02, -7.67578125e-01,
        3.92272949e-01, -1.97753906e-01,  2.23144531e-01, -5.07324219e-01,
        8.39843750e-02, -4.98657227e-02,  1.01074219e-01,  2.07885742e-01,
       -2.77343750e-01,  1.03027344e-01, -1.38671875e-01,  2.87353516e-01,
       -4.81895447e-01, -1.66748047e-01, -1.47277832e-01,  3.61633301e-01,
        6.38504028e-02, -6.69189453e-01,  1.95312500e-03, -7.34375000e-01,
       -1.28158569e-01,  9.76562500e-04, -7.08007812e-02,  3.72558594e-01,
        8.31176758e-01,  5.94482422e-01,  5.37109375e-02, -3.00140381e-01,
       -4.53857422e-01,  1.11511230e-01, -1.32812500e-01,  1.25732422e-01,
        3.39843750e-01, -2.48352051e-01, -1.62353516e-02, -2.84667969e-01,
        4.70703125e-01, -4.48242188e-01,  8.50753784e-02,  2.69042969e-01,
        3.98254395e-03, -3.53759766e-01, -3.90625000e-02, -3.22753906e-01,
       -6.90917969e-02, -4.13818359e-02,  1.35314941e-01, -8.50396156e-02,
        1.28417969e-01,  6.15966797e-01,  3.55957031e-01, -6.05468750e-02,
       -2.25463867e-01, -2.62207031e-01, -2.72949219e-01, -5.16113281e-01,
        1.59179688e-01,  2.74902344e-01, -7.61718750e-02, -3.41796875e-03,
        4.37500000e-01,  2.98583984e-01, -4.40795898e-01, -3.43261719e-01,
        1.73583984e-01,  3.32092285e-01, -2.12646484e-01,  5.76171875e-01,
        2.06787109e-01, -7.91015625e-02,  5.79695702e-02, -1.01806641e-01,
       -7.06787109e-01, -3.40576172e-02, -4.11865234e-01,  9.82666016e-02,
       -1.70410156e-01, -4.18212891e-01,  8.39233398e-01, -1.15722656e-01,
        1.28173828e-01, -2.07763672e-01, -4.08203125e-01, -1.77612305e-01,
        1.01196289e-01,  4.24072266e-01, -5.26428223e-02, -5.58593750e-01,
        1.12304688e-02, -1.12060547e-01, -9.42382812e-02,  2.35595703e-02,
       -3.92578125e-01, -7.12890625e-02,  5.69824219e-01,  9.81445312e-02],
      dtype=float32)
\end{Verbatim}
\end{tcolorbox}
        
    \begin{tcolorbox}[breakable, size=fbox, boxrule=1pt, pad at break*=1mm,colback=cellbackground, colframe=cellborder]
\prompt{In}{incolor}{16}{\boxspacing}
\begin{Verbatim}[commandchars=\\\{\}]
\PY{n}{find\PYZus{}closest\PYZus{}word}\PY{p}{(}\PY{n}{doc2vec}\PY{p}{)}
\end{Verbatim}
\end{tcolorbox}

            \begin{tcolorbox}[breakable, size=fbox, boxrule=.5pt, pad at break*=1mm, opacityfill=0]
\prompt{Out}{outcolor}{16}{\boxspacing}
\begin{Verbatim}[commandchars=\\\{\}]
'petroleum'
\end{Verbatim}
\end{tcolorbox}
        
    \textbf{Congratulations! You have finished the introduction to word
embeddings manipulation!}


    % Add a bibliography block to the postdoc
    
    
    
\end{document}
