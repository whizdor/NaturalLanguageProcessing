\documentclass[11pt]{article}

    \usepackage[breakable]{tcolorbox}
    \usepackage{parskip} % Stop auto-indenting (to mimic markdown behaviour)
    
    \usepackage{iftex}
    \ifPDFTeX
    	\usepackage[T1]{fontenc}
    	\usepackage{mathpazo}
    \else
    	\usepackage{fontspec}
    \fi

    % Basic figure setup, for now with no caption control since it's done
    % automatically by Pandoc (which extracts ![](path) syntax from Markdown).
    \usepackage{graphicx}
    % Maintain compatibility with old templates. Remove in nbconvert 6.0
    \let\Oldincludegraphics\includegraphics
    % Ensure that by default, figures have no caption (until we provide a
    % proper Figure object with a Caption API and a way to capture that
    % in the conversion process - todo).
    \usepackage{caption}
    \DeclareCaptionFormat{nocaption}{}
    \captionsetup{format=nocaption,aboveskip=0pt,belowskip=0pt}

    \usepackage[Export]{adjustbox} % Used to constrain images to a maximum size
    \adjustboxset{max size={0.9\linewidth}{0.9\paperheight}}
    \usepackage{float}
    \floatplacement{figure}{H} % forces figures to be placed at the correct location
    \usepackage{xcolor} % Allow colors to be defined
    \usepackage{enumerate} % Needed for markdown enumerations to work
    \usepackage{geometry} % Used to adjust the document margins
    \usepackage{amsmath} % Equations
    \usepackage{amssymb} % Equations
    \usepackage{textcomp} % defines textquotesingle
    % Hack from http://tex.stackexchange.com/a/47451/13684:
    \AtBeginDocument{%
        \def\PYZsq{\textquotesingle}% Upright quotes in Pygmentized code
    }
    \usepackage{upquote} % Upright quotes for verbatim code
    \usepackage{eurosym} % defines \euro
    \usepackage[mathletters]{ucs} % Extended unicode (utf-8) support
    \usepackage{fancyvrb} % verbatim replacement that allows latex
    \usepackage{grffile} % extends the file name processing of package graphics 
                         % to support a larger range
    \makeatletter % fix for grffile with XeLaTeX
    \def\Gread@@xetex#1{%
      \IfFileExists{"\Gin@base".bb}%
      {\Gread@eps{\Gin@base.bb}}%
      {\Gread@@xetex@aux#1}%
    }
    \makeatother

    % The hyperref package gives us a pdf with properly built
    % internal navigation ('pdf bookmarks' for the table of contents,
    % internal cross-reference links, web links for URLs, etc.)
    \usepackage{hyperref}
    % The default LaTeX title has an obnoxious amount of whitespace. By default,
    % titling removes some of it. It also provides customization options.
    \usepackage{titling}
    \usepackage{longtable} % longtable support required by pandoc >1.10
    \usepackage{booktabs}  % table support for pandoc > 1.12.2
    \usepackage[inline]{enumitem} % IRkernel/repr support (it uses the enumerate* environment)
    \usepackage[normalem]{ulem} % ulem is needed to support strikethroughs (\sout)
                                % normalem makes italics be italics, not underlines
    \usepackage{mathrsfs}
    

    
    % Colors for the hyperref package
    \definecolor{urlcolor}{rgb}{0,.145,.698}
    \definecolor{linkcolor}{rgb}{.71,0.21,0.01}
    \definecolor{citecolor}{rgb}{.12,.54,.11}

    % ANSI colors
    \definecolor{ansi-black}{HTML}{3E424D}
    \definecolor{ansi-black-intense}{HTML}{282C36}
    \definecolor{ansi-red}{HTML}{E75C58}
    \definecolor{ansi-red-intense}{HTML}{B22B31}
    \definecolor{ansi-green}{HTML}{00A250}
    \definecolor{ansi-green-intense}{HTML}{007427}
    \definecolor{ansi-yellow}{HTML}{DDB62B}
    \definecolor{ansi-yellow-intense}{HTML}{B27D12}
    \definecolor{ansi-blue}{HTML}{208FFB}
    \definecolor{ansi-blue-intense}{HTML}{0065CA}
    \definecolor{ansi-magenta}{HTML}{D160C4}
    \definecolor{ansi-magenta-intense}{HTML}{A03196}
    \definecolor{ansi-cyan}{HTML}{60C6C8}
    \definecolor{ansi-cyan-intense}{HTML}{258F8F}
    \definecolor{ansi-white}{HTML}{C5C1B4}
    \definecolor{ansi-white-intense}{HTML}{A1A6B2}
    \definecolor{ansi-default-inverse-fg}{HTML}{FFFFFF}
    \definecolor{ansi-default-inverse-bg}{HTML}{000000}

    % commands and environments needed by pandoc snippets
    % extracted from the output of `pandoc -s`
    \providecommand{\tightlist}{%
      \setlength{\itemsep}{0pt}\setlength{\parskip}{0pt}}
    \DefineVerbatimEnvironment{Highlighting}{Verbatim}{commandchars=\\\{\}}
    % Add ',fontsize=\small' for more characters per line
    \newenvironment{Shaded}{}{}
    \newcommand{\KeywordTok}[1]{\textcolor[rgb]{0.00,0.44,0.13}{\textbf{{#1}}}}
    \newcommand{\DataTypeTok}[1]{\textcolor[rgb]{0.56,0.13,0.00}{{#1}}}
    \newcommand{\DecValTok}[1]{\textcolor[rgb]{0.25,0.63,0.44}{{#1}}}
    \newcommand{\BaseNTok}[1]{\textcolor[rgb]{0.25,0.63,0.44}{{#1}}}
    \newcommand{\FloatTok}[1]{\textcolor[rgb]{0.25,0.63,0.44}{{#1}}}
    \newcommand{\CharTok}[1]{\textcolor[rgb]{0.25,0.44,0.63}{{#1}}}
    \newcommand{\StringTok}[1]{\textcolor[rgb]{0.25,0.44,0.63}{{#1}}}
    \newcommand{\CommentTok}[1]{\textcolor[rgb]{0.38,0.63,0.69}{\textit{{#1}}}}
    \newcommand{\OtherTok}[1]{\textcolor[rgb]{0.00,0.44,0.13}{{#1}}}
    \newcommand{\AlertTok}[1]{\textcolor[rgb]{1.00,0.00,0.00}{\textbf{{#1}}}}
    \newcommand{\FunctionTok}[1]{\textcolor[rgb]{0.02,0.16,0.49}{{#1}}}
    \newcommand{\RegionMarkerTok}[1]{{#1}}
    \newcommand{\ErrorTok}[1]{\textcolor[rgb]{1.00,0.00,0.00}{\textbf{{#1}}}}
    \newcommand{\NormalTok}[1]{{#1}}
    
    % Additional commands for more recent versions of Pandoc
    \newcommand{\ConstantTok}[1]{\textcolor[rgb]{0.53,0.00,0.00}{{#1}}}
    \newcommand{\SpecialCharTok}[1]{\textcolor[rgb]{0.25,0.44,0.63}{{#1}}}
    \newcommand{\VerbatimStringTok}[1]{\textcolor[rgb]{0.25,0.44,0.63}{{#1}}}
    \newcommand{\SpecialStringTok}[1]{\textcolor[rgb]{0.73,0.40,0.53}{{#1}}}
    \newcommand{\ImportTok}[1]{{#1}}
    \newcommand{\DocumentationTok}[1]{\textcolor[rgb]{0.73,0.13,0.13}{\textit{{#1}}}}
    \newcommand{\AnnotationTok}[1]{\textcolor[rgb]{0.38,0.63,0.69}{\textbf{\textit{{#1}}}}}
    \newcommand{\CommentVarTok}[1]{\textcolor[rgb]{0.38,0.63,0.69}{\textbf{\textit{{#1}}}}}
    \newcommand{\VariableTok}[1]{\textcolor[rgb]{0.10,0.09,0.49}{{#1}}}
    \newcommand{\ControlFlowTok}[1]{\textcolor[rgb]{0.00,0.44,0.13}{\textbf{{#1}}}}
    \newcommand{\OperatorTok}[1]{\textcolor[rgb]{0.40,0.40,0.40}{{#1}}}
    \newcommand{\BuiltInTok}[1]{{#1}}
    \newcommand{\ExtensionTok}[1]{{#1}}
    \newcommand{\PreprocessorTok}[1]{\textcolor[rgb]{0.74,0.48,0.00}{{#1}}}
    \newcommand{\AttributeTok}[1]{\textcolor[rgb]{0.49,0.56,0.16}{{#1}}}
    \newcommand{\InformationTok}[1]{\textcolor[rgb]{0.38,0.63,0.69}{\textbf{\textit{{#1}}}}}
    \newcommand{\WarningTok}[1]{\textcolor[rgb]{0.38,0.63,0.69}{\textbf{\textit{{#1}}}}}
    
    
    % Define a nice break command that doesn't care if a line doesn't already
    % exist.
    \def\br{\hspace*{\fill} \\* }
    % Math Jax compatibility definitions
    \def\gt{>}
    \def\lt{<}
    \let\Oldtex\TeX
    \let\Oldlatex\LaTeX
    \renewcommand{\TeX}{\textrm{\Oldtex}}
    \renewcommand{\LaTeX}{\textrm{\Oldlatex}}
    % Document parameters
    % Document title
    \title{C1\_W3\_Assignment}
    
    
    
    
    
% Pygments definitions
\makeatletter
\def\PY@reset{\let\PY@it=\relax \let\PY@bf=\relax%
    \let\PY@ul=\relax \let\PY@tc=\relax%
    \let\PY@bc=\relax \let\PY@ff=\relax}
\def\PY@tok#1{\csname PY@tok@#1\endcsname}
\def\PY@toks#1+{\ifx\relax#1\empty\else%
    \PY@tok{#1}\expandafter\PY@toks\fi}
\def\PY@do#1{\PY@bc{\PY@tc{\PY@ul{%
    \PY@it{\PY@bf{\PY@ff{#1}}}}}}}
\def\PY#1#2{\PY@reset\PY@toks#1+\relax+\PY@do{#2}}

\expandafter\def\csname PY@tok@w\endcsname{\def\PY@tc##1{\textcolor[rgb]{0.73,0.73,0.73}{##1}}}
\expandafter\def\csname PY@tok@c\endcsname{\let\PY@it=\textit\def\PY@tc##1{\textcolor[rgb]{0.25,0.50,0.50}{##1}}}
\expandafter\def\csname PY@tok@cp\endcsname{\def\PY@tc##1{\textcolor[rgb]{0.74,0.48,0.00}{##1}}}
\expandafter\def\csname PY@tok@k\endcsname{\let\PY@bf=\textbf\def\PY@tc##1{\textcolor[rgb]{0.00,0.50,0.00}{##1}}}
\expandafter\def\csname PY@tok@kp\endcsname{\def\PY@tc##1{\textcolor[rgb]{0.00,0.50,0.00}{##1}}}
\expandafter\def\csname PY@tok@kt\endcsname{\def\PY@tc##1{\textcolor[rgb]{0.69,0.00,0.25}{##1}}}
\expandafter\def\csname PY@tok@o\endcsname{\def\PY@tc##1{\textcolor[rgb]{0.40,0.40,0.40}{##1}}}
\expandafter\def\csname PY@tok@ow\endcsname{\let\PY@bf=\textbf\def\PY@tc##1{\textcolor[rgb]{0.67,0.13,1.00}{##1}}}
\expandafter\def\csname PY@tok@nb\endcsname{\def\PY@tc##1{\textcolor[rgb]{0.00,0.50,0.00}{##1}}}
\expandafter\def\csname PY@tok@nf\endcsname{\def\PY@tc##1{\textcolor[rgb]{0.00,0.00,1.00}{##1}}}
\expandafter\def\csname PY@tok@nc\endcsname{\let\PY@bf=\textbf\def\PY@tc##1{\textcolor[rgb]{0.00,0.00,1.00}{##1}}}
\expandafter\def\csname PY@tok@nn\endcsname{\let\PY@bf=\textbf\def\PY@tc##1{\textcolor[rgb]{0.00,0.00,1.00}{##1}}}
\expandafter\def\csname PY@tok@ne\endcsname{\let\PY@bf=\textbf\def\PY@tc##1{\textcolor[rgb]{0.82,0.25,0.23}{##1}}}
\expandafter\def\csname PY@tok@nv\endcsname{\def\PY@tc##1{\textcolor[rgb]{0.10,0.09,0.49}{##1}}}
\expandafter\def\csname PY@tok@no\endcsname{\def\PY@tc##1{\textcolor[rgb]{0.53,0.00,0.00}{##1}}}
\expandafter\def\csname PY@tok@nl\endcsname{\def\PY@tc##1{\textcolor[rgb]{0.63,0.63,0.00}{##1}}}
\expandafter\def\csname PY@tok@ni\endcsname{\let\PY@bf=\textbf\def\PY@tc##1{\textcolor[rgb]{0.60,0.60,0.60}{##1}}}
\expandafter\def\csname PY@tok@na\endcsname{\def\PY@tc##1{\textcolor[rgb]{0.49,0.56,0.16}{##1}}}
\expandafter\def\csname PY@tok@nt\endcsname{\let\PY@bf=\textbf\def\PY@tc##1{\textcolor[rgb]{0.00,0.50,0.00}{##1}}}
\expandafter\def\csname PY@tok@nd\endcsname{\def\PY@tc##1{\textcolor[rgb]{0.67,0.13,1.00}{##1}}}
\expandafter\def\csname PY@tok@s\endcsname{\def\PY@tc##1{\textcolor[rgb]{0.73,0.13,0.13}{##1}}}
\expandafter\def\csname PY@tok@sd\endcsname{\let\PY@it=\textit\def\PY@tc##1{\textcolor[rgb]{0.73,0.13,0.13}{##1}}}
\expandafter\def\csname PY@tok@si\endcsname{\let\PY@bf=\textbf\def\PY@tc##1{\textcolor[rgb]{0.73,0.40,0.53}{##1}}}
\expandafter\def\csname PY@tok@se\endcsname{\let\PY@bf=\textbf\def\PY@tc##1{\textcolor[rgb]{0.73,0.40,0.13}{##1}}}
\expandafter\def\csname PY@tok@sr\endcsname{\def\PY@tc##1{\textcolor[rgb]{0.73,0.40,0.53}{##1}}}
\expandafter\def\csname PY@tok@ss\endcsname{\def\PY@tc##1{\textcolor[rgb]{0.10,0.09,0.49}{##1}}}
\expandafter\def\csname PY@tok@sx\endcsname{\def\PY@tc##1{\textcolor[rgb]{0.00,0.50,0.00}{##1}}}
\expandafter\def\csname PY@tok@m\endcsname{\def\PY@tc##1{\textcolor[rgb]{0.40,0.40,0.40}{##1}}}
\expandafter\def\csname PY@tok@gh\endcsname{\let\PY@bf=\textbf\def\PY@tc##1{\textcolor[rgb]{0.00,0.00,0.50}{##1}}}
\expandafter\def\csname PY@tok@gu\endcsname{\let\PY@bf=\textbf\def\PY@tc##1{\textcolor[rgb]{0.50,0.00,0.50}{##1}}}
\expandafter\def\csname PY@tok@gd\endcsname{\def\PY@tc##1{\textcolor[rgb]{0.63,0.00,0.00}{##1}}}
\expandafter\def\csname PY@tok@gi\endcsname{\def\PY@tc##1{\textcolor[rgb]{0.00,0.63,0.00}{##1}}}
\expandafter\def\csname PY@tok@gr\endcsname{\def\PY@tc##1{\textcolor[rgb]{1.00,0.00,0.00}{##1}}}
\expandafter\def\csname PY@tok@ge\endcsname{\let\PY@it=\textit}
\expandafter\def\csname PY@tok@gs\endcsname{\let\PY@bf=\textbf}
\expandafter\def\csname PY@tok@gp\endcsname{\let\PY@bf=\textbf\def\PY@tc##1{\textcolor[rgb]{0.00,0.00,0.50}{##1}}}
\expandafter\def\csname PY@tok@go\endcsname{\def\PY@tc##1{\textcolor[rgb]{0.53,0.53,0.53}{##1}}}
\expandafter\def\csname PY@tok@gt\endcsname{\def\PY@tc##1{\textcolor[rgb]{0.00,0.27,0.87}{##1}}}
\expandafter\def\csname PY@tok@err\endcsname{\def\PY@bc##1{\setlength{\fboxsep}{0pt}\fcolorbox[rgb]{1.00,0.00,0.00}{1,1,1}{\strut ##1}}}
\expandafter\def\csname PY@tok@kc\endcsname{\let\PY@bf=\textbf\def\PY@tc##1{\textcolor[rgb]{0.00,0.50,0.00}{##1}}}
\expandafter\def\csname PY@tok@kd\endcsname{\let\PY@bf=\textbf\def\PY@tc##1{\textcolor[rgb]{0.00,0.50,0.00}{##1}}}
\expandafter\def\csname PY@tok@kn\endcsname{\let\PY@bf=\textbf\def\PY@tc##1{\textcolor[rgb]{0.00,0.50,0.00}{##1}}}
\expandafter\def\csname PY@tok@kr\endcsname{\let\PY@bf=\textbf\def\PY@tc##1{\textcolor[rgb]{0.00,0.50,0.00}{##1}}}
\expandafter\def\csname PY@tok@bp\endcsname{\def\PY@tc##1{\textcolor[rgb]{0.00,0.50,0.00}{##1}}}
\expandafter\def\csname PY@tok@fm\endcsname{\def\PY@tc##1{\textcolor[rgb]{0.00,0.00,1.00}{##1}}}
\expandafter\def\csname PY@tok@vc\endcsname{\def\PY@tc##1{\textcolor[rgb]{0.10,0.09,0.49}{##1}}}
\expandafter\def\csname PY@tok@vg\endcsname{\def\PY@tc##1{\textcolor[rgb]{0.10,0.09,0.49}{##1}}}
\expandafter\def\csname PY@tok@vi\endcsname{\def\PY@tc##1{\textcolor[rgb]{0.10,0.09,0.49}{##1}}}
\expandafter\def\csname PY@tok@vm\endcsname{\def\PY@tc##1{\textcolor[rgb]{0.10,0.09,0.49}{##1}}}
\expandafter\def\csname PY@tok@sa\endcsname{\def\PY@tc##1{\textcolor[rgb]{0.73,0.13,0.13}{##1}}}
\expandafter\def\csname PY@tok@sb\endcsname{\def\PY@tc##1{\textcolor[rgb]{0.73,0.13,0.13}{##1}}}
\expandafter\def\csname PY@tok@sc\endcsname{\def\PY@tc##1{\textcolor[rgb]{0.73,0.13,0.13}{##1}}}
\expandafter\def\csname PY@tok@dl\endcsname{\def\PY@tc##1{\textcolor[rgb]{0.73,0.13,0.13}{##1}}}
\expandafter\def\csname PY@tok@s2\endcsname{\def\PY@tc##1{\textcolor[rgb]{0.73,0.13,0.13}{##1}}}
\expandafter\def\csname PY@tok@sh\endcsname{\def\PY@tc##1{\textcolor[rgb]{0.73,0.13,0.13}{##1}}}
\expandafter\def\csname PY@tok@s1\endcsname{\def\PY@tc##1{\textcolor[rgb]{0.73,0.13,0.13}{##1}}}
\expandafter\def\csname PY@tok@mb\endcsname{\def\PY@tc##1{\textcolor[rgb]{0.40,0.40,0.40}{##1}}}
\expandafter\def\csname PY@tok@mf\endcsname{\def\PY@tc##1{\textcolor[rgb]{0.40,0.40,0.40}{##1}}}
\expandafter\def\csname PY@tok@mh\endcsname{\def\PY@tc##1{\textcolor[rgb]{0.40,0.40,0.40}{##1}}}
\expandafter\def\csname PY@tok@mi\endcsname{\def\PY@tc##1{\textcolor[rgb]{0.40,0.40,0.40}{##1}}}
\expandafter\def\csname PY@tok@il\endcsname{\def\PY@tc##1{\textcolor[rgb]{0.40,0.40,0.40}{##1}}}
\expandafter\def\csname PY@tok@mo\endcsname{\def\PY@tc##1{\textcolor[rgb]{0.40,0.40,0.40}{##1}}}
\expandafter\def\csname PY@tok@ch\endcsname{\let\PY@it=\textit\def\PY@tc##1{\textcolor[rgb]{0.25,0.50,0.50}{##1}}}
\expandafter\def\csname PY@tok@cm\endcsname{\let\PY@it=\textit\def\PY@tc##1{\textcolor[rgb]{0.25,0.50,0.50}{##1}}}
\expandafter\def\csname PY@tok@cpf\endcsname{\let\PY@it=\textit\def\PY@tc##1{\textcolor[rgb]{0.25,0.50,0.50}{##1}}}
\expandafter\def\csname PY@tok@c1\endcsname{\let\PY@it=\textit\def\PY@tc##1{\textcolor[rgb]{0.25,0.50,0.50}{##1}}}
\expandafter\def\csname PY@tok@cs\endcsname{\let\PY@it=\textit\def\PY@tc##1{\textcolor[rgb]{0.25,0.50,0.50}{##1}}}

\def\PYZbs{\char`\\}
\def\PYZus{\char`\_}
\def\PYZob{\char`\{}
\def\PYZcb{\char`\}}
\def\PYZca{\char`\^}
\def\PYZam{\char`\&}
\def\PYZlt{\char`\<}
\def\PYZgt{\char`\>}
\def\PYZsh{\char`\#}
\def\PYZpc{\char`\%}
\def\PYZdl{\char`\$}
\def\PYZhy{\char`\-}
\def\PYZsq{\char`\'}
\def\PYZdq{\char`\"}
\def\PYZti{\char`\~}
% for compatibility with earlier versions
\def\PYZat{@}
\def\PYZlb{[}
\def\PYZrb{]}
\makeatother


    % For linebreaks inside Verbatim environment from package fancyvrb. 
    \makeatletter
        \newbox\Wrappedcontinuationbox 
        \newbox\Wrappedvisiblespacebox 
        \newcommand*\Wrappedvisiblespace {\textcolor{red}{\textvisiblespace}} 
        \newcommand*\Wrappedcontinuationsymbol {\textcolor{red}{\llap{\tiny$\m@th\hookrightarrow$}}} 
        \newcommand*\Wrappedcontinuationindent {3ex } 
        \newcommand*\Wrappedafterbreak {\kern\Wrappedcontinuationindent\copy\Wrappedcontinuationbox} 
        % Take advantage of the already applied Pygments mark-up to insert 
        % potential linebreaks for TeX processing. 
        %        {, <, #, %, $, ' and ": go to next line. 
        %        _, }, ^, &, >, - and ~: stay at end of broken line. 
        % Use of \textquotesingle for straight quote. 
        \newcommand*\Wrappedbreaksatspecials {% 
            \def\PYGZus{\discretionary{\char`\_}{\Wrappedafterbreak}{\char`\_}}% 
            \def\PYGZob{\discretionary{}{\Wrappedafterbreak\char`\{}{\char`\{}}% 
            \def\PYGZcb{\discretionary{\char`\}}{\Wrappedafterbreak}{\char`\}}}% 
            \def\PYGZca{\discretionary{\char`\^}{\Wrappedafterbreak}{\char`\^}}% 
            \def\PYGZam{\discretionary{\char`\&}{\Wrappedafterbreak}{\char`\&}}% 
            \def\PYGZlt{\discretionary{}{\Wrappedafterbreak\char`\<}{\char`\<}}% 
            \def\PYGZgt{\discretionary{\char`\>}{\Wrappedafterbreak}{\char`\>}}% 
            \def\PYGZsh{\discretionary{}{\Wrappedafterbreak\char`\#}{\char`\#}}% 
            \def\PYGZpc{\discretionary{}{\Wrappedafterbreak\char`\%}{\char`\%}}% 
            \def\PYGZdl{\discretionary{}{\Wrappedafterbreak\char`\$}{\char`\$}}% 
            \def\PYGZhy{\discretionary{\char`\-}{\Wrappedafterbreak}{\char`\-}}% 
            \def\PYGZsq{\discretionary{}{\Wrappedafterbreak\textquotesingle}{\textquotesingle}}% 
            \def\PYGZdq{\discretionary{}{\Wrappedafterbreak\char`\"}{\char`\"}}% 
            \def\PYGZti{\discretionary{\char`\~}{\Wrappedafterbreak}{\char`\~}}% 
        } 
        % Some characters . , ; ? ! / are not pygmentized. 
        % This macro makes them "active" and they will insert potential linebreaks 
        \newcommand*\Wrappedbreaksatpunct {% 
            \lccode`\~`\.\lowercase{\def~}{\discretionary{\hbox{\char`\.}}{\Wrappedafterbreak}{\hbox{\char`\.}}}% 
            \lccode`\~`\,\lowercase{\def~}{\discretionary{\hbox{\char`\,}}{\Wrappedafterbreak}{\hbox{\char`\,}}}% 
            \lccode`\~`\;\lowercase{\def~}{\discretionary{\hbox{\char`\;}}{\Wrappedafterbreak}{\hbox{\char`\;}}}% 
            \lccode`\~`\:\lowercase{\def~}{\discretionary{\hbox{\char`\:}}{\Wrappedafterbreak}{\hbox{\char`\:}}}% 
            \lccode`\~`\?\lowercase{\def~}{\discretionary{\hbox{\char`\?}}{\Wrappedafterbreak}{\hbox{\char`\?}}}% 
            \lccode`\~`\!\lowercase{\def~}{\discretionary{\hbox{\char`\!}}{\Wrappedafterbreak}{\hbox{\char`\!}}}% 
            \lccode`\~`\/\lowercase{\def~}{\discretionary{\hbox{\char`\/}}{\Wrappedafterbreak}{\hbox{\char`\/}}}% 
            \catcode`\.\active
            \catcode`\,\active 
            \catcode`\;\active
            \catcode`\:\active
            \catcode`\?\active
            \catcode`\!\active
            \catcode`\/\active 
            \lccode`\~`\~ 	
        }
    \makeatother

    \let\OriginalVerbatim=\Verbatim
    \makeatletter
    \renewcommand{\Verbatim}[1][1]{%
        %\parskip\z@skip
        \sbox\Wrappedcontinuationbox {\Wrappedcontinuationsymbol}%
        \sbox\Wrappedvisiblespacebox {\FV@SetupFont\Wrappedvisiblespace}%
        \def\FancyVerbFormatLine ##1{\hsize\linewidth
            \vtop{\raggedright\hyphenpenalty\z@\exhyphenpenalty\z@
                \doublehyphendemerits\z@\finalhyphendemerits\z@
                \strut ##1\strut}%
        }%
        % If the linebreak is at a space, the latter will be displayed as visible
        % space at end of first line, and a continuation symbol starts next line.
        % Stretch/shrink are however usually zero for typewriter font.
        \def\FV@Space {%
            \nobreak\hskip\z@ plus\fontdimen3\font minus\fontdimen4\font
            \discretionary{\copy\Wrappedvisiblespacebox}{\Wrappedafterbreak}
            {\kern\fontdimen2\font}%
        }%
        
        % Allow breaks at special characters using \PYG... macros.
        \Wrappedbreaksatspecials
        % Breaks at punctuation characters . , ; ? ! and / need catcode=\active 	
        \OriginalVerbatim[#1,codes*=\Wrappedbreaksatpunct]%
    }
    \makeatother

    % Exact colors from NB
    \definecolor{incolor}{HTML}{303F9F}
    \definecolor{outcolor}{HTML}{D84315}
    \definecolor{cellborder}{HTML}{CFCFCF}
    \definecolor{cellbackground}{HTML}{F7F7F7}
    
    % prompt
    \makeatletter
    \newcommand{\boxspacing}{\kern\kvtcb@left@rule\kern\kvtcb@boxsep}
    \makeatother
    \newcommand{\prompt}[4]{
        \ttfamily\llap{{\color{#2}[#3]:\hspace{3pt}#4}}\vspace{-\baselineskip}
    }
    

    
    % Prevent overflowing lines due to hard-to-break entities
    \sloppy 
    % Setup hyperref package
    \hypersetup{
      breaklinks=true,  % so long urls are correctly broken across lines
      colorlinks=true,
      urlcolor=urlcolor,
      linkcolor=linkcolor,
      citecolor=citecolor,
      }
    % Slightly bigger margins than the latex defaults
    
    \geometry{verbose,tmargin=1in,bmargin=1in,lmargin=1in,rmargin=1in}
    
    

\begin{document}
    
    \maketitle
    
    

    
    \hypertarget{assignment-3-hello-vectors}{%
\section{Assignment 3: Hello Vectors}\label{assignment-3-hello-vectors}}

Welcome to this week's programming assignment of the specialization. In
this assignment we will explore word vectors. In natural language
processing, we represent each word as a vector consisting of numbers.
The vector encodes the meaning of the word. These numbers (or weights)
for each word are learned using various machine learning models, which
we will explore in more detail later in this specialization. Rather than
make you code the machine learning models from scratch, we will show you
how to use them. In the real world, you can always load the trained word
vectors, and you will almost never have to train them from scratch. In
this assignment you will

\begin{itemize}
\tightlist
\item
  Predict analogies between words.
\item
  Use PCA to reduce the dimensionality of the word embeddings and plot
  them in two dimensions.
\item
  Compare word embeddings by using a similarity measure (the cosine
  similarity).
\item
  Understand how these vector space models work.
\end{itemize}

    \hypertarget{important-note-on-submission-to-the-autograder}{%
\subsection{Important Note on Submission to the
AutoGrader}\label{important-note-on-submission-to-the-autograder}}

Before submitting your assignment to the AutoGrader, please make sure
you are not doing the following:

\begin{enumerate}
\def\labelenumi{\arabic{enumi}.}
\tightlist
\item
  You have not added any \emph{extra} \texttt{print} statement(s) in the
  assignment.
\item
  You have not added any \emph{extra} code cell(s) in the assignment.
\item
  You have not changed any of the function parameters.
\item
  You are not using any global variables inside your graded exercises.
  Unless specifically instructed to do so, please refrain from it and
  use the local variables instead.
\item
  You are not changing the assignment code where it is not required,
  like creating \emph{extra} variables.
\end{enumerate}

If you do any of the following, you will get something like,
\texttt{Grader\ Error:\ Grader\ feedback\ not\ found} (or similarly
unexpected) error upon submitting your assignment. Before asking for
help/debugging the errors in your assignment, check for these first. If
this is the case, and you don't remember the changes you have made, you
can get a fresh copy of the assignment by following these
\href{https://www.coursera.org/learn/classification-vector-spaces-in-nlp/supplement/YLuAg/h-ow-to-refresh-your-workspace}{instructions}.

    \hypertarget{table-of-contents}{%
\subsection{Table of Contents}\label{table-of-contents}}

\begin{itemize}
\tightlist
\item
  Section \ref{1}

  \begin{itemize}
  \tightlist
  \item
    Section \ref{1-1}
  \item
    Section \ref{1-2}

    \begin{itemize}
    \tightlist
    \item
      Section \ref{ex-1}
    \end{itemize}
  \item
    Section \ref{1-3}

    \begin{itemize}
    \tightlist
    \item
      Section \ref{ex-2}
    \end{itemize}
  \item
    Section \ref{1-4}

    \begin{itemize}
    \tightlist
    \item
      Section \ref{ex-3}
    \end{itemize}
  \item
    Section \ref{1-5}

    \begin{itemize}
    \tightlist
    \item
      Section \ref{ex-4}
    \end{itemize}
  \end{itemize}
\item
  Section \ref{2}

  \begin{itemize}
  \tightlist
  \item
    Section \ref{ex-5}
  \end{itemize}
\end{itemize}

    \#\# 1 - Predict the Countries from Capitals

During the presentation of the module, we have illustrated the word
analogies by finding the capital of a country from the country. In this
part of the assignment we have changed the problem a bit. You are asked
to predict the \textbf{countries} that correspond to some
\textbf{capitals}. You are playing trivia against some second grader who
just took their geography test and knows all the capitals by heart.
Thanks to NLP, you will be able to answer the questions properly. In
other words, you will write a program that can give you the country by
its capital. That way you are pretty sure you will win the trivia game.
We will start by exploring the data set.

\#\#\# 1.1 Importing the Data

As usual, you start by importing some essential Python libraries and the
load dataset. The dataset will be loaded as a
\href{https://pandas.pydata.org/pandas-docs/stable/getting_started/dsintro.html}{Pandas
DataFrame}, which is very a common method in data science. Because of
the large size of the data, this may take a few minutes.

    \begin{tcolorbox}[breakable, size=fbox, boxrule=1pt, pad at break*=1mm,colback=cellbackground, colframe=cellborder]
\prompt{In}{incolor}{1}{\boxspacing}
\begin{Verbatim}[commandchars=\\\{\}]
\PY{c+c1}{\PYZsh{} Run this cell to import packages.}
\PY{k+kn}{import} \PY{n+nn}{pickle}
\PY{k+kn}{import} \PY{n+nn}{numpy} \PY{k}{as} \PY{n+nn}{np}
\PY{k+kn}{import} \PY{n+nn}{pandas} \PY{k}{as} \PY{n+nn}{pd}
\PY{k+kn}{import} \PY{n+nn}{matplotlib}\PY{n+nn}{.}\PY{n+nn}{pyplot} \PY{k}{as} \PY{n+nn}{plt}
\PY{k+kn}{import} \PY{n+nn}{w3\PYZus{}unittest}

\PY{k+kn}{from} \PY{n+nn}{utils} \PY{k+kn}{import} \PY{n}{get\PYZus{}vectors}
\end{Verbatim}
\end{tcolorbox}

    \begin{tcolorbox}[breakable, size=fbox, boxrule=1pt, pad at break*=1mm,colback=cellbackground, colframe=cellborder]
\prompt{In}{incolor}{2}{\boxspacing}
\begin{Verbatim}[commandchars=\\\{\}]
\PY{n}{data} \PY{o}{=} \PY{n}{pd}\PY{o}{.}\PY{n}{read\PYZus{}csv}\PY{p}{(}\PY{l+s+s1}{\PYZsq{}}\PY{l+s+s1}{./data/capitals.txt}\PY{l+s+s1}{\PYZsq{}}\PY{p}{,} \PY{n}{delimiter}\PY{o}{=}\PY{l+s+s1}{\PYZsq{}}\PY{l+s+s1}{ }\PY{l+s+s1}{\PYZsq{}}\PY{p}{)}
\PY{n}{data}\PY{o}{.}\PY{n}{columns} \PY{o}{=} \PY{p}{[}\PY{l+s+s1}{\PYZsq{}}\PY{l+s+s1}{city1}\PY{l+s+s1}{\PYZsq{}}\PY{p}{,} \PY{l+s+s1}{\PYZsq{}}\PY{l+s+s1}{country1}\PY{l+s+s1}{\PYZsq{}}\PY{p}{,} \PY{l+s+s1}{\PYZsq{}}\PY{l+s+s1}{city2}\PY{l+s+s1}{\PYZsq{}}\PY{p}{,} \PY{l+s+s1}{\PYZsq{}}\PY{l+s+s1}{country2}\PY{l+s+s1}{\PYZsq{}}\PY{p}{]}

\PY{c+c1}{\PYZsh{} print first five elements in the DataFrame}
\PY{n}{data}\PY{o}{.}\PY{n}{head}\PY{p}{(}\PY{l+m+mi}{5}\PY{p}{)}
\end{Verbatim}
\end{tcolorbox}

            \begin{tcolorbox}[breakable, size=fbox, boxrule=.5pt, pad at break*=1mm, opacityfill=0]
\prompt{Out}{outcolor}{2}{\boxspacing}
\begin{Verbatim}[commandchars=\\\{\}]
    city1 country1    city2     country2
0  Athens   Greece  Bangkok     Thailand
1  Athens   Greece  Beijing        China
2  Athens   Greece   Berlin      Germany
3  Athens   Greece     Bern  Switzerland
4  Athens   Greece    Cairo        Egypt
\end{Verbatim}
\end{tcolorbox}
        
    \begin{center}\rule{0.5\linewidth}{0.5pt}\end{center}

\hypertarget{to-run-this-code-on-your-own-machine}{%
\paragraph{To Run This Code On Your Own
Machine:}\label{to-run-this-code-on-your-own-machine}}

Note that because the original google news word embedding dataset is
about 3.64 gigabytes, the workspace is not able to handle the full file
set. So we've downloaded the full dataset, extracted a sample of the
words that we're going to analyze in this assignment, and saved it in a
pickle file called word\_embeddings\_capitals.p

If you want to download the full dataset on your own and choose your own
set of word embeddings, please see the instructions and some helper
code.

\begin{itemize}
\tightlist
\item
  Download the dataset from this
  \href{https://code.google.com/archive/p/word2vec/}{page}.
\item
  Search in the page for `GoogleNews-vectors-negative300.bin.gz' and
  click the link to download.
\item
  You'll need to unzip the file.
\end{itemize}

    Copy-paste the code below and run it on your local machine after
downloading the dataset to the same directory as the notebook.

\begin{Shaded}
\begin{Highlighting}[]
\ImportTok{import}\NormalTok{ nltk}
\ImportTok{from}\NormalTok{ gensim.models }\ImportTok{import}\NormalTok{ KeyedVectors}


\NormalTok{embeddings }\OperatorTok{=}\NormalTok{ KeyedVectors.load\_word2vec\_format(}\StringTok{\textquotesingle{}./GoogleNews{-}vectors{-}negative300.bin\textquotesingle{}}\NormalTok{, binary }\OperatorTok{=} \VariableTok{True}\NormalTok{)}
\NormalTok{f }\OperatorTok{=} \BuiltInTok{open}\NormalTok{(}\StringTok{\textquotesingle{}capitals.txt\textquotesingle{}}\NormalTok{, }\StringTok{\textquotesingle{}r\textquotesingle{}}\NormalTok{).read()}
\NormalTok{set\_words }\OperatorTok{=} \BuiltInTok{set}\NormalTok{(nltk.word\_tokenize(f))}
\NormalTok{select\_words }\OperatorTok{=}\NormalTok{ words }\OperatorTok{=}\NormalTok{ [}\StringTok{\textquotesingle{}king\textquotesingle{}}\NormalTok{, }\StringTok{\textquotesingle{}queen\textquotesingle{}}\NormalTok{, }\StringTok{\textquotesingle{}oil\textquotesingle{}}\NormalTok{, }\StringTok{\textquotesingle{}gas\textquotesingle{}}\NormalTok{, }\StringTok{\textquotesingle{}happy\textquotesingle{}}\NormalTok{, }\StringTok{\textquotesingle{}sad\textquotesingle{}}\NormalTok{, }\StringTok{\textquotesingle{}city\textquotesingle{}}\NormalTok{, }\StringTok{\textquotesingle{}town\textquotesingle{}}\NormalTok{, }\StringTok{\textquotesingle{}village\textquotesingle{}}\NormalTok{, }\StringTok{\textquotesingle{}country\textquotesingle{}}\NormalTok{, }\StringTok{\textquotesingle{}continent\textquotesingle{}}\NormalTok{, }\StringTok{\textquotesingle{}petroleum\textquotesingle{}}\NormalTok{, }\StringTok{\textquotesingle{}joyful\textquotesingle{}}\NormalTok{]}
\ControlFlowTok{for}\NormalTok{ w }\KeywordTok{in}\NormalTok{ select\_words:}
\NormalTok{    set\_words.add(w)}

\KeywordTok{def}\NormalTok{ get\_word\_embeddings(embeddings):}

\NormalTok{    word\_embeddings }\OperatorTok{=}\NormalTok{ \{\}}
    \ControlFlowTok{for}\NormalTok{ word }\KeywordTok{in}\NormalTok{ embeddings.vocab:}
        \ControlFlowTok{if}\NormalTok{ word }\KeywordTok{in}\NormalTok{ set\_words:}
\NormalTok{            word\_embeddings[word] }\OperatorTok{=}\NormalTok{ embeddings[word]}
    \ControlFlowTok{return}\NormalTok{ word\_embeddings}


\CommentTok{\# Testing your function}
\NormalTok{word\_embeddings }\OperatorTok{=}\NormalTok{ get\_word\_embeddings(embeddings)}
\BuiltInTok{print}\NormalTok{(}\BuiltInTok{len}\NormalTok{(word\_embeddings))}
\NormalTok{pickle.dump( word\_embeddings, }\BuiltInTok{open}\NormalTok{( }\StringTok{"word\_embeddings\_subset.p"}\NormalTok{, }\StringTok{"wb"}\NormalTok{ ) )}
\end{Highlighting}
\end{Shaded}

\begin{center}\rule{0.5\linewidth}{0.5pt}\end{center}

    Now we will load the word embeddings as a
\href{https://docs.python.org/3/tutorial/datastructures.html\#dictionaries}{Python
dictionary}. As stated, these have already been obtained through a
machine learning algorithm.

    \begin{tcolorbox}[breakable, size=fbox, boxrule=1pt, pad at break*=1mm,colback=cellbackground, colframe=cellborder]
\prompt{In}{incolor}{3}{\boxspacing}
\begin{Verbatim}[commandchars=\\\{\}]
\PY{n}{word\PYZus{}embeddings} \PY{o}{=} \PY{n}{pickle}\PY{o}{.}\PY{n}{load}\PY{p}{(}\PY{n+nb}{open}\PY{p}{(}\PY{l+s+s2}{\PYZdq{}}\PY{l+s+s2}{./data/word\PYZus{}embeddings\PYZus{}subset.p}\PY{l+s+s2}{\PYZdq{}}\PY{p}{,} \PY{l+s+s2}{\PYZdq{}}\PY{l+s+s2}{rb}\PY{l+s+s2}{\PYZdq{}}\PY{p}{)}\PY{p}{)}
\PY{n+nb}{len}\PY{p}{(}\PY{n}{word\PYZus{}embeddings}\PY{p}{)}  \PY{c+c1}{\PYZsh{} there should be 243 words that will be used in this assignment}
\end{Verbatim}
\end{tcolorbox}

            \begin{tcolorbox}[breakable, size=fbox, boxrule=.5pt, pad at break*=1mm, opacityfill=0]
\prompt{Out}{outcolor}{3}{\boxspacing}
\begin{Verbatim}[commandchars=\\\{\}]
243
\end{Verbatim}
\end{tcolorbox}
        
    Each of the word embedding is a 300-dimensional vector.

    \begin{tcolorbox}[breakable, size=fbox, boxrule=1pt, pad at break*=1mm,colback=cellbackground, colframe=cellborder]
\prompt{In}{incolor}{4}{\boxspacing}
\begin{Verbatim}[commandchars=\\\{\}]
\PY{n+nb}{print}\PY{p}{(}\PY{l+s+s2}{\PYZdq{}}\PY{l+s+s2}{dimension: }\PY{l+s+si}{\PYZob{}\PYZcb{}}\PY{l+s+s2}{\PYZdq{}}\PY{o}{.}\PY{n}{format}\PY{p}{(}\PY{n}{word\PYZus{}embeddings}\PY{p}{[}\PY{l+s+s1}{\PYZsq{}}\PY{l+s+s1}{Spain}\PY{l+s+s1}{\PYZsq{}}\PY{p}{]}\PY{o}{.}\PY{n}{shape}\PY{p}{[}\PY{l+m+mi}{0}\PY{p}{]}\PY{p}{)}\PY{p}{)}
\end{Verbatim}
\end{tcolorbox}

    \begin{Verbatim}[commandchars=\\\{\}]
dimension: 300
    \end{Verbatim}

    \hypertarget{predict-relationships-among-words}{%
\paragraph{Predict relationships among
words}\label{predict-relationships-among-words}}

Now you will write a function that will use the word embeddings to
predict relationships among words. * The function will take as input
three words. * The first two are related to each other. * It will
predict a 4th word which is related to the third word in a similar
manner as the two first words are related to each other. * As an
example, ``Athens is to Greece as Bangkok is to \_\_\_\_\_\_''? * You
will write a program that is capable of finding the fourth word. * We
will give you a hint to show you how to compute this.

A similar analogy would be the following:

You will implement a function that can tell you the capital of a
country. You should use the same methodology shown in the figure above.
To do this, you'll first compute the cosine similarity metric or the
Euclidean distance.

    \#\#\# 1.2 Cosine Similarity

The cosine similarity function is:

\[\cos (\theta)=\frac{\mathbf{A} \cdot \mathbf{B}}{\|\mathbf{A}\|\|\mathbf{B}\|}=\frac{\sum_{i=1}^{n} A_{i} B_{i}}{\sqrt{\sum_{i=1}^{n} A_{i}^{2}} \sqrt{\sum_{i=1}^{n} B_{i}^{2}}}\tag{1}\]

\(A\) and \(B\) represent the word vectors and \(A_i\) or \(B_i\)
represent index i of that vector. Note that if A and B are identical,
you will get \(cos(\theta) = 1\). * Otherwise, if they are the total
opposite, meaning, \(A= -B\), then you would get \(cos(\theta) = -1\). *
If you get \(cos(\theta) =0\), that means that they are orthogonal (or
perpendicular). * Numbers between 0 and 1 indicate a similarity score. *
Numbers between -1 and 0 indicate a dissimilarity score.

\#\#\# Exercise 1 - cosine\_similarity Implement a function that takes
in two word vectors and computes the cosine distance.

    Hints

Python's NumPy library adds support for linear algebra operations (e.g.,
dot product, vector norm \ldots).

Use numpy.dot .

Use numpy.linalg.norm .

    \begin{tcolorbox}[breakable, size=fbox, boxrule=1pt, pad at break*=1mm,colback=cellbackground, colframe=cellborder]
\prompt{In}{incolor}{7}{\boxspacing}
\begin{Verbatim}[commandchars=\\\{\}]
\PY{c+c1}{\PYZsh{} UNQ\PYZus{}C1 GRADED FUNCTION: cosine\PYZus{}similarity}

\PY{k}{def} \PY{n+nf}{cosine\PYZus{}similarity}\PY{p}{(}\PY{n}{A}\PY{p}{,} \PY{n}{B}\PY{p}{)}\PY{p}{:}
    \PY{l+s+sd}{\PYZsq{}\PYZsq{}\PYZsq{}}
\PY{l+s+sd}{    Input:}
\PY{l+s+sd}{        A: a numpy array which corresponds to a word vector}
\PY{l+s+sd}{        B: A numpy array which corresponds to a word vector}
\PY{l+s+sd}{    Output:}
\PY{l+s+sd}{        cos: numerical number representing the cosine similarity between A and B.}
\PY{l+s+sd}{    \PYZsq{}\PYZsq{}\PYZsq{}}

    \PY{c+c1}{\PYZsh{}\PYZsh{}\PYZsh{} START CODE HERE \PYZsh{}\PYZsh{}\PYZsh{}}
    \PY{n}{dot} \PY{o}{=} \PY{n}{np}\PY{o}{.}\PY{n}{dot}\PY{p}{(}\PY{n}{A}\PY{p}{,}\PY{n}{B}\PY{p}{)}    
    \PY{n}{norma} \PY{o}{=} \PY{n}{np}\PY{o}{.}\PY{n}{linalg}\PY{o}{.}\PY{n}{norm}\PY{p}{(}\PY{n}{A}\PY{p}{,}\PY{n}{axis} \PY{o}{=} \PY{l+m+mi}{0}\PY{p}{)}
    \PY{n}{normb} \PY{o}{=} \PY{n}{np}\PY{o}{.}\PY{n}{linalg}\PY{o}{.}\PY{n}{norm}\PY{p}{(}\PY{n}{B}\PY{p}{,}\PY{n}{axis} \PY{o}{=} \PY{l+m+mi}{0}\PY{p}{)}
    \PY{n}{cos} \PY{o}{=} \PY{n}{dot}\PY{o}{/}\PY{p}{(}\PY{n}{norma}\PY{o}{*}\PY{n}{normb}\PY{p}{)}

    \PY{c+c1}{\PYZsh{}\PYZsh{}\PYZsh{} END CODE HERE \PYZsh{}\PYZsh{}\PYZsh{}}
    \PY{k}{return} \PY{n}{cos}
\end{Verbatim}
\end{tcolorbox}

    \begin{tcolorbox}[breakable, size=fbox, boxrule=1pt, pad at break*=1mm,colback=cellbackground, colframe=cellborder]
\prompt{In}{incolor}{8}{\boxspacing}
\begin{Verbatim}[commandchars=\\\{\}]
\PY{c+c1}{\PYZsh{} feel free to try different words}
\PY{n}{king} \PY{o}{=} \PY{n}{word\PYZus{}embeddings}\PY{p}{[}\PY{l+s+s1}{\PYZsq{}}\PY{l+s+s1}{king}\PY{l+s+s1}{\PYZsq{}}\PY{p}{]}
\PY{n}{queen} \PY{o}{=} \PY{n}{word\PYZus{}embeddings}\PY{p}{[}\PY{l+s+s1}{\PYZsq{}}\PY{l+s+s1}{queen}\PY{l+s+s1}{\PYZsq{}}\PY{p}{]}

\PY{n}{cosine\PYZus{}similarity}\PY{p}{(}\PY{n}{king}\PY{p}{,} \PY{n}{queen}\PY{p}{)}
\end{Verbatim}
\end{tcolorbox}

            \begin{tcolorbox}[breakable, size=fbox, boxrule=.5pt, pad at break*=1mm, opacityfill=0]
\prompt{Out}{outcolor}{8}{\boxspacing}
\begin{Verbatim}[commandchars=\\\{\}]
0.65109557
\end{Verbatim}
\end{tcolorbox}
        
    \textbf{Expected Output}:

\(\approx\) 0.651095

    \begin{tcolorbox}[breakable, size=fbox, boxrule=1pt, pad at break*=1mm,colback=cellbackground, colframe=cellborder]
\prompt{In}{incolor}{9}{\boxspacing}
\begin{Verbatim}[commandchars=\\\{\}]
\PY{c+c1}{\PYZsh{} Test your function}
\PY{n}{w3\PYZus{}unittest}\PY{o}{.}\PY{n}{test\PYZus{}cosine\PYZus{}similarity}\PY{p}{(}\PY{n}{cosine\PYZus{}similarity}\PY{p}{)}
\end{Verbatim}
\end{tcolorbox}

    \begin{Verbatim}[commandchars=\\\{\}]
\textcolor{ansi-green-intense}{ All tests passed}
    \end{Verbatim}

    \#\#\# 1.3 Euclidean Distance

You will now implement a function that computes the similarity between
two vectors using the Euclidean distance. Euclidean distance is defined
as:

\[ \begin{aligned} d(\mathbf{A}, \mathbf{B})=d(\mathbf{B}, \mathbf{A}) &=\sqrt{\left(A_{1}-B_{1}\right)^{2}+\left(A_{2}-B_{2}\right)^{2}+\cdots+\left(A_{n}-B_{n}\right)^{2}} \\ &=\sqrt{\sum_{i=1}^{n}\left(A_{i}-B_{i}\right)^{2}} \end{aligned}\]

\begin{itemize}
\tightlist
\item
  \(n\) is the number of elements in the vector
\item
  \(A\) and \(B\) are the corresponding word vectors.
\item
  The more similar the words, the more likely the Euclidean distance
  will be close to 0.
\end{itemize}

\#\#\# Exercise 2 - euclidean Implement a function that computes the
Euclidean distance between two vectors.

    Hints

Use numpy.linalg.norm .

    \begin{tcolorbox}[breakable, size=fbox, boxrule=1pt, pad at break*=1mm,colback=cellbackground, colframe=cellborder]
\prompt{In}{incolor}{21}{\boxspacing}
\begin{Verbatim}[commandchars=\\\{\}]
\PY{c+c1}{\PYZsh{} UNQ\PYZus{}C2 GRADED FUNCTION: euclidean}

\PY{k}{def} \PY{n+nf}{euclidean}\PY{p}{(}\PY{n}{A}\PY{p}{,} \PY{n}{B}\PY{p}{)}\PY{p}{:}
    \PY{l+s+sd}{\PYZdq{}\PYZdq{}\PYZdq{}}
\PY{l+s+sd}{    Input:}
\PY{l+s+sd}{        A: a numpy array which corresponds to a word vector}
\PY{l+s+sd}{        B: A numpy array which corresponds to a word vector}
\PY{l+s+sd}{    Output:}
\PY{l+s+sd}{        d: numerical number representing the Euclidean distance between A and B.}
\PY{l+s+sd}{    \PYZdq{}\PYZdq{}\PYZdq{}}

    \PY{c+c1}{\PYZsh{}\PYZsh{}\PYZsh{} START CODE HERE \PYZsh{}\PYZsh{}\PYZsh{}}

    \PY{c+c1}{\PYZsh{} euclidean distance    }
    \PY{n}{d} \PY{o}{=} \PY{n}{np}\PY{o}{.}\PY{n}{linalg}\PY{o}{.}\PY{n}{norm}\PY{p}{(}\PY{n}{A}\PY{o}{\PYZhy{}}\PY{n}{B}\PY{p}{,}\PY{n}{axis} \PY{o}{=} \PY{l+m+mi}{0}\PY{p}{)}

    \PY{c+c1}{\PYZsh{}\PYZsh{}\PYZsh{} END CODE HERE \PYZsh{}\PYZsh{}\PYZsh{}}

    \PY{k}{return} \PY{n}{d}
\end{Verbatim}
\end{tcolorbox}

    \begin{tcolorbox}[breakable, size=fbox, boxrule=1pt, pad at break*=1mm,colback=cellbackground, colframe=cellborder]
\prompt{In}{incolor}{22}{\boxspacing}
\begin{Verbatim}[commandchars=\\\{\}]
\PY{c+c1}{\PYZsh{} Test your function}
\PY{n}{euclidean}\PY{p}{(}\PY{n}{king}\PY{p}{,} \PY{n}{queen}\PY{p}{)}
\end{Verbatim}
\end{tcolorbox}

            \begin{tcolorbox}[breakable, size=fbox, boxrule=.5pt, pad at break*=1mm, opacityfill=0]
\prompt{Out}{outcolor}{22}{\boxspacing}
\begin{Verbatim}[commandchars=\\\{\}]
2.4796925
\end{Verbatim}
\end{tcolorbox}
        
    \textbf{Expected Output:}

2.4796925

    \begin{tcolorbox}[breakable, size=fbox, boxrule=1pt, pad at break*=1mm,colback=cellbackground, colframe=cellborder]
\prompt{In}{incolor}{23}{\boxspacing}
\begin{Verbatim}[commandchars=\\\{\}]
\PY{c+c1}{\PYZsh{} Test your function}
\PY{n}{w3\PYZus{}unittest}\PY{o}{.}\PY{n}{test\PYZus{}euclidean}\PY{p}{(}\PY{n}{euclidean}\PY{p}{)}
\end{Verbatim}
\end{tcolorbox}

    \begin{Verbatim}[commandchars=\\\{\}]
\textcolor{ansi-green-intense}{ All tests passed}
    \end{Verbatim}

    \#\#\# 1.4 Finding the Country of each Capital

Now, you will use the previous functions to compute similarities between
vectors, and use these to find the capital cities of countries. You will
write a function that takes in three words, and the embeddings
dictionary. Your task is to find the capital cities. For example, given
the following words:

\begin{itemize}
\tightlist
\item
  1: Athens 2: Greece 3: Baghdad,
\end{itemize}

your task is to predict the country 4: Iraq.

\#\#\# Exercise 3 - get\_country \textbf{Instructions}:

\begin{enumerate}
\def\labelenumi{\arabic{enumi}.}
\item
  To predict the capital you might want to look at the \emph{King - Man
  + Woman = Queen} example above, and implement that scheme into a
  mathematical function, using the word embeddings and a similarity
  function.
\item
  Iterate over the embeddings dictionary and compute the cosine
  similarity score between your vector and the current word embedding.
\item
  You should add a check to make sure that the word you return is not
  any of the words that you fed into your function. Return the one with
  the highest score.
\end{enumerate}

    \begin{tcolorbox}[breakable, size=fbox, boxrule=1pt, pad at break*=1mm,colback=cellbackground, colframe=cellborder]
\prompt{In}{incolor}{32}{\boxspacing}
\begin{Verbatim}[commandchars=\\\{\}]
\PY{c+c1}{\PYZsh{} UNQ\PYZus{}C3 GRADED FUNCTION: get\PYZus{}country}

\PY{k}{def} \PY{n+nf}{get\PYZus{}country}\PY{p}{(}\PY{n}{city1}\PY{p}{,} \PY{n}{country1}\PY{p}{,} \PY{n}{city2}\PY{p}{,} \PY{n}{embeddings}\PY{p}{,} \PY{n}{cosine\PYZus{}similarity}\PY{o}{=}\PY{n}{cosine\PYZus{}similarity}\PY{p}{)}\PY{p}{:}
    \PY{l+s+sd}{\PYZdq{}\PYZdq{}\PYZdq{}}
\PY{l+s+sd}{    Input:}
\PY{l+s+sd}{        city1: a string (the capital city of country1)}
\PY{l+s+sd}{        country1: a string (the country of capital1)}
\PY{l+s+sd}{        city2: a string (the capital city of country2)}
\PY{l+s+sd}{        embeddings: a dictionary where the keys are words and values are their emmbeddings}
\PY{l+s+sd}{    Output:}
\PY{l+s+sd}{        country: a tuple with the most likely country and its similarity score}
\PY{l+s+sd}{    \PYZdq{}\PYZdq{}\PYZdq{}}
    \PY{c+c1}{\PYZsh{}\PYZsh{}\PYZsh{} START CODE HERE \PYZsh{}\PYZsh{}\PYZsh{}}

    \PY{c+c1}{\PYZsh{} store the city1, country 1, and city 2 in a set called group}
    \PY{n}{group} \PY{o}{=} \PY{p}{\PYZob{}}\PY{n}{city1}\PY{p}{,}\PY{n}{country1}\PY{p}{,}\PY{n}{city2}\PY{p}{\PYZcb{}}

    \PY{c+c1}{\PYZsh{} get embeddings of city 1}
    \PY{n}{city1\PYZus{}emb} \PY{o}{=} \PY{n}{word\PYZus{}embeddings}\PY{p}{[}\PY{n}{city1}\PY{p}{]}

    \PY{c+c1}{\PYZsh{} get embedding of country 1}
    \PY{n}{country1\PYZus{}emb} \PY{o}{=} \PY{n}{word\PYZus{}embeddings}\PY{p}{[}\PY{n}{country1}\PY{p}{]}

    \PY{c+c1}{\PYZsh{} get embedding of city 2}
    \PY{n}{city2\PYZus{}emb} \PY{o}{=} \PY{n}{word\PYZus{}embeddings}\PY{p}{[}\PY{n}{city2}\PY{p}{]}

    \PY{c+c1}{\PYZsh{} get embedding of country 2 (it\PYZsq{}s a combination of the embeddings of country 1, city 1 and city 2)}
    \PY{c+c1}{\PYZsh{} Remember: King \PYZhy{} Man + Woman = Queen}
    \PY{n}{vec} \PY{o}{=} \PY{n}{country1\PYZus{}emb} \PY{o}{\PYZhy{}} \PY{n}{city1\PYZus{}emb} \PY{o}{+} \PY{n}{city2\PYZus{}emb}

    \PY{c+c1}{\PYZsh{} Initialize the similarity to \PYZhy{}1 (it will be replaced by a similarities that are closer to +1)}
    \PY{n}{similarity} \PY{o}{=} \PY{o}{\PYZhy{}}\PY{l+m+mi}{1}

    \PY{c+c1}{\PYZsh{} initialize country to an empty string}
    \PY{n}{country} \PY{o}{=} \PY{l+s+s1}{\PYZsq{}}\PY{l+s+s1}{\PYZsq{}}

    \PY{c+c1}{\PYZsh{} loop through all words in the embeddings dictionary}
    \PY{k}{for} \PY{n}{word} \PY{o+ow}{in} \PY{n}{embeddings}\PY{o}{.}\PY{n}{keys}\PY{p}{(}\PY{p}{)}\PY{p}{:}

        \PY{c+c1}{\PYZsh{} first check that the word is not already in the \PYZsq{}group\PYZsq{}}
        \PY{k}{if} \PY{n}{word} \PY{o+ow}{not} \PY{o+ow}{in} \PY{n}{group}\PY{p}{:}

            \PY{c+c1}{\PYZsh{} get the word embedding}
            \PY{n}{word\PYZus{}emb} \PY{o}{=} \PY{n}{word\PYZus{}embeddings}\PY{p}{[}\PY{n}{word}\PY{p}{]}
            
            \PY{c+c1}{\PYZsh{} calculate cosine similarity between embedding of country 2 and the word in the embeddings dictionary}
            \PY{n}{cur\PYZus{}similarity} \PY{o}{=} \PY{n}{cosine\PYZus{}similarity}\PY{p}{(}\PY{n}{vec}\PY{p}{,} \PY{n}{word\PYZus{}emb}\PY{p}{)}

            \PY{c+c1}{\PYZsh{} if the cosine similarity is more similar than the previously best similarity...}
            \PY{k}{if} \PY{n}{cur\PYZus{}similarity} \PY{o}{\PYZgt{}} \PY{n}{similarity}\PY{p}{:}

                \PY{c+c1}{\PYZsh{} update the similarity to the new, better similarity}
                \PY{n}{similarity} \PY{o}{=} \PY{n}{cur\PYZus{}similarity}

                \PY{c+c1}{\PYZsh{} store the country as a tuple, which contains the word and the similarity}
                \PY{n}{country} \PY{o}{=} \PY{p}{(}\PY{n}{word}\PY{p}{,}\PY{n}{similarity}\PY{p}{)}

    \PY{c+c1}{\PYZsh{}\PYZsh{}\PYZsh{} END CODE HERE \PYZsh{}\PYZsh{}\PYZsh{}}

    \PY{k}{return} \PY{n}{country}
\end{Verbatim}
\end{tcolorbox}

    \begin{tcolorbox}[breakable, size=fbox, boxrule=1pt, pad at break*=1mm,colback=cellbackground, colframe=cellborder]
\prompt{In}{incolor}{33}{\boxspacing}
\begin{Verbatim}[commandchars=\\\{\}]
\PY{c+c1}{\PYZsh{} Testing your function, note to make it more robust you can return the 5 most similar words.}
\PY{n}{get\PYZus{}country}\PY{p}{(}\PY{l+s+s1}{\PYZsq{}}\PY{l+s+s1}{Athens}\PY{l+s+s1}{\PYZsq{}}\PY{p}{,} \PY{l+s+s1}{\PYZsq{}}\PY{l+s+s1}{Greece}\PY{l+s+s1}{\PYZsq{}}\PY{p}{,} \PY{l+s+s1}{\PYZsq{}}\PY{l+s+s1}{Cairo}\PY{l+s+s1}{\PYZsq{}}\PY{p}{,} \PY{n}{word\PYZus{}embeddings}\PY{p}{)}
\end{Verbatim}
\end{tcolorbox}

            \begin{tcolorbox}[breakable, size=fbox, boxrule=.5pt, pad at break*=1mm, opacityfill=0]
\prompt{Out}{outcolor}{33}{\boxspacing}
\begin{Verbatim}[commandchars=\\\{\}]
('Egypt', 0.7626822)
\end{Verbatim}
\end{tcolorbox}
        
    \textbf{Expected Output: (Approximately)}

(`Egypt', 0.7626821)

    \begin{tcolorbox}[breakable, size=fbox, boxrule=1pt, pad at break*=1mm,colback=cellbackground, colframe=cellborder]
\prompt{In}{incolor}{34}{\boxspacing}
\begin{Verbatim}[commandchars=\\\{\}]
\PY{c+c1}{\PYZsh{} Test your function}
\PY{n}{w3\PYZus{}unittest}\PY{o}{.}\PY{n}{test\PYZus{}get\PYZus{}country}\PY{p}{(}\PY{n}{get\PYZus{}country}\PY{p}{)}
\end{Verbatim}
\end{tcolorbox}

    \begin{Verbatim}[commandchars=\\\{\}]
\textcolor{ansi-green-intense}{ All tests passed}
    \end{Verbatim}

    \#\#\# 1.5 Model Accuracy

Now you will test your new function on the dataset and check the
accuracy of the model:

\[\text{Accuracy}=\frac{\text{Correct # of predictions}}{\text{Total # of predictions}}\]

\#\#\# Exercise 4 - get\_accuracy \textbf{Instructions}: Implement a
program that can compute the accuracy on the dataset provided for you.
You have to iterate over every row to get the corresponding words and
feed them into you \texttt{get\_country} function above.

    Hints

Use pandas.DataFrame.iterrows .

    \begin{tcolorbox}[breakable, size=fbox, boxrule=1pt, pad at break*=1mm,colback=cellbackground, colframe=cellborder]
\prompt{In}{incolor}{37}{\boxspacing}
\begin{Verbatim}[commandchars=\\\{\}]
\PY{c+c1}{\PYZsh{} UNQ\PYZus{}C4 GRADED FUNCTION: get\PYZus{}accuracy}

\PY{k}{def} \PY{n+nf}{get\PYZus{}accuracy}\PY{p}{(}\PY{n}{word\PYZus{}embeddings}\PY{p}{,} \PY{n}{data}\PY{p}{,} \PY{n}{get\PYZus{}country}\PY{o}{=}\PY{n}{get\PYZus{}country}\PY{p}{)}\PY{p}{:}
    \PY{l+s+sd}{\PYZsq{}\PYZsq{}\PYZsq{}}
\PY{l+s+sd}{    Input:}
\PY{l+s+sd}{        word\PYZus{}embeddings: a dictionary where the key is a word and the value is its embedding}
\PY{l+s+sd}{        data: a pandas DataFrame containing all the country and capital city pairs}

\PY{l+s+sd}{    \PYZsq{}\PYZsq{}\PYZsq{}}

    \PY{c+c1}{\PYZsh{}\PYZsh{}\PYZsh{} START CODE HERE \PYZsh{}\PYZsh{}\PYZsh{}}
    \PY{c+c1}{\PYZsh{} initialize num correct to zero}
    \PY{n}{num\PYZus{}correct} \PY{o}{=} \PY{l+m+mi}{0}

    \PY{c+c1}{\PYZsh{} loop through the rows of the dataframe}
    \PY{k}{for} \PY{n}{i}\PY{p}{,} \PY{n}{row} \PY{o+ow}{in} \PY{n}{data}\PY{o}{.}\PY{n}{iterrows}\PY{p}{(}\PY{p}{)}\PY{p}{:}

        \PY{c+c1}{\PYZsh{} get city1}
        \PY{n}{city1} \PY{o}{=} \PY{n}{row}\PY{p}{[}\PY{l+m+mi}{0}\PY{p}{]}

        \PY{c+c1}{\PYZsh{} get country1}
        \PY{n}{country1} \PY{o}{=} \PY{n}{row}\PY{p}{[}\PY{l+m+mi}{1}\PY{p}{]}

        \PY{c+c1}{\PYZsh{} get city2}
        \PY{n}{city2} \PY{o}{=} \PY{n}{row}\PY{p}{[}\PY{l+m+mi}{2}\PY{p}{]}

        \PY{c+c1}{\PYZsh{} get country2}
        \PY{n}{country2} \PY{o}{=} \PY{n}{row}\PY{p}{[}\PY{l+m+mi}{3}\PY{p}{]}

        \PY{c+c1}{\PYZsh{} use get\PYZus{}country to find the predicted country2}
        \PY{n}{predicted\PYZus{}country2}\PY{p}{,} \PY{n}{\PYZus{}} \PY{o}{=} \PY{n}{get\PYZus{}country}\PY{p}{(}\PY{n}{city1}\PY{p}{,} \PY{n}{country1}\PY{p}{,} \PY{n}{city2}\PY{p}{,} \PY{n}{word\PYZus{}embeddings}\PY{p}{)}

        \PY{c+c1}{\PYZsh{} if the predicted country2 is the same as the actual country2...}
        \PY{k}{if} \PY{n}{predicted\PYZus{}country2} \PY{o}{==} \PY{n}{country2}\PY{p}{:}
            \PY{c+c1}{\PYZsh{} increment the number of correct by 1}
            \PY{n}{num\PYZus{}correct} \PY{o}{+}\PY{o}{=} \PY{l+m+mi}{1}

    \PY{c+c1}{\PYZsh{} get the number of rows in the data dataframe (length of dataframe)}
    \PY{n}{m} \PY{o}{=} \PY{n+nb}{len}\PY{p}{(}\PY{n}{data}\PY{p}{)}

    \PY{c+c1}{\PYZsh{} calculate the accuracy by dividing the number correct by m}
    \PY{n}{accuracy} \PY{o}{=} \PY{n}{num\PYZus{}correct}\PY{o}{/}\PY{n}{m}\PY{p}{;}

    \PY{c+c1}{\PYZsh{}\PYZsh{}\PYZsh{} END CODE HERE \PYZsh{}\PYZsh{}\PYZsh{}}
    \PY{k}{return} \PY{n}{accuracy}
\end{Verbatim}
\end{tcolorbox}

    \textbf{NOTE: The cell below takes about 30 SECONDS to run.}

    \begin{tcolorbox}[breakable, size=fbox, boxrule=1pt, pad at break*=1mm,colback=cellbackground, colframe=cellborder]
\prompt{In}{incolor}{38}{\boxspacing}
\begin{Verbatim}[commandchars=\\\{\}]
\PY{n}{accuracy} \PY{o}{=} \PY{n}{get\PYZus{}accuracy}\PY{p}{(}\PY{n}{word\PYZus{}embeddings}\PY{p}{,} \PY{n}{data}\PY{p}{)}
\PY{n+nb}{print}\PY{p}{(}\PY{l+s+sa}{f}\PY{l+s+s2}{\PYZdq{}}\PY{l+s+s2}{Accuracy is }\PY{l+s+si}{\PYZob{}}\PY{n}{accuracy}\PY{l+s+si}{:}\PY{l+s+s2}{.2f}\PY{l+s+si}{\PYZcb{}}\PY{l+s+s2}{\PYZdq{}}\PY{p}{)}
\end{Verbatim}
\end{tcolorbox}

    \begin{Verbatim}[commandchars=\\\{\}]
Accuracy is 0.92
    \end{Verbatim}

    \textbf{Expected Output:}

\(\approx\) 0.92

    \begin{tcolorbox}[breakable, size=fbox, boxrule=1pt, pad at break*=1mm,colback=cellbackground, colframe=cellborder]
\prompt{In}{incolor}{39}{\boxspacing}
\begin{Verbatim}[commandchars=\\\{\}]
\PY{c+c1}{\PYZsh{} Test your function}
\PY{n}{w3\PYZus{}unittest}\PY{o}{.}\PY{n}{test\PYZus{}get\PYZus{}accuracy}\PY{p}{(}\PY{n}{get\PYZus{}accuracy}\PY{p}{,} \PY{n}{data}\PY{p}{)}
\end{Verbatim}
\end{tcolorbox}

    \begin{Verbatim}[commandchars=\\\{\}]
\textcolor{ansi-green-intense}{ All tests passed}
    \end{Verbatim}

    \#\# 2 - Plotting the vectors using PCA

Now you will explore the distance between word vectors after reducing
their dimension. The technique we will employ is known as
\href{https://en.wikipedia.org/wiki/Principal_component_analysis}{\emph{principal
component analysis} (PCA)}. As we saw, we are working in a
300-dimensional space in this case. Although from a computational
perspective we were able to perform a good job, it is impossible to
visualize results in such high dimensional spaces.

You can think of PCA as a method that projects our vectors in a space of
reduced dimension, while keeping the maximum information about the
original vectors in their reduced counterparts. In this case, by
\emph{maximum infomation} we mean that the Euclidean distance between
the original vectors and their projected siblings is minimal. Hence
vectors that were originally close in the embeddings dictionary, will
produce lower dimensional vectors that are still close to each other.

You will see that when you map out the words, similar words will be
clustered next to each other. For example, the words `sad', `happy',
`joyful' all describe emotion and are supposed to be near each other
when plotted. The words: `oil', `gas', and `petroleum' all describe
natural resources. Words like `city', `village', `town' could be seen as
synonyms and describe a similar thing.

Before plotting the words, you need to first be able to reduce each word
vector with PCA into 2 dimensions and then plot it. The steps to compute
PCA are as follows:

\begin{enumerate}
\def\labelenumi{\arabic{enumi}.}
\tightlist
\item
  Mean normalize the data
\item
  Compute the covariance matrix of your data (\(\Sigma\)).
\item
  Compute the eigenvectors and the eigenvalues of your covariance matrix
\item
  Multiply the first K eigenvectors by your normalized data. The
  transformation should look something as follows:
\end{enumerate}

    \#\#\# Exercise 5 - compute\_pca

\textbf{Instructions}:

Implement a program that takes in a data set where each row corresponds
to a word vector. * The word vectors are of dimension 300. * Use PCA to
change the 300 dimensions to \texttt{n\_components} dimensions. * The
new matrix should be of dimension \texttt{m,\ n\_components}.

\begin{itemize}
\tightlist
\item
  First de-mean the data
\item
  Get the eigenvalues using \texttt{linalg.eigh}. Use `eigh' rather than
  `eig' since R is symmetric. The performance gain when using eigh
  instead of eig is substantial.
\item
  Sort the eigenvectors and eigenvalues by decreasing order of the
  eigenvalues.
\item
  Get a subset of the eigenvectors (choose how many principle components
  you want to use using n\_components).
\item
  Return the new transformation of the data by multiplying the
  eigenvectors with the original data.
\end{itemize}

    Hints

Use numpy.mean(a,axis=None) which takes one required parameter. You need
to specify the optional argument axis for this exercise: If you set axis
= 0, you take the mean for each column. If you set axis = 1, you take
the mean for each row. Remember that each row is a word vector, and the
number of columns are the number of dimensions in a word vector.

Use numpy.cov(m, rowvar=True) which takes one required parameter. You
need to specify the optional argument rowvar for this exercise. This
calculates the covariance matrix. By default rowvar is True. From the
documentation: ``If rowvar is True (default), then each row represents a
variable, with observations in the columns.'' In our case, each row is a
word vector observation, and each column is a feature (variable).

Use numpy.linalg.eigh(a, UPLO=`L')

Use numpy.argsort sorts the values in an array from smallest to largest,
then returns the indices from this sort.

In order to reverse the order of a list, you can use: x{[}::-1{]}.

To apply the sorted indices to eigenvalues, you can use this format
x{[}indices\_sorted{]}.

When applying the sorted indices to eigen vectors, note that each column
represents an eigenvector. In order to preserve the rows but sort on the
columns, you can use this format x{[}:,indices\_sorted{]}

To transform the data using a subset of the most relevant principle
components, take the matrix multiplication of the eigenvectors with the
original data.

The data is of shape (n\_observations, n\_features).

The subset of eigenvectors are in a matrix of shape (n\_features,
n\_components).

To multiply these together, take the transposes of both the eigenvectors
(n\_components, n\_features) and the data (n\_features,
n\_observations).

The product of these two has dimensions (n\_components,n\_observations).
Take its transpose to get the shape (n\_observations, n\_components).

    \begin{tcolorbox}[breakable, size=fbox, boxrule=1pt, pad at break*=1mm,colback=cellbackground, colframe=cellborder]
\prompt{In}{incolor}{52}{\boxspacing}
\begin{Verbatim}[commandchars=\\\{\}]
\PY{c+c1}{\PYZsh{} UNQ\PYZus{}C5 GRADED FUNCTION: compute\PYZus{}pca}


\PY{k}{def} \PY{n+nf}{compute\PYZus{}pca}\PY{p}{(}\PY{n}{X}\PY{p}{,} \PY{n}{n\PYZus{}components}\PY{o}{=}\PY{l+m+mi}{2}\PY{p}{)}\PY{p}{:}
    \PY{l+s+sd}{\PYZdq{}\PYZdq{}\PYZdq{}}
\PY{l+s+sd}{    Input:}
\PY{l+s+sd}{        X: of dimension (m,n) where each row corresponds to a word vector}
\PY{l+s+sd}{        n\PYZus{}components: Number of components you want to keep.}
\PY{l+s+sd}{    Output:}
\PY{l+s+sd}{        X\PYZus{}reduced: data transformed in 2 dims/columns + regenerated original data}
\PY{l+s+sd}{    pass in: data as 2D NumPy array}
\PY{l+s+sd}{    \PYZdq{}\PYZdq{}\PYZdq{}}

    \PY{c+c1}{\PYZsh{}\PYZsh{}\PYZsh{} START CODE HERE \PYZsh{}\PYZsh{}\PYZsh{}}
    \PY{c+c1}{\PYZsh{} mean center the data}
    \PY{n}{X\PYZus{}demeaned} \PY{o}{=} \PY{n}{X} \PY{o}{\PYZhy{}} \PY{n}{np}\PY{o}{.}\PY{n}{mean}\PY{p}{(}\PY{n}{X}\PY{p}{,}\PY{n}{axis} \PY{o}{=} \PY{l+m+mi}{0}\PY{p}{)}

    \PY{c+c1}{\PYZsh{} calculate the covariance matrix}
    \PY{n}{covariance\PYZus{}matrix} \PY{o}{=} \PY{n}{np}\PY{o}{.}\PY{n}{cov}\PY{p}{(}\PY{n}{X\PYZus{}demeaned}\PY{p}{,} \PY{n}{rowvar} \PY{o}{=} \PY{k+kc}{False}\PY{p}{)}

    \PY{c+c1}{\PYZsh{} calculate eigenvectors \PYZam{} eigenvalues of the covariance matrix}
    \PY{n}{eigen\PYZus{}vals}\PY{p}{,} \PY{n}{eigen\PYZus{}vecs} \PY{o}{=} \PY{n}{np}\PY{o}{.}\PY{n}{linalg}\PY{o}{.}\PY{n}{eigh}\PY{p}{(}\PY{n}{covariance\PYZus{}matrix}\PY{p}{,} \PY{n}{UPLO}\PY{o}{=}\PY{l+s+s1}{\PYZsq{}}\PY{l+s+s1}{L}\PY{l+s+s1}{\PYZsq{}}\PY{p}{)}

    \PY{c+c1}{\PYZsh{} sort eigenvalue in increasing order (get the indices from the sort)}
    \PY{n}{idx\PYZus{}sorted} \PY{o}{=} \PY{n}{np}\PY{o}{.}\PY{n}{argsort}\PY{p}{(}\PY{n}{eigen\PYZus{}vals}\PY{p}{)}
    
    \PY{c+c1}{\PYZsh{} reverse the order so that it\PYZsq{}s from highest to lowest.}
    \PY{n}{idx\PYZus{}sorted\PYZus{}decreasing} \PY{o}{=} \PY{n}{idx\PYZus{}sorted}\PY{p}{[}\PY{p}{:}\PY{p}{:}\PY{o}{\PYZhy{}}\PY{l+m+mi}{1}\PY{p}{]}

    \PY{c+c1}{\PYZsh{} sort the eigen values by idx\PYZus{}sorted\PYZus{}decreasing}
    \PY{n}{eigen\PYZus{}vals\PYZus{}sorted} \PY{o}{=} \PY{n}{eigen\PYZus{}vals}\PY{p}{[}\PY{n}{idx\PYZus{}sorted\PYZus{}decreasing}\PY{p}{]}

    \PY{c+c1}{\PYZsh{} sort eigenvectors using the idx\PYZus{}sorted\PYZus{}decreasing indices}
    \PY{n}{eigen\PYZus{}vecs\PYZus{}sorted} \PY{o}{=} \PY{n}{eigen\PYZus{}vecs}\PY{p}{[}\PY{p}{:}\PY{p}{,}\PY{n}{idx\PYZus{}sorted\PYZus{}decreasing}\PY{p}{]}

    \PY{c+c1}{\PYZsh{} select the first n eigenvectors (n is desired dimension}
    \PY{c+c1}{\PYZsh{} of rescaled data array, or n\PYZus{}components)}
    \PY{n}{eigen\PYZus{}vecs\PYZus{}subset} \PY{o}{=} \PY{n}{eigen\PYZus{}vecs\PYZus{}sorted}\PY{p}{[}\PY{p}{:}\PY{p}{,}\PY{l+m+mi}{0}\PY{p}{:}\PY{n}{n\PYZus{}components}\PY{p}{]}

    \PY{c+c1}{\PYZsh{} transform the data by multiplying the transpose of the eigenvectors with the transpose of the de\PYZhy{}meaned data}
    \PY{c+c1}{\PYZsh{} Then take the transpose of that product.}
    \PY{n}{X\PYZus{}reduced} \PY{o}{=} \PY{n}{np}\PY{o}{.}\PY{n}{dot}\PY{p}{(}\PY{n}{eigen\PYZus{}vecs\PYZus{}subset}\PY{o}{.}\PY{n}{transpose}\PY{p}{(}\PY{p}{)}\PY{p}{,}\PY{n}{X\PYZus{}demeaned}\PY{o}{.}\PY{n}{transpose}\PY{p}{(}\PY{p}{)}\PY{p}{)}\PY{o}{.}\PY{n}{transpose}\PY{p}{(}\PY{p}{)}

    \PY{c+c1}{\PYZsh{}\PYZsh{}\PYZsh{} END CODE HERE \PYZsh{}\PYZsh{}\PYZsh{}}

    \PY{k}{return} \PY{n}{X\PYZus{}reduced}
\end{Verbatim}
\end{tcolorbox}

    \begin{tcolorbox}[breakable, size=fbox, boxrule=1pt, pad at break*=1mm,colback=cellbackground, colframe=cellborder]
\prompt{In}{incolor}{53}{\boxspacing}
\begin{Verbatim}[commandchars=\\\{\}]
\PY{c+c1}{\PYZsh{} Testing your function}
\PY{n}{np}\PY{o}{.}\PY{n}{random}\PY{o}{.}\PY{n}{seed}\PY{p}{(}\PY{l+m+mi}{1}\PY{p}{)}
\PY{n}{X} \PY{o}{=} \PY{n}{np}\PY{o}{.}\PY{n}{random}\PY{o}{.}\PY{n}{rand}\PY{p}{(}\PY{l+m+mi}{3}\PY{p}{,} \PY{l+m+mi}{10}\PY{p}{)}
\PY{n}{X\PYZus{}reduced} \PY{o}{=} \PY{n}{compute\PYZus{}pca}\PY{p}{(}\PY{n}{X}\PY{p}{,} \PY{n}{n\PYZus{}components}\PY{o}{=}\PY{l+m+mi}{2}\PY{p}{)}
\PY{n+nb}{print}\PY{p}{(}\PY{l+s+s2}{\PYZdq{}}\PY{l+s+s2}{Your original matrix was }\PY{l+s+s2}{\PYZdq{}} \PY{o}{+} \PY{n+nb}{str}\PY{p}{(}\PY{n}{X}\PY{o}{.}\PY{n}{shape}\PY{p}{)} \PY{o}{+} \PY{l+s+s2}{\PYZdq{}}\PY{l+s+s2}{ and it became:}\PY{l+s+s2}{\PYZdq{}}\PY{p}{)}
\PY{n+nb}{print}\PY{p}{(}\PY{n}{X\PYZus{}reduced}\PY{p}{)}
\end{Verbatim}
\end{tcolorbox}

    \begin{Verbatim}[commandchars=\\\{\}]
Your original matrix was (3, 10) and it became:
[[ 0.43437323  0.49820384]
 [ 0.42077249 -0.50351448]
 [-0.85514571  0.00531064]]
    \end{Verbatim}

    \begin{tcolorbox}[breakable, size=fbox, boxrule=1pt, pad at break*=1mm,colback=cellbackground, colframe=cellborder]
\prompt{In}{incolor}{48}{\boxspacing}
\begin{Verbatim}[commandchars=\\\{\}]
\PY{c+c1}{\PYZsh{} Test your function}
\PY{n}{w3\PYZus{}unittest}\PY{o}{.}\PY{n}{test\PYZus{}compute\PYZus{}pca}\PY{p}{(}\PY{n}{compute\PYZus{}pca}\PY{p}{)}
\end{Verbatim}
\end{tcolorbox}

    \begin{Verbatim}[commandchars=\\\{\}]
\textcolor{ansi-green-intense}{ All tests passed}
    \end{Verbatim}

    \textbf{Expected Output:}

Your original matrix was: (3,10) and it became:

0.43437323

\begin{verbatim}
    <td>
        0.49820384
        </td>
</tr>
<tr>
    <td>
        0.42077249
        </td>
            <td>
       -0.50351448
        </td>
</tr>
<tr>
    <td>
        -0.85514571
        </td>
            <td>
       0.00531064
        </td>
</tr>
\end{verbatim}

Now you will use your pca function to plot a few words we have chosen
for you. You will see that similar words tend to be clustered near each
other. Sometimes, even antonyms tend to be clustered near each other.
Antonyms describe the same thing but just tend to be on the other end of
the scale They are usually found in the same location of a sentence,
have the same parts of speech, and thus when learning the word vectors,
you end up getting similar weights. In the next week we will go over how
you learn them, but for now let's just enjoy using them.

\textbf{Instructions:} Run the cell below.

    \begin{tcolorbox}[breakable, size=fbox, boxrule=1pt, pad at break*=1mm,colback=cellbackground, colframe=cellborder]
\prompt{In}{incolor}{49}{\boxspacing}
\begin{Verbatim}[commandchars=\\\{\}]
\PY{n}{words} \PY{o}{=} \PY{p}{[}\PY{l+s+s1}{\PYZsq{}}\PY{l+s+s1}{oil}\PY{l+s+s1}{\PYZsq{}}\PY{p}{,} \PY{l+s+s1}{\PYZsq{}}\PY{l+s+s1}{gas}\PY{l+s+s1}{\PYZsq{}}\PY{p}{,} \PY{l+s+s1}{\PYZsq{}}\PY{l+s+s1}{happy}\PY{l+s+s1}{\PYZsq{}}\PY{p}{,} \PY{l+s+s1}{\PYZsq{}}\PY{l+s+s1}{sad}\PY{l+s+s1}{\PYZsq{}}\PY{p}{,} \PY{l+s+s1}{\PYZsq{}}\PY{l+s+s1}{city}\PY{l+s+s1}{\PYZsq{}}\PY{p}{,} \PY{l+s+s1}{\PYZsq{}}\PY{l+s+s1}{town}\PY{l+s+s1}{\PYZsq{}}\PY{p}{,}
         \PY{l+s+s1}{\PYZsq{}}\PY{l+s+s1}{village}\PY{l+s+s1}{\PYZsq{}}\PY{p}{,} \PY{l+s+s1}{\PYZsq{}}\PY{l+s+s1}{country}\PY{l+s+s1}{\PYZsq{}}\PY{p}{,} \PY{l+s+s1}{\PYZsq{}}\PY{l+s+s1}{continent}\PY{l+s+s1}{\PYZsq{}}\PY{p}{,} \PY{l+s+s1}{\PYZsq{}}\PY{l+s+s1}{petroleum}\PY{l+s+s1}{\PYZsq{}}\PY{p}{,} \PY{l+s+s1}{\PYZsq{}}\PY{l+s+s1}{joyful}\PY{l+s+s1}{\PYZsq{}}\PY{p}{]}

\PY{c+c1}{\PYZsh{} given a list of words and the embeddings, it returns a matrix with all the embeddings}
\PY{n}{X} \PY{o}{=} \PY{n}{get\PYZus{}vectors}\PY{p}{(}\PY{n}{word\PYZus{}embeddings}\PY{p}{,} \PY{n}{words}\PY{p}{)}

\PY{n+nb}{print}\PY{p}{(}\PY{l+s+s1}{\PYZsq{}}\PY{l+s+s1}{You have 11 words each of 300 dimensions thus X.shape is:}\PY{l+s+s1}{\PYZsq{}}\PY{p}{,} \PY{n}{X}\PY{o}{.}\PY{n}{shape}\PY{p}{)}
\end{Verbatim}
\end{tcolorbox}

    \begin{Verbatim}[commandchars=\\\{\}]
You have 11 words each of 300 dimensions thus X.shape is: (11, 300)
    \end{Verbatim}

    \begin{tcolorbox}[breakable, size=fbox, boxrule=1pt, pad at break*=1mm,colback=cellbackground, colframe=cellborder]
\prompt{In}{incolor}{51}{\boxspacing}
\begin{Verbatim}[commandchars=\\\{\}]
\PY{c+c1}{\PYZsh{} We have done the plotting for you. Just run this cell.}
\PY{n}{result} \PY{o}{=} \PY{n}{compute\PYZus{}pca}\PY{p}{(}\PY{n}{X}\PY{p}{,} \PY{l+m+mi}{2}\PY{p}{)}
\PY{n}{plt}\PY{o}{.}\PY{n}{scatter}\PY{p}{(}\PY{n}{result}\PY{p}{[}\PY{p}{:}\PY{p}{,} \PY{l+m+mi}{0}\PY{p}{]}\PY{p}{,} \PY{n}{result}\PY{p}{[}\PY{p}{:}\PY{p}{,} \PY{l+m+mi}{1}\PY{p}{]}\PY{p}{)}
\PY{k}{for} \PY{n}{i}\PY{p}{,} \PY{n}{word} \PY{o+ow}{in} \PY{n+nb}{enumerate}\PY{p}{(}\PY{n}{words}\PY{p}{)}\PY{p}{:}
    \PY{n}{plt}\PY{o}{.}\PY{n}{annotate}\PY{p}{(}\PY{n}{word}\PY{p}{,} \PY{n}{xy}\PY{o}{=}\PY{p}{(}\PY{n}{result}\PY{p}{[}\PY{n}{i}\PY{p}{,} \PY{l+m+mi}{0}\PY{p}{]} \PY{o}{\PYZhy{}} \PY{l+m+mf}{0.05}\PY{p}{,} \PY{n}{result}\PY{p}{[}\PY{n}{i}\PY{p}{,} \PY{l+m+mi}{1}\PY{p}{]} \PY{o}{+} \PY{l+m+mf}{0.1}\PY{p}{)}\PY{p}{)}

\PY{n}{plt}\PY{o}{.}\PY{n}{show}\PY{p}{(}\PY{p}{)}
\end{Verbatim}
\end{tcolorbox}

    \begin{center}
    \adjustimage{max size={0.9\linewidth}{0.9\paperheight}}{output_45_0.png}
    \end{center}
    { \hspace*{\fill} \\}
    
    \textbf{What do you notice?}

The word vectors for gas, oil and petroleum appear related to each
other, because their vectors are close to each other. Similarly, sad,
joyful and happy all express emotions, and are also near each other.


    % Add a bibliography block to the postdoc
    
    
    
\end{document}
